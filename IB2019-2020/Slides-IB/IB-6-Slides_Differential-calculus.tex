\documentclass{beamer}
\usepackage{geometry}
\usepackage[english]{babel}
\usepackage[utf8]{inputenc}
\usepackage{amsmath}
\usepackage{amsfonts}
\usepackage{amssymb}
\usepackage{tikz}
\usepackage{graphicx}
\usepackage{venndiagram}

%\usepackage{pgfplots}
%\pgfplotsset{width=10cm,compat=1.9}
%\usepackage{pgfplotstable}

\setlength{\headheight}{26pt}%doesn't seem to fix warning

\usepackage{fancyhdr}
\pagestyle{fancy}
\fancyhf{}

\lhead{\small{BECA / Dr. Huson / IB Math Unit 6 - Differential Calculus}}

\renewcommand{\headrulewidth}{0pt}

\title{Mathematics Class Slides}
\subtitle{Bronx Early College Academy}
\author{Chris Huson}
\date{2 March 2020}

\begin{document}
\frame{\titlepage}
\section[Outline]{}
\frame{\tableofcontents}

\section{6.1 Intro to calculus \hfill Wednesday 26 February}
\frame
{
  \frametitle{GQ: How do we graph tangents to functions?}
  \framesubtitle{CCSS: HSF.IF.C8.A Understanding rate of change \hfill \alert{6.1 Wednesday 26 February}}
  \begin{block}{Do Now: Linear equation practice}
    \begin{enumerate}
      \item Write down the equation of the line through $(2,-3)$ with slope $m=2$
      \item Write down the equation of the line through $(-1,0)$ perpendicular to the line with slope $m=2$
      \item Sketch the function $f(x)=x^2+1$ and $g(x)=-2x$ on the same axes
    \end{enumerate}
    \end{block}
    Lesson: Polynomial function terminology, the power rule
    Homework: Deltamath calculus practice
    }

  \section{6.2 Intro to calculus \hfill Thursday 27 February}
  \frame
  {
    \frametitle{GQ: How do we graph tangents to functions?}
    \framesubtitle{CCSS: HSF.IF.C8.A Understanding rate of change \hfill \alert{6.2  Thursday 27 February}}
    \begin{block}{Do Now: Linear equation practice}
      \begin{enumerate}
        \item Write down the equation of the line through $(2,-3)$ with slope $m=2$
        \item Write down the equation of the line through $(-1,0)$ perpendicular to the line with slope $m=2$
        \item Sketch the function $f(x)=x^2+1$ and $g(x)=-2x$ on the same axes
      \end{enumerate}
      \end{block}
      Lesson: Polynomial function terminology, the power rule
      Homework: Deltamath calculus practice
      }

  \section{6.3 Power rule - Deltamath practice \hfill Friday 28 February}
  \frame
  {
    \frametitle{GQ: How do we graph tangents to functions?}
    \framesubtitle{CCSS: HSF.IF.C8.A Understanding rate of change \hfill \alert{6.3  Friday 28 February}}
    \begin{block}{Do Now: Differentiation of polynomials practice}
      \begin{enumerate}
        \item Find the derivative of $h(x)=x^2+5$
        \item Given $g(x)=x^3 + 12x^2-1$. Find $g'(x)$
        \item Given $f(x)=x^3 + 7$. 
        \begin{enumerate}
          \item Find $f(-1)$
          \item Find $f'(x)$
          \item Find the derivative of $f$ when $x=-1$.
          \item Write down the equation of the tangent to $f$ at $x=-1$
        \end{enumerate}
      \end{enumerate}
      \end{block}
      Lesson: Apply the power rule for taking derivatives\\
      Classwork: Deltamath calculus practice (finish for homework)
      }

  \section{6.4 Review calculator functions \hfill Monday 2 March}
  \frame
  {
    \frametitle{GQ: How do we graph tangents to functions?}
    \framesubtitle{CCSS: HSF.IF.C8.A Understanding rate of change \hfill \alert{6.4 Monday 2 March}}
    \begin{block}{Do Now: $f(x)=x^3-5x^2+5x+2$}
      \begin{enumerate}
        \item What point does $f$ go through when $x=1$?
        \item Find $f'(x)$
        \item What is the slope of the line tangent to the function when $x=1$?
        \item Write down the equation of the tangent to $f$ at $x=1$
        \item Graph the function and its tangent at $x=1$ on your calculator.
        \item Sketch the graph.
      \end{enumerate}
      \end{block}
      Lesson: Using the Casio to calculate derivatives\\
      Classwork: Practice calculator functions (\alert{pop quiz warning!})
      }
      
      \frame
      {
        \frametitle{GQ: How do we graph tangents to functions?}
        \framesubtitle{CCSS: HSF.IF.C8.A Understanding rate of change \hfill \alert{6.4 Monday 2 March}}
        {Calculator practice}
          \begin{enumerate}
            \item Find the solutions for the system, $f(x)=g(x)$.\\
                $f(x)=-2x^2+5x+7$ \qquad $g(x)=-2x+4$ \vspace{0.3cm}
                % x=-0.386, 3.886
            \item Perform a linear regression on the data, finding $y=ax+b$. 
              \begin{center}
              \begin{tabular}{|l|c|c|c|c|c|c|c|}
                  \hline
                  $x$ & 17 & 18 & 17 & 19 & 23 & 15 & 16 \\ 
                  \hline 
                  $y$ & 71.1 & 78.6 & 69.2 & 71.2 & 80.5 & 55.7 & 58.4 \\ 
                  \hline 
                  \end{tabular}
              \end{center}
              \begin{enumerate}
                  \item Write down the value of $a$, $b$. %a=2.90629, b=17.3447
                  \item Write down the correlation coefficient $r$.  %r=0.81267128
                  \item Use your regression line to estimate $y$ for $x=22$.  %y=81.2832
              \end{enumerate}
            \item $a=12.3$, $b=14.7$, $\theta = 71^\circ$. Find the third side length, $c$.
            \item $a=11.4$, $b=17.1$, $c=16.0$.
          \end{enumerate}
          }

\section{6.5 Quiz calculator functions, Deltamath calculus \hfill Tuesday 3 March}
  \frame
  {
    \frametitle{GQ: How do we graph tangents to functions?}
    \framesubtitle{CCSS: HSF.IF.C8.A Understanding rate of change \hfill \alert{6.5 Tuesday 3 March}}
    \begin{block}{Do Now Quiz: Calculator functions D}
      \begin{enumerate}
        \item Solving systems of equations with handheld technology
        \item Linear regression
        \item Using the Casio to calculate derivatives
      \end{enumerate}
      \end{block}
      Classwork: Deltamath calculus Equations of tangent lines \\
      Homework: Complete Deltamath
      }

\section{6.6 Solve for extrema with derivative \hfill Thursday 5 March}
  \frame
  {
    \frametitle{GQ: How do we solve for extrema?}
    \framesubtitle{CCSS: HSF.IF.C8.A Understanding rate of change \hfill \alert{6.6 Thursday 5 March}}
    \begin{block}{Do Now: Calculator functions E}
      \begin{enumerate}
        \item Solving systems of equations with handheld technology
        \item Statistical summary of frequency table data
        \item Using the Casio to calculate derivatives
      \end{enumerate}
      \end{block}
      Classwork: Solving for horizontal tangent lines\\
      Homework: Practice calculator functions (\alert{quiz tomorrow!})
      }

\section{6.7 Solve for extrema with derivative \hfill Friday 6 March}
  \frame
  {
    \frametitle{GQ: How do we solve for extrema?}
    \framesubtitle{CCSS: HSF.IF.C8.A Understanding rate of change \hfill \alert{6.7 Friday 6 March}}
    \begin{block}{Do Now Quiz: Calculator functions F}
      \begin{enumerate}
        \item Tangent to a polynomial function
        \item Solving systems of equations with handheld technology
        \item Statistical summary of frequency table data
        \item Using the Casio to calculate derivatives
      \end{enumerate}
      \end{block}
      Classwork: Solving for horizontal tangent lines\\
      Homework: Practice calculator functions
      }

\section{6.8 Solve for extrema with derivative \hfill Monday 9 March}
  \frame
  {
    \frametitle{GQ: How do we solve for extrema?}
    \framesubtitle{CCSS: HSF.IF.C8.A Understanding rate of change \hfill \alert{6.8 Monday 9 March}}
    \begin{block}{Do Now: Calculator functions G}
      \begin{enumerate}
        \item Tangent to a polynomial function
        \item Solving systems of equations with handheld technology
        \item Complex calculations: Law of cosine applications
        \item Using the Casio to calculate derivatives
      \end{enumerate}
      \end{block}
      Lesson: The derivative of a fractional or negative exponent \\ 
      Solving for horizontal tangent lines; polynomial end behavior, roots \\
      Homework: Deltamath differentiation practice \\ 
      Practice calculator functions (\alert{quiz tomorrow?})
      }

\section{6.9 DN Quiz, Gradescope review \hfill Tuesday 10 March}
  \frame
  {
    \frametitle{GQ: How do we solve for extrema?}
    \framesubtitle{CCSS: HSF.IF.C8.A Understanding rate of change \hfill \alert{6.9 Tuesday 10 March}}
    \begin{block}{Do Now Quiz: Calculator functions H}
      \begin{enumerate}
        \item Tangent to a polynomial function
        \item Solving systems of equations with handheld technology
        \item Complex calculations: Law of cosine applications
      \end{enumerate}
      \end{block}
      Lesson: Polynomial end behavior, roots \\
      Classwork: Deltamath differentiation practice \\ 
      Homework: complete Deltamath problem set
      }

\section{6.10 Derivatives of exp, trig function \hfill Wednesday 11 March}
  \frame
  {
    \frametitle{GQ: How do we differentiate exponential \& trig functions?}
    \framesubtitle{CCSS: HSF.IF.C8.A Understanding rate of change \hfill \alert{6.10 Wednesday 11 March}}
    \begin{block}{Do Now: Calculator functions J}
      \begin{enumerate}
        \item Tangent to a negative power of $x$
        \item Solving systems of equations with handheld technology
        \item Applications of sine function, radians and degrees
      \end{enumerate}
      \end{block}
      Lesson: Polynomial end behavior, roots \\
      Classwork: Solving for extrema algebraically \\ 
      Homework: Study for quiz, complete Deltamath
      }
                  
\end{document}

