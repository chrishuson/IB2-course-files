\documentclass{beamer}
\usepackage{geometry}
\usepackage[english]{babel}
\usepackage[utf8]{inputenc}
\usepackage{amsmath}
\usepackage{amsfonts}
\usepackage{amssymb}
\usepackage{tikz}
\usepackage{graphicx}
\usepackage{venndiagram}

%\usepackage{pgfplots}
%\pgfplotsset{width=10cm,compat=1.9}
%\usepackage{pgfplotstable}

\setlength{\headheight}{26pt}%doesn't seem to fix warning

\usepackage{fancyhdr}
\pagestyle{fancy}
\fancyhf{}

\lhead{\small{BECA / Huson / IB Math Unit 7 - Sequences \& exponential functions}}

\renewcommand{\headrulewidth}{0pt}

\title{Mathematics Class Slides}
\subtitle{Bronx Early College Academy}
\author{Chris Huson}
\date{24 March 2020}

\begin{document}
\frame{\titlepage}
\section[Outline]{}
\frame{\tableofcontents}

\section{7.1 Online startup - exponents \hfill Tuesday 24 March}
  \frame
  {
    \frametitle{GQ: How do we use exponents (and logs)?}
    \framesubtitle{CCSS: HSF.IF.C8.A Understanding rate of change \hfill \alert{7.1 Tuesday 24 March}}
    \begin{block}{Do Now: Welcome to Beca Online!}
      \begin{itemize}
        \item Complete the attendance question in Google Classroom
        \item Write in your notebook my new email, chuson@beca324.org
        \item Complete the G-Classroom "Do Now" questions
      \end{itemize}
  
      \end{block}
      BECA Online expectations \\[0.25cm]
      Lesson: \\
      Applications of exponential functions: log plots \\
      Exit note: complete G-Classroom checkin survey \\
      Homework: Kognity assignment, due by 10:00pm Thursday
      }

\section{7.2 Arithmetic sequences \hfill Friday 26 March}
\frame
{
  \frametitle{GQ: How do we model sequences?}
  \framesubtitle{CCSS: HSF.IF.C8.A Understanding rate of change \hfill \alert{7.2 Friday 26 March}}
  \begin{block}{Do Now: Study the COVID-19 Expert Forecast for the U.S.}
    \begin{itemize}
      \item When will the number of hospitalizations peak, in their opinion?
      \item How much worse than the flu will this be in terms of annual deaths?
      \item Give a short written answer by private Zoom chat
    \end{itemize}

    \end{block}
    Kognity textbook feedback \\[0.25cm]
    Lesson: \\
    Analyzing the pandemic as frequencies, probabilities, \& sequences \\
    Breakouts: infections, ER seasonality, hospitalizations, deaths \\
    Project: Pandemic analysis, due 10:00pm Tuesday
    }

\frame
{
  \frametitle{GQ: How do we model sequences?}
  \framesubtitle{CCSS: HSF.IF.C8.A Understanding rate of change \hfill \alert{7.2 Friday 26 March}}

  \begin{block}{Project: Pandemic analysis}
    \begin{itemize}
      \item Organize as an exploration: intro, body, conclusion, reflection, engagement
      \item ``The aim of this exploration is to understand COVID-19 in NY State as a exponential or geometric process''
      \item Select one metric: infections, hospitalizations, or deaths
      \item Use a table(s) and/or chart(s) (pie, bar) as background
      \item Fit an exponential function / geometric sequence to the NYC COVID-Tracking data
    \end{itemize}
    \end{block}
    }

\section{7.4 Arithmetic sequences and series \hfill Monday 6 April}
\frame
{
  \frametitle{GQ: How do we model sequences?}
  \framesubtitle{CCSS: HSF.IF.C8.A Understanding rate of change \hfill \alert{7.4 Monday 6 April}}
  \begin{block}{Do Now: Reported NY State COVID-19 deaths }
    \begin{itemize}
      \item On Thursday and Friday last week, NY State saw 562 and 630 fatalities respectively. Find the percent change.
      \item Identify the key features of the Desmos model and graph (for discussion).
    \end{itemize}

    \end{block}
    Kognity textbook arithmetic sequences and series problems \\[0.25cm]
    Lesson: \\
    Analyzing the pandemic as geometric sequence \\
    Short writing exercise (Google docs)
    }

\frame
{
  \frametitle{Arithmetic sequences and series}
  \[u_n=u_1+(n-1)d\]
  \[S_n=\frac{n}{2}(2u_1+(n-1)d); \; S_n=\frac{n}{2}(u_1+u_n)\]
  \begin{enumerate}
    \item In an arithmetic sequence, the first term is 3 and the second term is 9.
      \begin{enumerate}[(a)]
          \item Find the common difference. \vspace{0.5cm}
          \item Find the eighth term.\vspace{1cm}
          \item Find the sum of the first eight terms of the sequence.
      \end{enumerate}
    \end{enumerate}\vspace{5cm}
    }

\frame
{
  \frametitle{Arithmetic sequences and series}
  \begin{enumerate}
    \item The first three terms of an arithmetic sequence are $u_1=16$, $u_2=13.5$, and $u_3=11$.
      \begin{enumerate}[(a)]
          \item Find the common difference. \vspace{1cm}
          \item Find the eleventh term.\vspace{2cm}
          \item Given that the $k$th term of the sequence, $u_k=1$. Find $k$.
      \end{enumerate}
    \end{enumerate}\vspace{8cm}
    }

\frame
{
  \frametitle{Arithmetic sequences and series}
  \begin{enumerate}
    \item In an arithmetic sequence, $u_2=14$ and $u_5=23$.
      \begin{enumerate}[(a)]
          \item Find the common difference and the first term. \vspace{2.5cm}
          \item The sum of the first $k$ terms of the sequence $S_k=207$. Find $k$.
      \end{enumerate}
    \end{enumerate}\vspace{5cm}
    }

\section{7.5 Geometric sequences \hfill Monday 27 April}
\frame
{
  \frametitle{GQ: How do we model geometric sequences?}
  \framesubtitle{CCSS: HSF.IF.C8.A Understanding rate of change \hfill \alert{7.5 Monday 27 April}}
  \begin{block}{Do Now: Arithmetic sequence and series}
      Given an arithmetic sequence, 19, 12, 5, ...
        \begin{enumerate}[(a)]
            \item Find the common difference.
            \item Find the 14th term.
            \item Find the sum of the first 14 terms of the sequence.
        \end{enumerate}
    \end{block}

    Lesson: \\
    Geometric sequences review \\
    Deltamath practice in class. Written problem set due 10:00 PM\\[0.25cm]
    Kognity textbook compound interest 
    }

\frame
{
  \frametitle{Geometric sequences and series: $u_n=u_1 \times r^{n-1}$}
  \begin{enumerate}
    \item Find the common ratio of each geometric sequence
      \begin{enumerate}[(a)]
          \item $u_1=6, u_2=2$ \vspace{1.5cm}
          \item $2.5, -7.5, 22.5$\vspace{2cm}
          \item $u_1=16, u_5=81$ \vspace{2cm}
      \end{enumerate}
    \end{enumerate}\vspace{5cm}
    }
    
\frame
{
  \frametitle{Geometric sequences and series: $u_n=u_1 \times r^{n-1}$}

  \begin{enumerate}
    \item Given the geometric sequence, find the $7$th term \\
          $\displaystyle \frac{1}{3}, -\frac{1}{9}, \frac{1}{27}, \dots$;  $u_7=?$ \vspace{2cm}
    \item $u_1=31.25, r=\frac{4}{5}$, find $u_5$ \vspace{2cm}
    \end{enumerate}\vspace{5cm}
    }

\frame
{
  \frametitle{Geometric sequences and series: $u_n=u_1 \times r^{n-1}$}

  \begin{enumerate}
    \item Given the geometric sequence, find the 1st term and the common ratio \\
         $u_2=4.8, u_5=16.2$ \vspace{2cm}
    \end{enumerate}\vspace{5cm}
    }

\frame
{
  \frametitle{Geometric sequences and series}
  \[u_n=u_1 \times r^{n-1}\]
  \[S_n=\frac{u_1  (r^{n}-1)}{r-1} =\frac{u_1  (1-r^{n})}{1-r} \]
    }
\end{document}


