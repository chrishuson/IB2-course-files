\documentclass{beamer}
\usepackage{geometry}
\usepackage[english]{babel}
\usepackage[utf8]{inputenc}
\usepackage{amsmath}
\usepackage{amsfonts}
\usepackage{amssymb}
\usepackage{tikz}
\usepackage{graphicx}
\usepackage{venndiagram}

%\usepackage{pgfplots}
%\pgfplotsset{width=10cm,compat=1.9}
%\usepackage{pgfplotstable}

\setlength{\headheight}{26pt}%doesn't seem to fix warning

\usepackage{fancyhdr}
\pagestyle{fancy}
\fancyhf{}

\lhead{\small{BECA / Dr. Huson / IB Math Unit 5 - Polynomial functions}}

\renewcommand{\headrulewidth}{0pt}

\title{Mathematics Class Slides}
\subtitle{Bronx Early College Academy}
\author{Chris Huson}
\date{28 January 2020}

\begin{document}
\frame{\titlepage}
\section[Outline]{}
\frame{\tableofcontents}

\section{5.1 Solving quadratic equations \hfill Tuesday 28 January}
\frame
{
  \frametitle{GQ: How do we factor and solve quadratic equations?}
  \framesubtitle{CCSS: HSF.IF.C8.A Factor quadratic functions to show zeros \hfill \alert{5.1 Tuesday 28 January}}

  \begin{block}{Do Now: Quadratic functions}
    \begin{enumerate}
      \item Function operations, composition, inverse
      \item Interpreting quadratic functions in vertex form
      \item Solving graphical situations
    \end{enumerate}
    \end{block}
    Exam makeup: Dayna, Monica, Wendy\\
    \qquad (collect take-homes from vector group)\\
    Lesson: Factored form; ``solutions,'' ``roots,'' ``zeros,'' $x$-intercepts\\
    \qquad Quadratic functions pp. 233-236 \\ \smallskip
    Homework: Deltamath Factoring (continue textbook problems)
    }

\section{5.2 Quadratic functions where $a \neq 1$ \hfill Wednesday 29 January}
\frame
{
  \frametitle{GQ: How do we graph quadratics when $a \neq 1$?}
  \framesubtitle{CCSS: HSF.IF.C8.A Factor quadratic functions to show zeros \hfill \alert{5.2 Wednesday 29 January}}

  \begin{block}{Do Now: Solving quadratic equations}
    \begin{enumerate}
      \item Factoring quadratic functions
      \item Interpreting quadratic functions in vertex form
      \item Solving graphical situations
    \end{enumerate}
    \end{block}
    Lesson: Vertical stretch, the $a$ parameter\\
    \qquad The axis of symmetry \\ \smallskip
    Homework: Deltamath Completing the square, due Thursday (continue textbook problems)
    }

\section{5.3 Solving quadratic equations \hfill Thursday 30 January}
\frame
{
  \frametitle{GQ: How do we factor by ``completing the square''?}
  \framesubtitle{CCSS: HSF.IF.C8.A Complete the square of quadratic functions \hfill \alert{5.3 Thursday 30 January}}

  \begin{block}{Do Now: Solving quadratic equations}
    \begin{enumerate}
      \item Graphing quadratic functions
      \item Factoring quadratic functions
      \item Solving graphical situations
    \end{enumerate}
    \end{block}
    Lesson: Completing the square, converting to vertex form\\ \smallskip
    Homework: Deltamath due Sunday (continue textbook problems)
    }

\section{5.4 Quadratic formula \hfill Friday 31 January}
\frame
{
  \frametitle{GQ: How do we use the quadratic formula?}
  \framesubtitle{CCSS: HSF.IF.C8.A Complete the square of quadratic functions \hfill \alert{5.4 Friday 31 January}}

  \begin{block}{Do Now: Solving quadratic equations}
    \begin{enumerate}
      \item Completing the square
      \item Graphing quadratic functions
      \item Factoring quadratic functions
    \end{enumerate}
    \end{block}
    Lesson: Completing the square, converting to vertex form\\ \smallskip
    Homework: Textbook exercises 6D (Deltamath)
    }

\section{5.4 Quadratic formula \hfill Monday 3 February}
\frame
{
  \frametitle{GQ: How do we use the quadratic formula?}
  \framesubtitle{CCSS: HSF.IF.C8.A Complete the square of quadratic functions \hfill \alert{5.4 Monday 3 February}}

  \begin{block}{Do Now: Solving quadratic equations}
    \begin{enumerate}
      \item Completing the square
      \item Graphing quadratic functions
      \item Factoring quadratic functions
    \end{enumerate}
    \end{block}
    Lesson: Completing the square, converting to vertex form\\ \smallskip
    Homework: Textbook exercises 6D (Deltamath)
    }

\end{document}
