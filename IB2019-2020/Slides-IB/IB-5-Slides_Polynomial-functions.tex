\documentclass{beamer}
\usepackage{geometry}
\usepackage[english]{babel}
\usepackage[utf8]{inputenc}
\usepackage{amsmath}
\usepackage{amsfonts}
\usepackage{amssymb}
\usepackage{tikz}
\usepackage{graphicx}
\usepackage{venndiagram}

%\usepackage{pgfplots}
%\pgfplotsset{width=10cm,compat=1.9}
%\usepackage{pgfplotstable}

\setlength{\headheight}{26pt}%doesn't seem to fix warning

\usepackage{fancyhdr}
\pagestyle{fancy}
\fancyhf{}

\lhead{\small{BECA / Dr. Huson / IB Math Unit 5 - Polynomial functions}}

\renewcommand{\headrulewidth}{0pt}

\title{Mathematics Class Slides}
\subtitle{Bronx Early College Academy}
\author{Chris Huson}
\date{28 January 2020}

\begin{document}
\frame{\titlepage}
\section[Outline]{}
\frame{\tableofcontents}

\section{5.1 Solving quadratic equations \hfill Tuesday 28 January}
\frame
{
  \frametitle{GQ: How do we factor and solve quadratic equations?}
  \framesubtitle{CCSS: HSF.IF.C8.A Factor quadratic functions to show zeros \hfill \alert{5.1 Tuesday 28 January}}

  \begin{block}{Do Now: Quadratic functions}
    \begin{enumerate}
      \item Function operations, composition, inverse
      \item Interpreting quadratic functions in vertex form
      \item Solving graphical situations
    \end{enumerate}
    \end{block}
    Exam makeup: Dayna, Monica, Wendy\\
    \qquad (collect take-homes from vector group)\\
    Lesson: Factored form; ``solutions,'' ``roots,'' ``zeros,'' $x$-intercepts\\
    \qquad Quadratic functions pp. 233-236 \\ \smallskip
    Homework: Deltamath Factoring (continue textbook problems)
    }

\section{5.2 Quadratic functions where $a \neq 1$ \hfill Wednesday 29 January}
\frame
{
  \frametitle{GQ: How do we graph quadratics when $a \neq 1$?}
  \framesubtitle{CCSS: HSF.IF.C8.A Factor quadratic functions to show zeros \hfill \alert{5.2 Wednesday 29 January}}

  \begin{block}{Do Now: Solving quadratic equations}
    \begin{enumerate}
      \item Factoring quadratic functions
      \item Interpreting quadratic functions in vertex form
      \item Solving graphical situations
    \end{enumerate}
    \end{block}
    Lesson: Vertical stretch, the $a$ parameter\\
    \qquad The axis of symmetry \\ \smallskip
    Homework: Deltamath Completing the square, due Thursday (continue textbook problems)
    }

\section{5.3 Completing the square \hfill Friday 31 January}
\frame
{
  \frametitle{GQ: How do we factor by ``completing the square''?}
  \framesubtitle{CCSS: HSF.IF.C8.A Complete the square of quadratic functions \hfill \alert{5.3 Friday 31 January}}

  \begin{block}{Do Now: Solving quadratic equations\\[0.25cm]
    \alert{Classwork counts double while Dr. Huson is out!}}
    \begin{enumerate}
      \item Graphing quadratic functions
      \item Factoring quadratic functions
      \item Solving graphical situations
    \end{enumerate}
    \end{block}
    Please complete Workplace Visits career questionaire. \\ \smallskip
    Lesson: Completing the square, converting to vertex form\\ \smallskip
    Homework: Deltamath due Sunday 10:00pm (continue textbook problems)
    }

\section{5.4 Quadratic formula \hfill Monday 3 February}
\frame
{
  \frametitle{GQ: How do we use the quadratic formula?}
  \framesubtitle{CCSS: HSF.IF.C8.A Complete the square of quadratic functions \hfill \alert{5.4 Monday 3 February}}

  \begin{block}{Do Now: Solving quadratic equations}
    \begin{enumerate}
      \item Completing the square
      \item Graphing quadratic functions
      \item Factoring quadratic functions
    \end{enumerate}
    \end{block}
    Lesson: Applying the quadratic formula\\ \smallskip
    Homework: Deltamath due tomorrow 10:00pm \\ \quad (continue textbook problems)
    }


\section{5.5 Quadratic formula \hfill Tuesday 4 February}
\frame
{
  \frametitle{GQ: How do we use the quadratic formula?}
  \framesubtitle{CCSS: HSF.IF.C8.A Complete the square of quadratic functions \hfill \alert{5.5 Tuesday 4 February}}

  \begin{block}{Do Now: Solving quadratic equations}
    \begin{enumerate}
      \item Completing the square
      \item Graphing quadratic functions
      \item Factoring quadratic functions
    \end{enumerate}
    \end{block}
    Lesson: Applying the quadratic formula\\ \smallskip
    Homework: Deltamath due tomorrow 10:00pm \\ \quad (continue textbook problems)
    }
  
  \frame
  {
    \frametitle{GQ: How do we derive the quadratic formula?}
    \framesubtitle{CCSS: HSF.IF.C8.A Complete the square of quadratic functions \hfill \alert{5.5 Tuesday 4 February}}
  
    \begin{block}{Solve by completing the square}
      \begin{enumerate}
        \item $x^2-8x+5=0$
        \item $x^2+12x+4=0$
        \item $3x^2-12x-7=0$
        \item $-2x^2-12x-9=0$
        \item $4x^2+8x-9=0$
        \item $-3x^2-18x-35=0$
        \item $5x^2+20x+32=0$
      \end{enumerate}
      \end{block}
      Complete the square using arbitrary coefficients
      \[ax^2+bx+c=0\]
      }

\section{5.5 The discriminant \hfill Tuesday 4 February}
\frame
{
  \frametitle{GQ: How do we know the number of solutions of a quadratic function?}
  \framesubtitle{CCSS: HSF.IF.C8.A Complete the square of quadratic functions \hfill \alert{5.5 Tuesday 4 February}}

  \begin{block}{Do Now Quiz: Calculator practice}
    \begin{enumerate}
      \item Graphical solutions of systems of functions
      \item Linear regression
      \item Frequency tables
      \item Complex calculations (e.g. cosine rule)
    \end{enumerate}
    \end{block}
    Lesson: Using the discriminant, $D$ or $\Delta$\\ \smallskip
    Homework: Deltamath due tonight 10:00pm (continue textbook problems)
    }

\section{5.6 The discriminant \hfill Wednesday 5 February}
\frame
{
  \frametitle{GQ: How do we know the number of solutions of a quadratic function?}
  \framesubtitle{CCSS: HSF.IF.C8.A Complete the square of quadratic functions \hfill \alert{5.6 Wednesday 5 Feb}}

  \begin{block}{Do Now: Solving quadratic equations}
    \begin{enumerate}
      \item Graphing quadratic functions
      \item Factoring quadratic functions
      \item Completing the square
    \end{enumerate}
    \end{block}
    Lesson: Completing the square, converting to vertex form\\ \smallskip
    Homework: Deltamath due tonight 10:00pm (continue textbook problems)
    }

\section{5.7 The discriminant \hfill Thursday 6 February}
\frame
{
  \frametitle{GQ: How do we graph polynomial functions?}
  \framesubtitle{CCSS: HSF.IF.C8.A Complete the square of quadratic functions \hfill \alert{5.7 Thursday 6 February}}

  \begin{block}{Do Now: Solving quadratic equations}
    \begin{enumerate}
      \item Graphing quadratic functions
      \item Factoring quadratic functions
      \item Completing the square
    \end{enumerate}
    \end{block}
    Lesson: Features of polynomial graphs, increasing/decreasing \\ \smallskip
    Homework: Deltamath (continue textbook problems)
    }

\section{5.8 The discriminant \hfill Monday 10 February}
\frame
{
  \frametitle{GQ: How do we graph polynomial functions?}
  \framesubtitle{CCSS: HSF.IF.C8.A Complete the square of quadratic functions \hfill \alert{5.8 Monday 10 February}}

  \begin{block}{Do Now: Solving quadratic equations}
    \begin{enumerate}
      \item Graphing quadratic functions
      \item Factoring quadratic functions
      \item Completing the square
    \end{enumerate}
    \end{block}
    Lesson: Features of polynomial graphs, increasing/decreasing \\ \smallskip
    Assessment: calculator practice Problem Set C
    Homework: Deltamath (continue textbook problems)
    }

\section{5.9 Deltamath: polynomials, quadratics \hfill Tuesday 11 February}
  \frame
  {
    \frametitle{GQ: How do we graph polynomial functions?}
    \framesubtitle{CCSS: HSF.IF.C8.A Complete the square of quadratic functions \hfill \alert{5.9 Tuesday 11 February}}

    \begin{block}{Deltamath: Polynomial graph features}
      \begin{enumerate}
        \item Graphing quadratic functions
        \item Factoring quadratic functions
        \item Polynomial graphs, extrema, increasing, decreasing
      \end{enumerate}
      \end{block}
      Lesson: Features of polynomial graphs, increasing/decreasing \\ \smallskip
      Homework: 5.8 CW IB test problems pdf
      }

\section{5.10 Unit review \hfill Wednesday 12 February}
  \frame
  {
    \frametitle{GQ: How do we understand quadratics?}
    \framesubtitle{CCSS: HSF.IF.C8.A Complete the square of quadratic functions \hfill \alert{5.10 Wednesday 12 February}}

    \begin{block}{Do Now: Solving quadratic equations}
      \begin{enumerate}
        \item Graphing quadratic functions
        \item Factoring quadratic functions
        \item Completing the square
      \end{enumerate}
      \end{block}
      Lesson: Quadratics exam review \\ \smallskip
      Homework: Study for \alert{exam tomorrow}
      }

\section{5.11 Unit exam \hfill Thursday 13 February}
\frame
{
  \frametitle{GQ: How do we understand quadratics?}
  \framesubtitle{CCSS: HSF.IF.C8.A Complete the square of quadratic functions \hfill \alert{5.11  Thursday 13 February}}

  \begin{block}{Unit test - No Calculator}
    \begin{enumerate}
      \item Graphing quadratic functions
      \item Factoring quadratic functions
      \item Completing the square
      \item Quadratic formula
      \item Uses of the discriminant
    \end{enumerate}
    \end{block}
    Homework: Polynomials handout
    }

  \section{6.2 Intro to calculus \hfill Thursday 27 February}
  \frame
  {
    \frametitle{GQ: How do we graph tangents to functions?}
    \framesubtitle{CCSS: HSF.IF.C8.A Understanding rate of change \hfill \alert{5.11  Thursday 27 February}}

    \begin{block}{Do Now: Linear equation practice}
      \begin{enumerate}
        \item Write down the equation of the line through $(2,-3)$ with slope $m=2$
        \item Write down the equation of the line through $(-1,0)$ perpendicular to the line with slope $m=2$
        \item Sketch the function $f(x)=x^2+1$ and $g(x)=-2x$ on the same axes
      \end{enumerate}
      \end{block}
      Lesson: Polynomial function terminology, the power rule
      Homework: Deltamath calculus practice
      }

\end{document}
