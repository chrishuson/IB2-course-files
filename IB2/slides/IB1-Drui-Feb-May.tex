\documentclass{beamer}
\usepackage{geometry}
\usepackage[english]{babel}
\usepackage[utf8]{inputenc}
\usepackage{amsmath}
\usepackage{amsfonts}
\usepackage{amssymb}
\usepackage{tikz}
\usepackage{graphicx}
\usepackage{venndiagram}

%\usepackage{pgfplots}
%\pgfplotsset{width=10cm,compat=1.9}
%\usepackage{pgfplotstable}

\setlength{\headheight}{26pt}%doesn't seem to fix warning

\usepackage{fancyhdr}
\pagestyle{fancy}
\fancyhf{}

%\rhead{\small{21 May 2018}}
\lhead{\small{BECA / Dr. Huson / Mathematics}}

%\vspace{1cm}

\renewcommand{\headrulewidth}{0pt}


\title{Mathematics Class Slides}
\subtitle{Bronx Early College Academy}
\author{Chris Huson}
\date{28 August 2018}

\begin{document}

\frame{\titlepage}

%\section[Outline]{}
%\frame{\tableofcontents}

  \section{11.1 Drui}
  \frame
  {
    \frametitle{GQ: How do we organize data using Venn diagrams?}
    \framesubtitle{CCSS: HSS.CP.B.6 Probabilities}

    \begin{block}{Do Now}
    \begin{enumerate}
        \item Find $_5C_1$.
        \item Write down the first 5 rows of Pascal's triangle, and circle the coefficient nCr for $n=5, r=1$.
    \end{enumerate}
    Homework review \#3 p. 67
    \end{block}
    Lesson: Tools for counting, Venn diagrams (p. 68)\\*[5pt]
    Task: unions, intersections, complements of sets\\*[5pt]
    Assessment:  \\*[5pt]
    Homework: p 71-2 3B \#1-6
  }
  \section{11.1 Drui}
  \frame
  {
    \frametitle{GQ: How do we organize data using Venn diagrams?}
    \framesubtitle{CCSS: HSS.CP.B.6 Probabilities}

    \begin{block}{Do Now}
    \begin{enumerate}
        \item Solve for $r$ given the volume of a sphere, $V=\frac{4}{3} \pi r^3$.
        \item Make a tree diagram representing flipping a coin twice. Mark the edge probabilities, assuming a fair coin.
        \item Problem 3B \#4 p. 71.
    \end{enumerate}
    Review/peer grade homework
    \end{block}
    Lesson: The addition rule, Examples 4, 5 (p. 73)\\*[5pt]
    Task: Solving for the intersection of two sets\\*[5pt]
    Assessment: Discuss the difference between dividing by $\frac{4}{3}$ and multiplying by $\frac{3}{4}$\\*[5pt]
    Homework: p 74-5 3C \#1-8
  }

  \section{11.1 Drui}
  \frame
  {
    \frametitle{How do we organize data using sample space diagrams?}
    \framesubtitle{CCSS: HSS.CP.A.1 Probabilities: subsets of a sample space \qquad \alert{11.1}}

    \begin{block}{Do Now (use a diagram or table to support your answer)}
    \begin{enumerate}
        \item If a coin is flipped twice, what is the probability of getting at least one heads?
        \item How many ways are there to roll a 5 with two dice?
    \end{enumerate}
    \end{block}
    Lesson: Probability concepts: bias and fairness, random variation, \& combinations\\*[5pt]
    Task: Exercises 3F page 82.\\*[5pt]
    Assessment: Two cards are drawn from a deck, without replacement. Are the events independent?\\*[5pt]
    Homework: Handout review problems.
  }

  \section{11.1 Drui}
  \frame
  {
    \frametitle{How do we calculate conditional probability?}
    \framesubtitle{CCSS: HSS.CP.B.6 Find and interpret conditional probabilities \qquad \alert{11.1}}

    \begin{block}{Do Now: From four aces, two cards are drawn at random}
    \begin{enumerate}
        \item What is the probability both will be red cards? (Assume with replacement)
        \item Same question, without replacement
    \end{enumerate}
    \end{block}
    Lesson: Conditional probability\\*[5pt]
    Task: Example 11 p 86\\*[5pt]
    Assessment: Two cards are drawn from a deck, without replacement. Are the events independent?\\*[5pt]
    Homework: Exercises 3G p 86-8
  }
  \section{11.1 Drui}
  \frame
  {
    \frametitle{How do we calculate conditional probability?}
    \framesubtitle{CCSS: HSS.CP.B.6 Find and interpret conditional probabilities \qquad \alert{11.1}}

    \begin{block}{Do Now: Draw a Venn diagram summarizing the situation.}
    \begin{enumerate}
        \item Of the 53 staff at a school, 36 drink tea, 18 drink coffee, and 10 drink neither.
    \end{enumerate}
    \end{block}
    Lesson: Conditional probability, trees p. 89\\*[5pt]
    Task: Example 12 p 89\\*[5pt]
    Assessment: Two cards are drawn from a deck, without replacement. Are the events independent?\\*[5pt]
    Homework: Exercises 3H p 90
  }

  \section{11.1 Drui}
  \frame
  {
    \frametitle{How do we calculate conditional probability?}
    \framesubtitle{CCSS: HSS.CP.B.6 Find and interpret conditional probabilities \qquad \alert{11.1}}

    \begin{block}{Do Now: Review test results, answers}
    \begin{enumerate}
        \item Max 84 pts: 77, 72, 69, 68, 64 ...48, ...38, 34, 34, 28, 25, 15 \item Team performance bonus to two groups
        \item Test corrections due Thursday before exam
    \end{enumerate}
    \end{block}
    Homework review\\*[5pt]
    Lesson: Functions, quadratics review\\*[5pt]
    Task: Do review problems (packet)\\*[5pt]
    Assessment: Self-review, answers tomorrow (check website)\\*[5pt]
    Homework: Pretest (\alert{Trimester Final Exam Thursday})
  }

  \section{11.1 Drui}
  \frame
  {
    \frametitle{What are the features of the function representing the multiplicative inverse?}
    \framesubtitle{CCSS: HSS.IF.C.7d Graph and analyze rational functions \qquad \alert{11.1}}

    \begin{block}{Do Now: Graph $f(x)=\frac{1}{x}$}
    \begin{enumerate}
        \item Have the calculator find the point where $x=100$, hence calculating $f(100)$. (you may have to resize the window)
    \end{enumerate}
    \end{block}
    Homework review\\*[5pt]
    Lesson: The reciprocal function and its asymptotes\\*[5pt]
    Task: Show $x \xrightarrow{} \frac{1}{x}$ is a self-inverse\\*[5pt]
    Assessment: Identify asymptotes under vertical and horizontal translations\\*[5pt]
    Homework: Exercises 5B p. 146 (graph paper)
  }



  \section{11.1 Drui}
  \frame
  {
    \frametitle{What are the asymptotes of a rational function?}
    \framesubtitle{CCSS: HSS.IF.C.7d Graph and analyze rational functions \qquad \alert{11.1}}

    \begin{block}{Do Now: Iodine-131 decay formula (time $t$ in days): \[N(t)=N_0 \left( \frac{1}{2} \right)^{\frac{t}{8}}\]}
    \begin{enumerate}
        \item $N_0=100$ grams, how much will decay in the first 8 days?
        \item How much will decay in the second 8 days?
        \item When will 10 grams remain?
    \end{enumerate}
    \end{block}
    Lesson: IA Criterion A. Rational functions p. 147-151\\*[5pt]
    Task: Identify asymptotes under vertical and horizontal translations\\*
    Assessment: Exercise 5C \# 2i\\*
    Homework: Write an aim \& rationale, with a plan outline to investigate geometric series. Exercise 5D \#1-2 p. 151
  }


  \section{11.1 Drui}
  \frame
  {
    \frametitle{What are the rules for a geometric series?}
    \framesubtitle{CCSS: HSA.SSE.B.4 Geometric series\qquad \alert{11.1}}

    \begin{block}{Do Now: Given the parent function $f(x)=\frac{1}{x}$}
    \begin{enumerate}
        \item Write down the equations of $f$'s asymptotes.
        \item The function $f$ is shifted to the right 3 to make the function $g$. What is the equation for $g(x)$?
        \item $g$ is now shifted up 2 units to make $h$. Find $h(x)$.
    \end{enumerate}
    \end{block}
    Lesson: Geometric sequences \& series formulas. p. 167-179\\*[5pt]
    Task: Example 8 \& 9 (calculator) p. 168-9\\*
    Assessment: Exercise 6D \#1d, e, f\\*
    Homework: Exercise 6E \#1-6 p. 169
  }


  \section{11.1 Drui}
  \frame
  {
    \frametitle{What are the rules for a geometric series?}
    \framesubtitle{CCSS: HSA.SSE.B.4 Geometric series\qquad \alert{11.1}}

    \begin{block}{Do Now: The rule of 72}
    \begin{enumerate}
        \item Create a table in your calculator of a geometric sequence with $u_1=500$ and $r=1.05$.
        \item What is the first term in the sequence exceeding 1000? Write its value and the prior value.
        \item Solve the same problem using logarithms.
        \item Interpret the situation as a financial investment.
    \end{enumerate}
    \end{block}
    Mini-IA paper review\\*
    Lesson: The sum of a geometric series. p. 175-179\\*
    Task: Example 17 \& 19 p. 176-7\\*
    Assessment: Exercise 6J \#2\\*
    Homework: Exercise 6F \#1-3 p. 171
  }


  \section{11.1 Drui}
  \frame
  {
    \frametitle{What are the rules for an arithmetic series?}
    \framesubtitle{CCSS: HSA.SSE.B.4 Arithmetic series\qquad \alert{11.1}}

    \begin{block}{Do Now: Given a geometric sequence with $u_1=3$ and $r=2.25$}
    \begin{enumerate}
        \item Find $u_5$. ("Find" means you must show the appropriate values substituted into a formula)
        \item Find $S_5$, the sum of the first five terms of the sequence.
        \item $S_k=7980$. Find $k$ algebraically.
        \item Create a table in your calculator to check your answer.
    \end{enumerate}
    \end{block}
    Lesson: Arithmetic sequences and series, recursive formulas p. 162-6\\*
    Task: Example 3 \& 5 p. 165-6\\*
    Assessment: Exercise 6C \#1 p. 167\\*
    Homework: Handout packet of IB problems (work on paper)
  }



  \section{11.1 Drui}
  \frame
  {
    \frametitle{How do we solve for missing parameters given a series?}
    \framesubtitle{CCSS: HSA.SSE.B.4 Arithmetic series\qquad \alert{11.1}}

    \begin{block}{Do Now: Given an exponential function $y=A_0b^{kx}+c$}
    \begin{enumerate}
        \item Graph the function
        \item What evidence from the graph indicates the function is growing versus decaying?
        \item What algebraically features determine exponential growth versus decay?
    \end{enumerate}
    \end{block}
    Review sequences and notation conventions\\*
    Lesson: Solving for series variables p. 174 Example 15 \\*
    Task: Exercises 6H p. 174\\*
    Assessment: Exercise 6H \#3 p. 174 \alert{(test Thursday)}\\*
    Homework: Handout packet of IB problems (pretest)
  }

  \section{11.1 Drui}
  \frame
  {
    \frametitle{How do we solve for missing parameters given a series?}
    \framesubtitle{CCSS: HSA.SSE.B.4 Arithmetic series\qquad \alert{11.1}}

    \begin{block}{Do Now: Given the inverse function $f(x)=\frac{1}{x}$}
    \begin{enumerate}
        \item Find the inverse of the function, $f^{-1}(x)$
        \item The function $g(x)$ is $f$ translated three units to the right. Find $g$
        \item Find the inverse of $g$, $g^{-1}(x)$
        \item Find $h(x)$, $g(x)$ translated up two
    \end{enumerate}
    \end{block}
    Lesson: IA engagement \& reflection criteria\\*
    Task: Review pretest solutions\\*
    Assessment: \alert{(test Thursday)}\\*
    Homework: Read \& score example IA
  }



  \section{11.1 Drui}
  \frame
  {
    \frametitle{How do we summarize the features of a population?}
    \framesubtitle{CCSS: HSS.IC.A.1 Understand statistics as a process for making inferences about a population \qquad \alert{11.1}}

    \begin{block}{Do Now: Subway paper data}
    \begin{enumerate}
        \item Sketch two box \& whisker plots on the same axis, given the timing data
        \item What can you conclude?
    \end{enumerate}
    \end{block}
    Lesson: Review yesterday's DN p.2; compare aims, proposal requirements; Rules on page 281\\*
    Task: Review pretest problems\\*
    Assessment: Exam Tomorrow (Marking Period ``final")\\*
    Homework: Study for exam\\
    Subway comparison (proposal due tomorrow)\\
  }

  \section{11.1 IB Math SL Drui}
  \frame
  {
    \frametitle{How do we compare two variables?}
    \framesubtitle{CCSS: HSS.IC.A.1 Understand statistics as a process for making inferences about a population \qquad \alert{11.1}}

    \begin{block}{Do Now: Bivariate plot}
    \begin{enumerate}
        \item Plot the given values as points on a graph
        \item Compute $\bar{x}$ and $\bar{y}$
    \end{enumerate}
    \end{block}
    Lesson: Picking an exploration topic\\*
    Task: Logarithm and exam problem review\\*
    Assessment: Test corrections due Monday April 30\\*
    Homework: Study for SAT\\
  }

  \section{11.1 IB Math SL Drui}
  \frame
  {
    \frametitle{How do we compare two variables?}
    \framesubtitle{CCSS: HSS.IC.A.1 Understand statistics as a process for making inferences about a population \qquad \alert{11.1}}

    \begin{block}{Do Now: Given two independent events with probability 0.6 and 0.5. }
    \begin{enumerate}
        \item What is the probability of both happening together?
        \item Draw a Venn diagram representing their intersection.
    \end{enumerate}
    \end{block}
    Lesson: Applying independence tests in a real world context. Picking an exploration topic\\*
    Task: Regression of bivariate data. Exercise 10A \#4 p 339\\*
    Assessment: Test corrections due today\\*
    Homework: Statistics textbook problems 10B p. 341\\
  }


  \section{11.1 IB Math SL Drui}
  \frame
  {
    \frametitle{How do we summarize from a frequency table?}
    \framesubtitle{CCSS: HSS.IC.A.1 Understand statistics as a process for making inferences about a population \qquad \alert{11.1}}

  \begin{block}{Do Now: An investment of \$1,500 earns a continuous interest rate of 2.25\%. }
    \begin{enumerate}
        \item Write down the value of the investment as a function of time in years: $f(t)=$
        \item How much would the investment be worth after 10 years? How much of that is interest?
        \item Use a graphing calculator to compare two functions: $y=1500 \times e^{(10x)}$ and $y=3000$.
        \item For what $x$ are the functions equal? What does this point represent?
    \end{enumerate}
    \end{block}
    Lesson: Solving complex equations with graphing calculators\\*
    %Task: Handout IB cumulative distribution exam problem\\*
    Assessment: Exam Thursday\\*
    Homework: Handout of review problems\\
  }



  \section{11.1 IB Math SL Drui}
  \frame
  {
    \frametitle{How do we summarize from a frequency table?}
    \framesubtitle{CCSS: HSS.IC.A.1 Understand statistics as a process for making inferences about a population \qquad \alert{11.1}}

  \begin{block}{Do Now: Exam review problems}
    \begin{enumerate}
      \item Evaluate $f(x)=\sqrt{x^2}$ for $x=-2, -1, 0, 1, 2$
      \item Sketch the function for $x \in \mathbb{R}$.
      \item Write down another representation of $f(x)$.
      \item Solve for $a$ where $a^x=(e^{0.03925})^x$
      \item Enter into a calculator stats table\\
          \begin{tabular}{|c|r|r|r|r|r|}
          \hline
          $x$ & 2 & 4 & 6 & 8 & 10\\
          \hline
          $y$ & 12 & 20 & 30 & 36 & 52 \\
          \hline
          \end{tabular}
          \item Calculate a linear regression. What do the numbers mean?
      \end{enumerate}
    \end{block}
    Homework review\\
    Lesson: Exam results, substituting into formulas, precision\\*
    %Task: Handout IB cumulative distribution exam problem\\*
    %Assessment: \\*
    Homework: Test corrections due Wednesday\\
  }

  \section{11.1 IB Math SL Drui}
  \frame
  {
    \frametitle{How do we calculate triangle ratios?}
    \framesubtitle{HSF.TF.A.3 Use special right triangles to find sine \& cosine \qquad \alert{11.1}}

  \begin{block}{Do Now: Linear regression practice}
    \begin{enumerate}
      \item Enter the following data into a calculator stats table\\
          \begin{tabular}{|c|r|r|r|r|r|}
          \hline
          $x$ & 22 & 28 & 43 & 62 & 75\\
          \hline
          $y$ & 0.375 & 0.469 & 0.682 & 0.883 & 0.966 \\
          \hline
          \end{tabular}
      \item Using the 2-variable function, find the ``mean point," $(\Bar{x}, \Bar{y})$.
      \item Using the linear regression function, find the coefficients of fitted line $y=ax+b$ and the correlation, $r$. Characterize $r$.
      \item Estimate $y$ for $x=45$
      \end{enumerate}
    \end{block}
    Lesson: Trig function definitions, special right triangles\\*
    Task: Textbook notes p. 362-9 \\*
    Assessment: Calculator trig function, degree-radians switch\\*
    Homework: Exercises 11A odds p. 367, 11B \#1-5 p. 368-9\\
  }


  \section{11.1 IB Math SL Drui}
  \frame
  {
    \frametitle{How do we apply trig to solve problems?}
    \framesubtitle{HSG Define trigonometric ratios and solve problems involving right triangles \qquad \alert{11.1}}

  \begin{block}{Do Now: Right triangle relationships}
    \begin{enumerate}
      \item Write down (from memory, or derive using the Pythagorean formula) the values of $\sin{45^\circ}, \sin{30^\circ}, \sin{60^\circ}$
      \item Given $\sin{x}=\frac{4}{5}$. Find $\cos{x}$
      \item Draw a figure to explain why $\sin{x}=\cos{(90^\circ-x)}$
      \end{enumerate}
    \end{block}
    Lesson: Compass bearings \& elevations p. 369-373 (7th)\\*
    %Task: Textbook notes p. 362-9 \\*
    Assessment: Exercise 11C \#1 p. 372\\*
    Homework: Exercises 11C p. 372-3\\
  }

  \section{11.1 IB Math SL Drui}
  \frame
  {
    \frametitle{How do we apply trig to solve problems?}
    \framesubtitle{HSG Define trigonometric ratios and solve problems involving right triangles \qquad \alert{11.1}}

  \begin{block}{Do Now Quiz: Right triangle relationships\\
      \emph{Complete on lined paper to hand in. Work independently.}}
    \begin{enumerate}
      \item Sketch a $45^\circ, 45^\circ, 90^\circ$ triangle in standard position (with the right angle in the bottom right), label the legs of unit length.
      \item Find the length of the hypotenuse.
      \item Write down the values of $\sin{45^\circ}, \cos{45^\circ}$
      \item Sketch an equilateral triangle with sides of length 2. Divide it into two $30^\circ, 60^\circ, 90^\circ$ triangles.
      \item Find the altitude.
      \item Write down $\sin{30^\circ}, \sin{60^\circ}, \cos{30^\circ}, \cos{60^\circ}$
      %\item Given $\sin{x}=\frac{4}{5}$. Find $\cos{x}$
      %\item Draw a figure to explain why $\sin{x}=\cos{(90^\circ-x)}$
      \end{enumerate}
    \end{block}
    Lesson: Compass bearings \& elevations p. 369-373\\*
    %Task: Textbook notes p. 362-9 \\*
    Assessment: Exercise 11C \#1 p. 372\\*
    Homework: Exercises 11C p. 372-3\\
  }


\section{11.1 IB Math SL Drui}
\frame
{
  \frametitle{How do we apply trig to solve problems?}
  \framesubtitle{HSG Define trigonometric ratios and solve problems involving right triangles \qquad \alert{11.1}}

\begin{block}{Do Now handout: Sine graph, complex number review}
    %\emph{Complete on lined paper to hand in. Work independently.}}
  \begin{enumerate}
    \item Confirm that your calculator is in radian mode
    \item Use the Table $\xrightarrow{}$ SET (F5) function
    \item For complex number problems, first write down the four powers of $i$
    \end{enumerate}
  \end{block}
  Lesson: The Pythagorean identity p. 378. The sine rule p. 381\\*
  %Task: Textbook notes p. 362-9 \\*
  Assessment: Exercise 11G \#1a p. 383\\*
  Homework: Read through p. 382; Exercises 11G \#1-4 p. 383\\
}

\frame
{
  \frametitle{Exercise 11C \#1}
  %\framesubtitle{}
  Isosceles triangle $ABC$ has base $AC = 10 \text{ cm}$ and sides $AB=CB=15 \text{ cm}$.
\begin{itemize}
      \item Find the height of the triangle
      \item Find the sizes of $B\hat{A}C$ and $A\hat{B}C$
\end{itemize}
\begin{center}
\begin{tikzpicture}
\draw (0,0) node[anchor=north]{$A$}
  -- (3,0) node[anchor=north]{$C$}
  -- (1.5,4.2) node[anchor=south]{$B$}
  -- cycle;
  \draw [dashed] (1.5,0) -- (1.5,4.2);
\end{tikzpicture}
\end{center}
 }

\frame
{
  \frametitle{Unit circle}
  %\framesubtitle{}
  Circle with radius of one centered on the origin.
%\begin{itemize}
      %\item Find the height of the triangle
      %\item Find the sizes of $B\hat{A}C$ and $A\hat{B}C$
%\end{itemize}
\begin{center}
\begin{tikzpicture}[scale=3]
  \draw[font=\scriptsize]
    (-1.2, 0) -- (1.2, 0)
    (0, -1.1) -- (0, 1.1)
    %(0, 0) -- (0.866, .5)
    (0, 0) circle[radius=1]
    (-1, 0) node[below left] {$(-1,0)$}
    (1, 0) node[above right] {$(1,0)$}
    (1.1, 0) node[below right] {$\large{x}$}
    ;
\end{tikzpicture}
\end{center}
 }

\section{11.1 IB Math SL Drui}
\frame
{
  \frametitle{Graphing on a calculator to solve equations}

\begin{block}{To solve an equation, separate the equality into two functions, \[f(x)=g(x)\]The intersection of their graphs is the solution.}
  \begin{itemize}
      \item Solve for $x$: $|x-1|-3 = -2x^2+x+3$.
      \item As a check, first make a quick sketch of the functions.
      %\item Use a graphing calculator to compare two functions: $y=1500 \times e^{(10x)}$ and $y=3000$.
      %\item For what $x$ are the functions equal? What does this point represent?
  \end{itemize}
  \end{block}
  Notes: \\Learn to resize the calculator window efficiently\\
  Use the calculator's graph-solve function\\
  Even simple equations can be solved this way. e.g. $e^{0.12x} = 5.25$
}



\end{document}
