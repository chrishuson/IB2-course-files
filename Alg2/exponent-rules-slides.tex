\documentclass{beamer}
\usepackage{geometry}
\usepackage[english]{babel}
\usepackage[utf8]{inputenc}
\usepackage{amsmath}
\usepackage{amsfonts}
\usepackage{amssymb}
\usepackage{tikz}
\usepackage{graphicx}
\usepackage{venndiagram}

\usepackage{pgfplots}
%\pgfplotsset{width=10cm,compat=1.9}
%\usepackage{pgfplotstable}

\setlength{\headheight}{26pt}%doesn't seem to fix warning

\usepackage{fancyhdr}
\pagestyle{fancy}
\fancyhf{}

\rhead{\small{9 April 2018}}
\lhead{\small{BECA / Dr. Huson / Mathematics}}

%\vspace{1cm}

\renewcommand{\headrulewidth}{0pt}


\title{Mathematics Class Slides}
\subtitle{Bronx Early College Academy}
\author{Chris Huson}
\date{April 2018}

\begin{document}

\frame{\titlepage}

%\section[Outline]{}
%\frame{\tableofcontents}

\section{Exponent rules}
\frame
{
  \frametitle{How do we know the exponent rules work?}
  \framesubtitle{CCSS:  \qquad \alert{algebra}}

  \begin{block}{Using the following examples, identify the applicable exponent rule and explain why it works.}
    \begin{enumerate}
    \item $a^2 \cdot a^3 = a^{2+3} = a^5$\\*
    $a \cdot a \times a \cdot a \cdot a = a \cdot a \cdot a \cdot a \cdot a$\\*[15pt]
    \item $\displaystyle \frac{a^5}{a^3} = a^{5-3} = a^2$\\*[15pt]
    $\displaystyle \frac{a \cdot a \cdot a \cdot a \cdot a}{a \cdot a \cdot a} = a \cdot a \times \left( \frac{ a \cdot a \cdot a}{a \cdot a \cdot a} \right) = a \cdot a$
    \end{enumerate}
 \end{block}
}

\frame
{
  \frametitle{How do we know the exponent rules work?}
  \framesubtitle{CCSS:  \qquad \alert{algebra}}

  \begin{block}{Using the following examples, identify the applicable exponent rule and explain why it works.\\ \alert{power rule, fractional and negative exponents; identify false appl}}
    \begin{enumerate}
    \item $a^2 \cdot a^3 = a^{2+3} = a^5$\\*
    $a \cdot a \times a \cdot a \cdot a = a \cdot a \cdot a \cdot a \cdot a$\\*[15pt]
    \item $\displaystyle \frac{a^5}{a^3} = a^{5-3} = a^2$\\*[15pt]
    $\displaystyle \frac{a \cdot a \cdot a \cdot a \cdot a}{a \cdot a \cdot a} = a \cdot a \times \left( \frac{ a \cdot a \cdot a}{a \cdot a \cdot a} \right) = a \cdot a$
    \end{enumerate}
 \end{block}
}



\end{document}
