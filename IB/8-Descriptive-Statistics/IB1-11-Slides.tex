\documentclass{beamer}
\usepackage{geometry}
\usepackage[english]{babel}
\usepackage[utf8]{inputenc}
\usepackage{amsmath}
\usepackage{amsfonts}
\usepackage{amssymb}
\usepackage{tikz}
\usepackage{graphicx}
\usepackage{venndiagram}

%\usepackage{pgfplots}
%\pgfplotsset{width=10cm,compat=1.9}
%\usepackage{pgfplotstable}

\setlength{\headheight}{26pt}%doesn't seem to fix warning

\usepackage{fancyhdr}
\pagestyle{fancy}
\fancyhf{}

%\rhead{\small{26 November 2018}}
\lhead{\small{BECA / Dr. Huson / Mathematics}}

%\vspace{1cm}

\renewcommand{\headrulewidth}{0pt}


\title{Mathematics Class Slides}
\subtitle{Bronx Early College Academy}
\author{Chris Huson}
\date{26 November 2018}

\begin{document}

\frame{\titlepage}

%\section[Outline]{}
%\frame{\tableofcontents}

  \section{12.1 Drui}
  \frame
  {
    \frametitle{GQ: How is the binomial expansion like a probability tree?}
    \framesubtitle{CCSS: HSS.MD.A.3 Develop a probability distribution for a random variable \qquad \alert{12.1}}

    \begin{block}{Do Now: Make a tree representing three coin flips}
    \begin{enumerate}
        \item What is the probability of each outcome?
        \item If order doesn't matter, how can the results be consolidated into a probability distribution of the total number of heads?
    \end{enumerate}
    \end{block}
    Lesson:  Binomial expansion p. 186-8\\%*[5pt]
    Task: IB exam paper problems\\%*[5pt]
    Assessment: Test corrections due Thursday (snow pending)
    \\%*[5pt]
    Homework: Complete probability problem set
  }

  \section{12.1 Drui}
  \frame
  {
    \frametitle{GQ: How do we summarize the features of a population?}
    \framesubtitle{CCSS: HSS.MD.A.3 Develop a probability distribution for a random variable \qquad \alert{12.1}}

    \begin{block}{Do Now: Given $f(x)=x^2+3x$. \\*
    (work on paper you can turn in.}
    \begin{enumerate}
        \item Find $f'(x)$.
        \item What is the equation of the tangent to $f$ at $x=1$?
        \item Plot both $f$ and the tangent on your graphing calculator.
    \end{enumerate}
    \end{block}
    Lesson:  Cumulative distributions, summative stats p. 42-72\\%*[5pt]
    Task: Grouped frequency table calculations\\%*[5pt]
    Assessment: Problem \#3 2H p. 60
    \\%*[5pt]
    Homework: Problem set, univariate data statistics
  }


  \section{12.1 Drui}
  \frame
  {
    \frametitle{GQ: How do we interpret a cumulative distribution graph?}
    \framesubtitle{CCSS: HSS.MD.A.3 Develop a probability distribution for a random variable \qquad \alert{12.1}}

    \begin{block}{Do Now: Given the data in problem \#3 2H p. 60.\\*
    (work on paper you can turn in.)}
      \begin{enumerate}
        \item Write down the modal class.
        \item Write a formula for the mean test result.
        \item Compute the mean using the calculator stats function.
      \end{enumerate}
   \end{block}
    Homework review\\*
    Lesson:  Dispersion p. 73-83, Normal distributions p. 204-216\\%*[5pt]
    Task: Problem \#1 2K p. 72\\%*[5pt]
    Assessment: Problem \#7 Review exercise p. 79
    \\%*[5pt]
    Homework: Problem set, cumulative distributions
  }



  \section{12.1 Drui}
  \frame
  {
    \frametitle{GQ: How do we use the normal curve?}
    \framesubtitle{CCSS: HSS.MD.A.3 Develop a probability distribution for a random variable \qquad \alert{12.1}}

    \begin{block}{Do Now}
      \begin{enumerate}
      \item Confirm the regression equation fit to \#4 p. 348.
      \item How would you interpret the correlation of two variables having $r=-0.65$? (p. 359)
      \item Sketch a normal curve with $X\sim N(500, 100^2)$.
      \href{https://blog.prepscholar.com/sat-standard-deviation}{link}
      \end{enumerate}
   \end{block}
    Homework review\\*
    Lesson:  Regression, residuals, least-squares, extrapolation, mean point $(\bar{x},\bar{y})$ (p. 334-359)\\*
    Normal distribution, Z-score, inverse normal (p. 538-553) \\%*[5pt]
    Task: The standard normal and Z-score problems, 15H p. 541\\%*[5pt]
    Assessment: Exercise 15H \#1 \& 5 p. 541\\%*[5pt]
    Homework: Problem set regressions, probability distributions
  }

  \section{12.1 Drui}
  \frame
  {
    \frametitle{GQ: How do we use the inverse normal function?}
    \framesubtitle{CCSS: HSS.MD.A.3 Develop a probability distribution for a random variable \qquad \alert{12.1}}

    \begin{block}{Do Now: What formulas applying to pretest problems 4c \& 4d?}
      \begin{enumerate}
      \item Given $A$ and $B$ are independent and $\mathrm{P}(A)=0.2$, $\mathrm{P}(B)=0.8$. Find $\mathrm{P}(A \cap B)$
      \item \href{https://blog.prepscholar.com/sat-standard-deviation}{SAT link}
      \end{enumerate}
   \end{block}
    Pretest packet homework review\\*
    Lesson: Chapter 15 summary p. 553\\%*[5pt]
    Task: Review exercises p. 551\\%*[5pt]
    Assessment: Exam this week\\%*[5pt]
    Homework: IB problem set
  }

\end{document}
