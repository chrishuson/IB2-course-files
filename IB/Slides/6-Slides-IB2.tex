\documentclass{beamer}
\usepackage{geometry}
\usepackage[english]{babel}
\usepackage[utf8]{inputenc}
\usepackage{amsmath}
\usepackage{amsfonts}
\usepackage{amssymb}
\usepackage{tikz}
\usepackage{graphicx}
\usepackage{venndiagram}

%\usepackage{pgfplots}
%\pgfplotsset{width=10cm,compat=1.9}
%\usepackage{pgfplotstable}

\setlength{\headheight}{26pt}%doesn't seem to fix warning

\usepackage{fancyhdr}
\pagestyle{fancy}
\fancyhf{}

%\rhead{\small{21 May 2018}}
\lhead{\small{BECA / Dr. Huson / 12.1 IB Math - Unit 6 Trig \& Circular Functions}}

%\vspace{1cm}

\renewcommand{\headrulewidth}{0pt}


\title{12.1 IB Math - Unit 6: Trig \& Circular Functions}
\subtitle{Bronx Early College Academy}
\author{Christopher J. Huson PhD}
\date{5 - 14 March 2019}

\begin{document}

\frame{\titlepage}

\section[Outline]{}
\frame{\tableofcontents}

\section{6.1 Right triangle review. Tuesday 5 March}
  \frame
  {
    \frametitle{GQ: How do we define and calculate right triangle measures?}
    \framesubtitle{CCSS: HSG.SRT.C.8 Use trigonometric ratios to solve right triangles in applied problems \hfill \alert{6.1 Tuesday 5 March}}

    \begin{block}{Do Now: Calculator integration fluency\\For each: sketch, solve for $f(x)=g(x)$, and find the area between the curves (write down the integration expression)}
      \begin{enumerate}
        \item $f(x)=x$, $g(x)=x^2$
        \item $f(x)=-x^2+2$, $g(x)=-1$
        \item $f(x)=x^3-9x$, $g(x)=\sin x$
      \end{enumerate}
      \end{block}
    Lesson: Trig ratios, special triangles' values p. 362-9\\
    Practice: Calculator use, Examples \#1, 2  p. 365\\
    Exam review;
    Reminder: complete exploration papers, \alert{parent conferences} \\*[5pt]
    Homework: Part 2 take-home exam: Integration, no calculator
  }

\section{6.2 Trigonometry applications. Wednesday 6 March}
  \frame
  {
    \frametitle{GQ: How do we apply trigonometry to situations?}
    \framesubtitle{CCSS: HSG.SRT.C.8 Use trigonometric ratios to solve right triangles in applied problems \hfill \alert{6.3 Wednesday 6 March}}

    \begin{block}{Do Now: Solving triangles}
    \begin{enumerate}
        \item Exercise 11B \#2, p. 368
    \end{enumerate}
    \end{block}
    Lesson: Compass directions and modeling situations p. 369-373\\
    Exam review\\*[5pt]
    Homework: Trig IB papers problem set, handout
  }


\section{6.3 The unit circle. Thursday 7 March}
  \frame
  {
    \frametitle{GQ: How do we use periodic functions?}
    \framesubtitle{CCSS: HSF.TF.A.3 Extend trig functions with the unit circle  \hfill \alert{6.4 Thursday 7 March}}

      \begin{block}{Do Now: Create a unit circle and label the standard angles with their coordinate pairs.}
        \begin{enumerate}
            \item \emph{Medium} Find the values of $\sin 30^\circ$, $\sin 45^\circ$, \& $\sin 60^\circ$
            \item \emph{Spicy} Find $\sin \frac{\pi}{6}$, $\cos \frac{3\pi}{4}$, \& $\tan -\frac{\pi}{3}$
        \end{enumerate}
      \end{block}
    Lesson: Periodic functions\\%*[5pt]
    Task: Work homework problems on board\\%*[5pt]
    Assessment: problem set mark scheme\\%*[5pt]
    Homework: Sine curves \& mixed exam problems
  }


\section{6.4 The unit circle. Friday 8 March}
  \frame
  {
    \frametitle{GQ: How do we use periodic functions?}
    \framesubtitle{CCSS: HSF.TF.A.3 Extend trig functions with the unit circle  \hfill \alert{6.5 Friday 8 March}}

    \begin{block}{Do Now: Sketch the periodic function $f(x)=\sin{x}$}
      \begin{enumerate}
      \item Label the $x$-axis with multiples of $\pi$, including standard fractions in the first quadrant
      \item Mark the $y$-axis with the values of the standard angles (positive and negative).
      \item Mark points on the curve at the standard angles.
      \end{enumerate}
   \end{block}
    Homework review\\[5pt]
    Lesson: Applications calculating the period as $\frac{2\pi}{b}$\\%*[5pt]
    Task: Work homework problems on board\\%*[5pt]
    Assessment: problem set mark scheme\\%*[5pt]
    Homework: Trig \& mixed exam problems
  }

\section{6.5 Right triangle review. Monday 11 March}
  \frame
  {
    \frametitle{GQ: How do we use periodic functions?}
    \framesubtitle{CCSS: HSF.TF.A.3 Extend trig functions with the unit circle  \hfill \alert{6.6 Monday 11 March}}

    \begin{block}{Do Now: Calculator integration fluency\\
      Sketch, solve for $f(x)=g(x)$, and find the area between the curves (write down the integration expression and calculate)}
      \begin{enumerate}
        \item $f(x)=x$ for $x>0$, $g(x)=2 \sin x$
        \item $f(x)=\sqrt{x+1}$, $g(x)=\frac{1}{2}(x+1)$
        \item $f(x)=\sqrt{4-3x^2}$, $g(x)=0$
        \item The volume of \#3 rotated $360^\circ$ around the $x$-axis
      \end{enumerate}
      \end{block}
    Lesson: Test review, work problems on board\\%*[5pt]
    Homework: Trig \& mixed exam problems
  }

\section{6.6 Deltamath trigonometry review. Tuesday 12 March}
  \frame
  {
    \frametitle{GQ: How do we define and calculate right triangle measures?}
    \framesubtitle{CCSS: HSG.SRT.C.8 Use trigonometric ratios to solve right triangles in applied problems \hfill \alert{6.2 Tuesday 5 March}}

    Exam review\\
    Lesson: Deltamath trigonometry (\& calculus) review\\
    Homework: Complete Deltamath problem set
  }

\section{6.7 Circle sector, arc problems. Wednesday 13 March}
  \frame
  {
    \frametitle{GQ: How do we measure parts of a circle?}
    \framesubtitle{CCSS: HSF.TF.A.3 Extend trig functions with the unit circle  \hfill \alert{6.7 Wednesday 13 March}}

    \begin{block}{Do Now Quiz: Special triangle trig (exact) values, no calculator}
    \begin{enumerate}
        \item \emph{Medium} Find the values of $\sin 30^\circ$, $\sin 45^\circ$, \& $\sin 60^\circ$
        \item \emph{Spicy} Find $\sin \frac{\pi}{6}$, $\cos \frac{3\pi}{4}$, \& $\tan -\frac{\pi}{3}$
    \end{enumerate}
    \end{block}
    Lesson: Circle sector, arc problems, work problems on board\\%*[5pt]
    Homework: 3.1, 3.2, 3.3 Trig \& mixed exam problems
  }

\section{6.8 Periodic function situations. Thursday 14 March}
  \frame
  {
    \frametitle{GQ: How do we model periodic situations?}
    \framesubtitle{CCSS: HSF.TF.A.3 Extend trig functions with the unit circle  \hfill \alert{6.8 Thursday 14 March}}

    \begin{block}{Do Now: Use a calculator to find the extrema of \emph{one} function}
    \begin{enumerate}
        \item \emph{Mild:} $f(x)=(x-2)(x+3)(x+1)$
        \item \emph{Medium:} $g(x)=2ln(x^2+1)-x$
        \item \emph{Spicy:} $h(x)= (x-\pi)^2 sin(x-\frac{\pi}{2})$ for $0 \leq x\leq 2\pi$
    \end{enumerate}
    \end{block}
    Lesson: Periodic function problems, work problems on board\\*[5pt]
    Homework: 3.4, 3.5, 3.6 Trig \& mixed exam problems
  }

\section{6.9 Periodic function situations. Friday 15 March}
  \frame
  {
    \frametitle{GQ: How do we model periodic situations?}
    \framesubtitle{CCSS: HSF.TF.A.3 Extend trig functions with the unit circle  \hfill \alert{6.9 Friday 15 March}}

    \begin{block}{Do Now: Answer using your knowledge of functions, then check by graphing with a calculator. (all questions)}
    \begin{enumerate}
        \item \emph{Mild:} What is the amplitude and midline of $f(x)=3 \sin x+2$?
        \item \emph{Medium:} What is the period of $g(x)=-2 \sin \pi x$?
        \item \emph{Spicy:} What is the equation of a periodic function with a range of $[-1,3]$ and a period of $6 \pi$?
    \end{enumerate}
    \end{block}
    Lesson: Periodic function problems, work problems on board\\*[5pt]
    Homework: 3.4, 3.5, 3.6 Trig \& mixed exam problems
  }

\end{document}
