\documentclass{beamer}
\usepackage{geometry}
\usepackage[english]{babel}
\usepackage[utf8]{inputenc}
\usepackage{amsmath}
\usepackage{amsfonts}
\usepackage{amssymb}
\usepackage{tikz}
\usepackage{graphicx}
\usepackage{venndiagram}

%\usepackage{pgfplots}
%\pgfplotsset{width=10cm,compat=1.9}
%\usepackage{pgfplotstable}

\setlength{\headheight}{26pt}%doesn't seem to fix warning

\usepackage{fancyhdr}
\pagestyle{fancy}
\fancyhf{}

%\rhead{\small{24 September 2018}}
\lhead{\small{BECA / Dr. Huson / 12.1 IB Math}}

%\vspace{1cm}

\renewcommand{\headrulewidth}{0pt}


\title{Mathematics Class Slides}
\subtitle{Bronx Early College Academy}
\author{Chris Huson}
\date{22 October - 2 November 2018}

\begin{document}

\frame{\titlepage}

\section[Outline]{}
\frame{\tableofcontents}


\section{3b.1 Drui - Vector arithmetic, Friday Nov 16}
  \frame
  {
    \frametitle{GQ: How do we find the angle between vectors?}
    \framesubtitle{CCSS: HSG.SRT.D11 Apply the law of cosines \qquad \alert{3b.1 Friday Nov 16}}

    \begin{block}{Do Now: Given $\triangle SNO$ with $S(2,1),N(7,1),O(10,5)$.}
      \begin{tikzpicture}[scale=0.7]
        \draw [thick] (2,1)--(7,1)--(10,5)--(2,1);
        %\draw [dashed] (2,1)--(10,5);
        %\draw [dashed] (7,1)--(5,5);
        \draw [fill] (2,1) circle [radius=0.05] node[below]{$S$};
        \draw [fill] (7,1) circle [radius=0.05] node[below]{$N$};
        \draw [fill] (10,5) circle [radius=0.05] node[above right]{$O$};
        %\draw [fill] (5,5) circle [radius=0.05] node[above left]{$W$};
        %\draw [fill] (6,3) circle [radius=0.05] node[right]{$P$};
      \end{tikzpicture}
    \begin{enumerate}
        \item Write down the law of cosines
        \item Find the lengths $SN$ and $SO$
        \item Given $m\angle S=26.6^\circ$, find $NO$
    \end{enumerate}
    \end{block}

    Lesson: Law of cosines, the scalar product\\
    Homework exercise 12I pp. 428-9
  }


  \section{3b.2 Drui - Vector arithmetic, Monday Nov 19}
    \frame
    {
      \frametitle{GQ: How do we find the angle between vectors?}
      \framesubtitle{CCSS: HSG.SRT.D11 Apply the law of cosines \qquad \alert{3b.2 Monday Nov 19}}

      \begin{block}{Do Now: Exam Style Question \#5 p 439.}
      \end{block}

      Lesson: Practice with the law of cosines, the scalar product\\
      Homework: Calculus review problem set handout
    }

    \section{3b.4 Drui - Vector equations of lines, Wednesday Nov 21}
      \frame
      {
        \frametitle{GQ: How do we find the angle between vectors?}
        \framesubtitle{CCSS: HSG.SRT.D11 Apply the law of cosines \qquad \alert{3b.4 Wednesday Nov 21}}

        \begin{block}{Do Now: Given position vectors $\overrightarrow{OA}, \overrightarrow{OB}, \overrightarrow{OC}$ with $A(-3,2),B(4,1),C(3,k)$.}
            \begin{center}
              \begin{tikzpicture}[scale=0.6]
                \draw [<->, thick] (-3,2)--(0,0)--(4,1);
                \draw [->, dashed] (0,0)--(3,3) node[below right]{$C$};
                %\draw [dashed] (7,1)--(5,5);
                \draw [fill] (-3,2) circle [radius=0.05] node[below]{$A$};
                \draw [fill] (0,0) circle [radius=0.05] node[below left]{$O$};
                \draw [fill] (4,1) circle [radius=0.05] node[above right]{$B$};
                %\draw [fill] (5,5) circle [radius=0.05] node[above left]{$W$};
                %\draw [fill] (6,3) circle [radius=0.05] node[right]{$P$};
              \end{tikzpicture}
            \end{center}
          \begin{enumerate}
              \item Find $m\angle AOB$
              \item Find $k$ such that $\overrightarrow{OA} \perp \overrightarrow{OC}$
          \end{enumerate}
        \end{block}

        Review Exercise 12I pp. 428-9\\
        Lesson: Vector equations of lines p. 430-1\\
        Homework: Calculus review problem set handout
      }

  \section{3b.5 Drui - Vector equations of lines, intersections, Monday Nov 26}
    \frame
    {
      \frametitle{GQ: How do we use vector line equations?}
      \framesubtitle{CCSS: HSG.SRT.D11 Apply the law of cosines \qquad \alert{3b.5 Monday Nov 26}}

        Do Now: Given $A(-3,2)$ and direction vector $\overrightarrow{b} =2\overrightarrow{i}+\overrightarrow{j}$
        \begin{enumerate}
          \item Find the equation of the line through $A$ parallel to $\overrightarrow{b}$
          \item Is the point $C(3,4)$ on the specified line? Justify your answer.
        \end{enumerate}
        %\begin{center}
        \hspace{2cm}
            \begin{tikzpicture}[scale=0.6]
              \draw [->, thick] (0,0)--(2,1) node[right]{$b=2 \overrightarrow{i}+\overrightarrow{j}$};
              \draw [->, dashed] (-3,2)--(-1,3);
              %\draw [dashed] (7,1)--(5,5);
              \draw [fill] (-3,2) circle [radius=0.05] node[below]{$A(-3,2)$};
              \draw [fill] (0,0) circle [radius=0.05] node[below left]{$O$};
              %\draw [fill] (2,1) circle [radius=0.05] node[above left]{$b$};
              \draw [fill] (3,5) circle [radius=0.05] node[below right]{$C(3,4)$};
              %\draw [fill] (6,3) circle [radius=0.05] node[right]{$P$};
            \end{tikzpicture}
          %\end{center}

      Review vector equations of lines, Exercise 12J pp. 432-4\\
      Lesson: Finding the intersection of two lines p. 434-5\\
      Homework: Exercise 12J pp. 432-4\\
      \alert{Parent-teacher conferences Thursday \& Friday}
    }

  \section{3b.7 Drui - Vector equations of lines, applications, Wednesday Nov 28}
    \frame
    {
      \frametitle{GQ: How do we use vector line equations?}
      \framesubtitle{CCSS: HSG.SRT.D11 Apply the law of cosines \qquad \alert{3b.7 Wednesday Nov 28}}

        Do Now: Write directions to go from Room 414 to Yankee Stadium. Assume that streets in the Bronx run north-south and east-west. Estimate distances. \\[0.5cm]
        Spicy: include a third dimension (elevation).\\[1cm]

      Review intersections of two lines p. 434-5\\
      Lesson: Applications p. 437 \\
      Homework: Exercise 12K pp. 435-6\\[0.5cm]
      \alert{Parent-teacher conferences tomorrow \& Friday}
    }

  \section{3b.8 Drui - Vector applications, Thursday Nov 29}
    \frame
    {
      \frametitle{GQ: How do we use vector line equations?}
      \framesubtitle{CCSS: HSG.SRT.D11 Apply the law of cosines \qquad \alert{3b.8 Thursday Nov 29}}

        Do Now: Vector angle Review Exercise \#1 p. 440.\\
        Spicy: \#2, $\triangle PQR$ \\[1cm]

      Review intersections of lines in 3 dimensions p. 434-5\\
      Lesson: Applications p. 437 \\
      Homework: Exercise 12L pp. 437-8\\[0.5cm]
      \alert{Parent-teacher conferences today \& tomorrow}
    }

  \section{3b.9 Drui - Vector applications, Friday Nov 30}
    \frame
    {
      \frametitle{GQ: How do we use vectors to solve problems?}
      \framesubtitle{CCSS: HSG.SRT.D11 Apply the law of cosines \qquad \alert{3b.9 Friday Nov 30}}

        Do Now: Vector angle Review Exercise \#1 p. 438.\\
        Spicy: \#2 p. 438\\[1cm]

      Lesson: Applications p. 437 \\
      Homework: Review Exercises (evens) p. 438-441\\
      \alert{Note the calculator and NO-calculator sections}\\
      Spicy: evens and odds\\[0.5cm]
      \alert{Parent-teacher conferences today 12-3:00}
    }

  \section{3b.10 Drui - Vector applications, Monday Dec 3}
    \frame
    {
      \frametitle{GQ: How do we use vectors to solve problems?}
      \framesubtitle{CCSS: HSG.SRT.D11 Apply the law of cosines \qquad \alert{3b.10 Monday Dec 3}}

        %Do Now: Vector angle Review Exercise \#1 p. 438.\\
        %Spicy: \#2 p. 438\\[1cm]

      Lesson: Review applications homework problems p. 437 \\
      Homework: Review Exercises (evens) p. 438-441\\
      Spicy: evens and odds\\[0.5cm]
    }

  \section{3b.11 Drui - Deltamath: Vector applications, Tuesday Dec 4}
    \frame
    {
      \frametitle{GQ: How do we use vectors to solve problems?}
      \framesubtitle{CCSS: HSG.SRT.D11 Apply the law of cosines \qquad \alert{3b.11 Tuesday Dec 4}}

        %Do Now: Vector angle Review Exercise \#1 p. 438.\\
        %Spicy: \#2 p. 438\\[1cm]

      Lesson: Individualize practice, Deltamath assessment assignment \\
      Homework: Complete Deltamath homework section
    }

\end{document}
