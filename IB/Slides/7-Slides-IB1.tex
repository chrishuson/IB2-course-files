\documentclass{beamer}
\usepackage{geometry}
\usepackage[english]{babel}
\usepackage[utf8]{inputenc}
\usepackage{amsmath}
\usepackage{amsfonts}
\usepackage{amssymb}
\usepackage{tikz}
\usepackage{graphicx}
\usepackage{venndiagram}

%\usepackage{pgfplots}
%\pgfplotsset{width=10cm,compat=1.9}
%\usepackage{pgfplotstable}

\setlength{\headheight}{26pt}%doesn't seem to fix warning

\usepackage{fancyhdr}
\pagestyle{fancy}
\renewcommand{\headrulewidth}{0pt}

\lhead{\small{BECA / Dr. Huson / 11.1 IB Math - Unit 7 Sequences and Series}}

\title{11.1 IB Math - Unit 7 Sequences and Series}
\subtitle{Bronx Early College Academy}
\author{Christopher J. Huson PhD}
\date{18-28 March 2019}

\begin{document}

\frame{\titlepage}

\section[Outline]{}
\frame{\tableofcontents}

\section{7.1 Introduction and definitions Monday 18 March}
  \frame
  {
    \frametitle{GQ: How do we work with sequences?}
    \framesubtitle{CCSS: HSF.BF.A.2 Write arithmetic and geometric sequences, use them to model situations \hfill \alert{7.1 Monday 18 March}}

    \begin{block}{Do Now: Complete Investigation - \emph{Saving Money} p. 162}
    \end{block}
    Lesson: Arithmetic sequences, recursion, definitions p.161-6\\[1cm]
    Homework: Exercises 6A (a \& b only) p. 164, 6B p. 166
  }

  \section{7.2 Deltamath recursive notation practice, Tuesday 19 March}
    \frame
    {
      \frametitle{GQ: How do we use recursive notation?}
      \framesubtitle{CCSS: HSF.BF.A.2 Write arithmetic and geometric sequences, use them to model situations \hfill \alert{7.2 Tuesday 19 March}}

      \begin{block}{Deltamath probability practice}
      \end{block}
      Homework: Complete Deltamath exercises
    }

  \section{7.3 Geometric sequences, Wednesday 20 March}
    \frame
    {
      \frametitle{GQ: How do we model compound growth?}
      \framesubtitle{CCSS: HSF.BF.A.2 Write arithmetic and geometric sequences, use them to model situations \hfill \alert{7.3 Wednesday 20 March}}

      \begin{block}{Do Now: Exercise 6C p. 167}
      \end{block}
      Lesson: Geometric sequences, Sigma notation p.167-171\\[1cm]
      Homework: Exercises 6D, 6E, 6F (odds only) p. 168, 169, 171
    }

  \section{7.4 Arithmetic series, Thursday 21 March}
    \frame
    {
      \frametitle{GQ: How do we calculate the sum of a sequence?}
      \framesubtitle{CCSS: HSF.BF.A.2 Write arithmetic and geometric sequences, use them to model situations \hfill \alert{7.4 Thursday 21 March}}

      \begin{block}{Do Now: Exercise 6E \#4, \#6 p. 169-170}
      \end{block}
      Lesson: Arithmetic series, Sigma notation p.167-171\\[1cm]
      Homework: Exercises 6G, 6H (odds only) p.  173-5
    }

  \section{7.5 Geometric series, Monday 25 March}
    \frame
    {
      \frametitle{GQ: How do we calculate the sum of a sequence?}
      \framesubtitle{CCSS: HSF.BF.A.2 Write arithmetic and geometric sequences, use them to model situations \hfill \alert{7.5 Monday 25 March}}

      \begin{block}{Do Now: Review exercise \#1, \#2a, 2b, \#3 p. 189}
      \end{block}
      Lesson: Geometric series p. 175-7\\[1cm]
      Homework: Exercises 6I, 6J (a, c only) p.  176, 178
    }

\section{7.6 Geogebra Fibonacci sequence, Tuesday 26 March}
  \frame
  {
    \frametitle{GQ: How do we depict the Fibonacci sequence geometrically?}
    \framesubtitle{CCSS: HSF.BF.A.2 Write arithmetic and geometric sequences, use them to model situations \hfill \alert{7.6 Tuesday 26 March}}

    \begin{block}{Do Now: Find an example of the Golden Mean}
      \begin{enumerate}
        \item Measure three distances:  from the floor to your belly button ($b$), from your belly button to the top of your head ($a$), \& from the floor to the top of your head ($a+b$). (they should add up)
        \item Compute the following two ratios: $\frac{a}{b}$ and $\frac{b}{a+b}$
        \item Are the two ratios equal?
        \item Solve for $\frac{a}{b}$, such that $\frac{a}{b} = \frac{b}{a+b}$
    \end{enumerate}
    \end{block}
    Lesson: Geogebra construction of the Fibonacci spiral \\
    Homework: Complete a project paper. (good luck on the SAT tomorrow)
  }

\section{7.7 Geogebra Fibonacci sequence, Thursday 28 March}
  \frame
  {
    \frametitle{GQ: How do we depict the Fibonacci sequence?}
    \framesubtitle{CCSS: HSF.BF.A.2 Write arithmetic and geometric sequences, use them to model \hfill \alert{7.7 Thursday 28 March}}

    \begin{block}{Exploration: Fibonacci Spiral project}
      \begin{enumerate}
        \item Do Now: solve for $\frac{a}{b}$, such that $\frac{a}{b}=\frac{a+b}{b}$
        \item Read paper: \href{
        https://medium.com/i-math/what-is-the-golden-ratio-d3cc17c8fefd}{math hacks}, search online for images of the Golden Mean and Fibonacci Sequence
    \end{enumerate}
    \end{block}
    Lesson: MLA citations and references, table captions \\
    Homework: Complete the project paper. Store all files in Dropbox folder. Email a final pdf to me (filename: lastname-projectname.pdf)
  }

  \section{7.8 Geogebra Fibonacci sequence, Monday 1 April}
    \frame
    {
      \frametitle{GQ: How do we depict the Fibonacci sequence?}
      \framesubtitle{CCSS: HSF.BF.A.2 Write arithmetic and geometric sequences, use them to model \hfill \alert{7.8 Monday 1 April}}

      \begin{block}{Do Now: Geometric sequence \& series handout}
      \end{block}
      Lesson: Using Excel to compute a sequence, series, \& ratios\\
      Homework: Complete the project paper. Store all files in Dropbox folder. Email a final pdf to me (filename: lastname-projectname.pdf)
    }

\section{7.9 Free write Fibonacci sequence, Wednesday 3 April}
  \frame
  {
    \frametitle{GQ: How do we depict the Fibonacci sequence?}
    \framesubtitle{CCSS: HSF.BF.A.2 Write arithmetic and geometric sequences, use them to model \hfill \alert{7.9 Wednesday 3 April}}

    \begin{block}{Do Now: Geometric sequence \& series handout}
    \end{block}
    Lesson: Spend the period writing, using a laptop computer\\
    Homework: Complete the project paper. Store all files in Dropbox folder. Email a final pdf to me (filename: lastname-projectname.pdf)
  }

\section{7.10 Convergent (infinite) series, Thursday 4 April}
  \frame
  {
    \frametitle{GQ: How do we sum an infinite number of terms?}
    \framesubtitle{CCSS: HSF.BF.A.2 Write arithmetic and geometric sequences, use them to model situations \hfill \alert{7.10 Thursday 4 April}}

    \begin{block}{Do Now: IB sequences \& series examples}
    \end{block}
    Lesson: 6.7 Convergence of infinite geometric series p. 178-181\\
    Excel modeling of the ratio of Fibonacci sequence terms\\[1cm]
    Homework: Problem set of Regents exponential function questions
  }

\section{7.11 Applications \& exponential growth, Monday 8 April}
  \frame
  {
    \frametitle{GQ: How do we model situations involving sequences?}
    \framesubtitle{CCSS: HSF.BF.A.2 Write arithmetic \& geometric sequences, use them to model situations \hfill \alert{7.11 Monday 8 April}}

    \begin{block}{Do Now: IB sequences \& series handout (10 minutes)}
    \end{block}
    Homework review\\
    Lesson: 6.8 Applications of geometric and arithmetic patterns \\p. 181-183\\[1cm]
    Homework: Exercises 6L p.  182-3
  }

\section{7.12 Applications \& exponential growth, Tuesday 9 April}
  \frame
  {
    \frametitle{GQ: How do we model situations involving sequences?}
    \framesubtitle{CCSS: HSF.BF.A.2 Write arithmetic \& geometric sequences, use them to model situations \hfill \alert{7.12 Tuesday 9 April}}

    \begin{block}{Do Now: Deltamath practice problems}
    \end{block}
    Homework review\\
    Lesson: 6.8 Applications of geometric and arithmetic patterns \\p. 181-183\\[1cm]
    Homework: Complete Deltamath assignment
  }

\section{7.13 Applications \& exponential growth, Wednesday 10 April}
  \frame
  {
    \frametitle{GQ: How do we model situations involving sequences?}
    \framesubtitle{CCSS: HSF.BF.A.2 Write arithmetic \& geometric sequences, use them to model situations \hfill \alert{7.13 Wednesday 10 April}}

    \begin{block}{Do Now: IB sequences \& series handout (10 minutes)}
    \end{block}
    Homework review\\
    Lesson: 6.8 Applications of geometric and arithmetic patterns \\p. 181-183\\[1cm]
    Homework: Exercises 6L p.  182-3
  }

\section{7.14 Binomial expansion, Pascal's triangle Thursday 11 April}
  \frame
  {
    \frametitle{GQ: How do we model situations involving sequences?}
    \framesubtitle{CCSS: HSF.BF.A.2 Write arithmetic \& geometric sequences, use them to model situations \hfill \alert{7.14 Thursday 11 April}}

    \begin{block}{Do Now: IB sequences \& series handout (10 minutes)}
    \end{block}
    Homework review\\
    Lesson: 6.9 Binomial expansion, Pascal's triangle \\p. 184-188\\[1cm]
    Homework: Exercises 6O p.  188
  }

\section{7.15 Binomial expansion, Pascal's triangle Monday 15 April}
  \frame
  {
    \frametitle{GQ: How do we model situations involving sequences?}
    \framesubtitle{CCSS: HSF.BF.A.2 Write arithmetic \& geometric sequences, use them to model situations \hfill \alert{7.15 Monday 15 April}}

    \begin{block}{Do Now: Review Exercise p. 189}
      \begin{enumerate}
        \item Start with \#4a. Solve it two ways: intuitively \& with the formula.
        \item Go back to \#1, continue with 2-5.
    \end{enumerate}
    \end{block}
    Homework review\\
    Lesson: 6.9 Binomial expansion, Pascal's triangle \\p. 184-188\\[1cm]
    Homework: Exam-style questions p. 189-190 (If you haven't done them, Exercises 6O p.  188)
  }

\section{7.16 DeltaMath Binomial expansion Tuesday 16 April}
  \frame
  {
    \frametitle{GQ: How do we model situations involving sequences?}
    \framesubtitle{CCSS: HSF.BF.A.2 Write arithmetic \& geometric sequences, use them to model situations \hfill \alert{7.16 Tuesday 16 April}}

    \begin{block}{Do Now: Check Skedula for end of Marking Period}
    \end{block}
    Lesson: DeltaMath Binomial expansion\\[1cm]
    Homework: Complete Deltamath
  }

\section{7.17 Review for exam Wednesday 17 April}
  \frame
  {
    \frametitle{GQ: How do we model situations involving sequences?}
    \framesubtitle{CCSS: HSF.BF.A.2 Write arithmetic \& geometric sequences, use them to model situations \hfill \alert{7.17 Wednesday 17 April}}

    \begin{block}{Do Now: Review Exercises p. 189}
      \begin{enumerate}
        \item Review for exam
    \end{enumerate}
    \end{block}
    Lesson: Test review, sequences \& series\\[1cm]
    Homework: Study for \alert{exam tomorrow}
  }

\section{7.18 Exam Thursday 18 April}
  \frame
  {
    \frametitle{GQ: How do we model situations involving sequences?}
    \framesubtitle{CCSS: HSF.BF.A.2 Write arithmetic \& geometric sequences, use them to model situations \hfill \alert{7.18 Thursday 18 April}}

    Lesson: Test \\[1cm]
    Homework: Vacation packet
  }

  \section{7.19 Binomial expansion, Pascal's triangle Monday 29 April}
    \frame
    {
      \frametitle{GQ: How do we model situations involving sequences?}
      \framesubtitle{CCSS: HSF.BF.A.2 Write arithmetic \& geometric sequences, use them to model situations \hfill \alert{7.19 Monday 29 April}}

      \begin{block}{Do Now: Review Exercise p. 189}
        \begin{enumerate}
          \item Start with \#4a. Solve it two ways: intuitively \& with the formula.
          \item Go back to \#1, continue with 2-5.
      \end{enumerate}
      \end{block}
      Workplace visit debrief\\
      Exam review\\
      Lesson: 6.9 Binomial expansion, Pascal's triangle \\p. 184-188\\[1cm]
      Homework: Exam-style questions p. 189-190 (If you haven't done them, Exercises 6O p.  188)
    }

  \section{7.20 DeltaMath Binomial expansion Tuesday 30 April}
    \frame
    {
      \frametitle{GQ: How do we model situations involving sequences?}
      \framesubtitle{CCSS: HSF.BF.A.2 Write arithmetic \& geometric sequences, use them to model situations \hfill \alert{7.20 Tuesday 30 April}}

      \begin{block}{Do Now: Check Skedula for end of Marking Period}
      \end{block}
      Lesson: DeltaMath Binomial expansion\\[1cm]
      Homework: Complete Deltamath
    }


\end{document}
