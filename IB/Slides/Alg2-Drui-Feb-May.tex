\documentclass{beamer}
\usepackage{geometry}
\usepackage[english]{babel}
\usepackage[utf8]{inputenc}
\usepackage{amsmath}
\usepackage{amsfonts}
\usepackage{amssymb}
\usepackage{tikz}
\usepackage{graphicx}
\usepackage{venndiagram}

%\usepackage{pgfplots}
%\pgfplotsset{width=10cm,compat=1.9}
%\usepackage{pgfplotstable}

\setlength{\headheight}{26pt}%doesn't seem to fix warning

\usepackage{fancyhdr}
\pagestyle{fancy}
\fancyhf{}

%\rhead{\small{21 May 2018}}
\lhead{\small{BECA / Dr. Huson / Mathematics}}

%\vspace{1cm}

\renewcommand{\headrulewidth}{0pt}


\title{Mathematics Class Slides}
\subtitle{Bronx Early College Academy}
\author{Chris Huson}
\date{28 August 2018}

\begin{document}

\frame{\titlepage}

%\section[Outline]{}
%\frame{\tableofcontents}


\section{11.2 Drui}
\frame
{
  \frametitle{GQ: How do we organize data using frequency distributions?}
  \framesubtitle{CCSS: HSS.CP.B.6 Probabilities \qquad \qquad \qquad \alert{11.2}}

  \begin{block}{Do Now: Write the formula and perform the calculation}
  \begin{enumerate}
      \item How many groups of 2 students may be selected from a class of 12?
      \item How many groups of 10 students may be selected from a class of 12?
  \end{enumerate}
  \end{block}
  Lesson: Subsets, frequency distributions\\*[5pt]
  Task: Combinatorics problem\\*[5pt]
  Assessment: Final homework problem\\*[5pt]
  Homework: Set theory handout
}

\section{11.2 Drui}
\frame
{
  \frametitle{GQ: How do we graph polynomials?}
  \framesubtitle{CCSS: HSS.CP.B.6 Understand polynomial functions \qquad \qquad \qquad \alert{11.2}}

  \begin{block}{Do Now: Write down vocabulary words in notebook}
  \begin{enumerate}
    \item Standard form, factored form, order, degree\\*[5pt]
    \item substitution, long division, remainder\\*[5pt]
    \item $x$-intercepts, zeros, roots, solutions\\*[5pt]
    \item $y$-intercept\\*[5pt]
    \item end behavior, increasing/decreasing, turning points\\*[5pt]
    \item symmetry, odd/even\\*[5pt]
  \end{enumerate}
  \end{block}
  Lesson: Features of polynomial functions p. 280\\*[5pt]
  Task: Problems \# 8-19 odds, 32-37 p. 285\\*[5pt]
  Assessment: Graphing\\*[5pt]
  Homework: Workbook p. 119
}

\frame
{
  \frametitle{Graphing polynomials}
  \framesubtitle{Evaluating functions using the distributive property and substitution \qquad \qquad \qquad \alert{11.2}}

  \begin{block}{Group presentations}
  \begin{enumerate}
      \item $f(x)=x^3-5x^2+2x+8$ \qquad Group A\\*[10pt]
      %$f(x)=(x-2)(x-4)(x+1)$\\*
      \item $g(x)=-x^3-7x^2-14x-8$ \qquad Group B\\*[10pt]
      %$g(x)=(x-2)(x-4)(x+1)$\\*
      \item $h(x)=-x^3-4x$ \qquad \qquad \qquad Group C\\*[10pt]
      \item $j(x)=x^3+2x^2-5x-6$ \qquad Group D\\*[10pt]
      \item $k(x)=-x^3+4x$\\*
      $k(x)=-x(x+2)(x-2)$ \qquad Group D\\*[10pt]

  \end{enumerate}
  \end{block}
}
\frame
{
  \frametitle{Graphing polynomials}
  \framesubtitle{Evaluating functions using the distributive property and substitution \qquad \qquad \qquad \alert{11.2}}

  \begin{block}{Do Now: }
  \begin{enumerate}
      \item Group A:       $f\left(x\right)=x^3+2x^2-5x-6$
      $g\left(x\right)=\left(x-2\right)\left(x-4\right)\left(x+1\right)$\\*

  \end{enumerate}
  \end{block}
}

\frame
{
  \frametitle{Polynomials}
  \framesubtitle{Each polynomial function can be shown in two forms: standard and factored. \qquad \qquad \qquad \alert{11.2}}
\alert{Standard form}: From largest exponent to smallest\\*
\qquad \alert{Order or degree}: value of the largest exponent\\*
\qquad \alert{Constant term}: the ones value (8, in the example below)
\alert{Factored form}: Product of binomials\\*
\qquad \alert{Factor}: each monomial (e.g. "$(x+1)$)\\*[15pt]
  \begin{enumerate}
    \item Evaluate $f(0)$ and $f(2)$ for each function below.
      \item $f(x)=x^3-5x^2+2x+8$ \qquad \\*[10pt]
      $f(x)=(x+1)(x-2)(x-4)$\\*

  \end{enumerate}
}

\begin{frame}{Vocabulary for polynomial functions}
    Standard form, factored form, order, degree\\*[5pt]
    substitution, long division, remainder\\*[5pt]
    $x$-intercepts, zeros, roots, solutions\\*[5pt]
    $y$-intercept\\*[5pt]
    end behavior, increasing/decreasing, turning points\\*[5pt]
    symmetry, odd/even\\*[5pt]
\end{frame}

\section{11.2 Algebra II Drui}
\frame
{
  \frametitle{How do we convert the base of an exponential function?}
  \framesubtitle{CCSS: HSF.LB.B.5 Interpret the parameters in an exponential function in context \qquad \alert{11.2}}

  \begin{block}{Do Now: Exponent practice. Simplify:}
  \begin{enumerate}
      \item $a^3 \times b^2 \times a^2$
      \item $x^3 \cdot y^2 \div x^3$
      \item $\displaystyle m^\frac{1}{3} \times m^\frac{4}{3} \times m^\frac{1}{3}$
      \item $z^4 \cdot z^{-1} \cdot z^{-2}$
  \end{enumerate}
  \end{block}
  Lesson: Consolidating a coefficient into the base of an exponential\\*
  Task: Practice problems\\*
  Assessment: Test corrections due today\\*
  Homework: Exponential function \& review problems\\
}


\section{11.2 Algebra II Drui}
\frame
{
  \frametitle{How do we convert the base of an exponential function?}
  \framesubtitle{CCSS: HSF.LB.B.5 Interpret the parameters in an exponential function \qquad \alert{11.2}}

  \begin{block}{Do Now: Interest calculations, assume principal is \$100.}
  \begin{enumerate}
      \item Calculate interest for 6 months with an annual rate of 5\%.
      \item After one year at an annual rate of 5.25\%, what would be the combined principal and interest?
      \item How many interest payments would there be in two years given monthly compounding?
      \item If the interest after one month is \$1.50, what would the annual interest rate be?
  \end{enumerate}
  \end{block}
  Lesson: Continuously compounded interest\\*
  Task: Practice problems\\*
  Assessment: Convert to an annual rate given $P=P_0e^{0.05x}$\\*
  Homework: Exponential function \& review problems\\
}


\section{11.2 Algebra II Drui}
\frame
{
  \frametitle{How do we convert the base of an exponential function?}
  \framesubtitle{CCSS: HSF.LB.B.5 Interpret the parameters in an exponential function \qquad \alert{11.2}}

  \begin{block}{Do Now: An investment of \$1,500 earns a continuous interest rate of 2.25\%. }
  \begin{enumerate}
      \item Write down the value of the investment as a function of time in years: $f(t)=$
      \item How much would the investment be worth after 10 years? How much of that is interest?
      \item Use a graphing calculator to compare two functions: $y=1500 \times e^{(10x)}$ and $y=3000$.
      \item For what $x$ are the functions equal? What does this point represent?
  \end{enumerate}
  \end{block}
  Lesson: Homework problem set review, exponential functions\\*
  Task: Graphing and function problems\\*
  Assessment: \alert{Exam Thursday}\\
  Homework: Pretest problem set\\
}

\section{11.2 Algebra II Drui}
\frame
{
  \frametitle{How do we convert the base of an exponential function?}
  \framesubtitle{CCSS: HSF.LB.B.5 Interpret the parameters in an exponential function \qquad \alert{11.2}}

  \begin{block}{Do Now: Exam review problems}
  \begin{enumerate}
    \item Evaluate $f(x)=\sqrt{x^2}$ for $x=-2, -1, 0, 1, 2$
    \item Sketch the function for $x \in \mathbb{R}$.
    \item Write down another representation of $f(x)$.
    \item Solve for $a$ where $a^x=(e^{0.03925})^x$
    \end{enumerate}
  \end{block}
  Lesson: Exam results, substituting into formulas, precision\\*
  Homework: Test corrections due tomorrow\\}

\end{document}
