\documentclass{beamer}
\usepackage{geometry}
\usepackage[english]{babel}
\usepackage[utf8]{inputenc}
\usepackage{amsmath}
\usepackage{amsfonts}
\usepackage{amssymb}
\usepackage{tikz}
%\usepackage{pgfplots}
%\pgfplotsset{width=10cm,compat=1.9}
%\usepackage{pgfplotstable}
\usepackage{graphicx}
\usepackage{venndiagram}

%\usepackage{pgfplots}

\setlength{\headheight}{26pt}%doesn't seem to fix warning

\usepackage{fancyhdr}
\pagestyle{fancy}
\fancyhf{}

\rhead{\small{10 April 2018}}
\lhead{\small{BECA / Dr. Huson / Mathematics}}

%\vspace{1cm}

\renewcommand{\headrulewidth}{0pt}


\title{Mathematics Class Slides}
\subtitle{Bronx Early College Academy}
\author{Chris Huson}
\date{April 2018}

\begin{document}

\frame{\titlepage}

%\section[Outline]{}
%\frame{\tableofcontents}

\section{12.1 Drui}
\frame
{
  \frametitle{GQ: How do we integrate a function?}
  \framesubtitle{CCSS: HSF.IF.B.6 Calculate and interpret the area under a function \qquad \alert{12.1}}

  \begin{block}{Do Now: Skills problem set (for time)}
    %\begin{enumerate}
   % \item Given $A$ and $B$ are independent and $\mathrm{P}(A)=0.2$, $\mathrm{P}(B)=0.8$. Find $\mathrm{P}(A \cap B)$
    %\item \href{https://blog.prepscholar.com/sat-standard-deviation}{SAT link}
    %\end{enumerate}
 \end{block}
  Lesson: Take home exam papers assessment \& review\\%*[5pt]
  Task: Work problems on board\\%*[5pt]
  Assessment: Test corrections due Friday\\%*[5pt]
  Homework: Deltamath integration practice
}


\section{11.1 Drui}
\frame
{
  \frametitle{How do we summarize the features of a population?}
  \framesubtitle{CCSS: HSS.IC.A.1 Understand statistics as a process for making inferences about a population \qquad \alert{11.1}}

  \begin{block}{Do Now: Practice problem set}
  \begin{enumerate}
      \item Work rapidly and carefully
      \item Take note of any gaps in your knowledge
  \end{enumerate}
  \end{block}
  Lesson: Measures of dispersion pp. 254-276; Cumulative distribution\\*
  Task: Example 10 p. 272\\*
  Assessment: Test corrections due Friday (in my box)\\*
  Homework: Reciprocal paper; Khan Academy statistics unit (incremental due dates)
}

\frame
{
  \frametitle{Topics to cover in Reciprocals paper}
  \framesubtitle{Practice writing mathematics according to IB requirements, as per IA criteria.}
\begin{enumerate}
\item Asymptotic behavior explained with an example of end behavior. 
\item Translation of an example point ($A \xrightarrow{} A'$)
\item Translation's impact on asymptotes.
\item General case of horizontal and vertical translation. Use $h$ and $k$, and vector notation: $\begin{pmatrix} h \\ k \end{pmatrix}$
\item Define a function and its inverse as a composition resulting in the identity. $(f \circ f^{-1})(x)=x$. Include an example.
\item Inverse explained with an example point under reflection across $y=x$
\item Discussion of reversal of the position of $h$ and $k$ in the algebraic representation of the function and its inverse. (perhaps with just example values)
\end{enumerate}
}


\frame
{
  \frametitle{Standards for writing technical papers}
  \framesubtitle{Practice writing mathematics according to IB requirements, as per IA criteria.}
Criterion C: Personal engagement (0-4 points)
\begin{enumerate}
    \item Address a personal interest; ``make it your own"
    \item Think independently and/or creatively
    \item Present mathematical ideas in your own way 
\end{enumerate}
Criterion D: Reflection (0-3 points)
\begin{enumerate}
    \item Review, analyze, and evaluate the mathematics throughout the paper. Go beyond just describing results
    \item Link to the aims, comment on what has been learned, consider limitations, and compare different mathematical approaches
    \item Consider what's next, discuss the implications of results, strengths and weaknesses of approaches, and consider different perspectives
\end{enumerate}
}


\frame
{
  \frametitle{Standard conventions for mathematical notation}
  \framesubtitle{Practice writing mathematics according to IB requirements, as per exam rubrics.}
\begin{enumerate}
    \item Use the formula sheet.
    \item Chose the appropriate formula (M1).\\*
    (you do not have to copy the formula)
    \item Substitute values correctly (A1). 
    \item Solve, showing key steps (A1).\\
    (skip routine algebra if you like)
    \item Write down the exact solution or copy the calculator display. An ellipsis (\ldots) indicates more digits (A1).
    \item Round to 3 significant digits (use $\approx$)(A1).
\end{enumerate}
}

\frame
{
  \frametitle{Standard conventions for mathematical notation}
  \framesubtitle{Practice writing mathematics according to IB requirements, as per exam rubrics.}
Examples of key algebraic techniques
\begin{enumerate}
    \item Setting a quadratic function $=0$
    \item Converting an exponent to a log
    \item Reading a value from a graph
    \item When writing lists, you may write only the first two and the last terms. For example,
\[\sum_{k=1}^5 3 \cdot 2.25^k =3 + 6.75+\ldots+76.8867\ldots\]
\[=135.99609\ldots \approx 136\]
\end{enumerate}
}

\frame
{
  \frametitle{Aim and rationale}
  \framesubtitle{There are many important and amazing sequences and series in mathematics.}
  $\displaystyle \frac{1}{2}+\frac{1}{4}+\frac{1}{8}+\frac{1}{16}+\frac{1}{32}+\frac{1}{64} \dots$ Zeno's paradox (Greek)\\*[10pt]  $\displaystyle \frac{1}{2}+\frac{1}{3}+\frac{1}{4}+\frac{1}{5}+\frac{1}{6}+\frac{1}{7} \dots$ harmonic series\\*[10pt]
  $\displaystyle \frac{\pi}{4}=1-\frac{1}{3}+\frac{1}{5}-\frac{1}{7}+\frac{1}{9}-\frac{1}{11}+\frac{1}{13}$ \dots Madhava  (India)\\*[10pt]
  %1, 1, 2, 3, 5, 8, 13, 21, 34, 55, 89,... (Fibonacci)\\*[10pt]
  0.99999999...\\*[10pt]
The aim of this exploration is to discover the patterns, formulas, and rules of geometric series by experimenting and investigating using a spreadsheet to total various example series.\\*[5pt]
  
*Sequence: list of numbers. Series: sum of a sequence of numbers. Geometric sequence: consecutive terms have a constant ratio.
}

\frame
{
  \frametitle{Descriptive statistics terminology}
  \framesubtitle{Make a list of these terms, find their definitions in the textbook.}
  
  Univariate data, bivariate\\*
  Population, sample, random/biased sample, survey, census\\*
  Discrete/continuous data, quantitative/qualitative\\* 
  Central tendency, mean ($\overline{x}, \mu$), median, mode; quartiles, percentiles\\*
  5-figure summary, box \& whisker plots, range, interquartile range, outlier\\*
  Dispersion, standard deviation ($\sigma$), variance ($v=\sigma^2$)\\*
  Frequency distributions (tables/bar charts/histograms)\\*
  Grouped data, class, mid-interval value, boundaries, modal class\\*
  Cumulative frequency distributions

  
}

\frame
{
  \frametitle{Bias and fairness, random variation, \& combinations}
  \framesubtitle{When rolling two dice, why aren't all the possible totals equally likely?}
  Definition:\\*
  A \alert{fair} (p. 67) or \alert{unbiased} (p. 79) process \\*[15pt]
  In mathematics we usually simplify and assume a random process follows exact, idealized probabilities. For example, we assume heads and tails are equally likely results of a coin toss.
  
}

\frame
{
  \frametitle{Bias and fairness, random variation, \& combinations}
  \framesubtitle{When rolling two dice, why aren't all the possible totals equally likely?}
  Definition:\\*
  \alert{Experimental} or \alert{empirical} (p. 65) results \\*[15pt]
  In real life, the results of any experiment have a degree of \alert{random variation}. The observed relative frequencies are estimates of the underlying theoretical probabilities, which grow more accurate with additional trials.
  
}

\frame
{
  \frametitle{Bias and fairness, random variation, \& combinations}
  \framesubtitle{When rolling two dice, why aren't all the possible totals equally likely?}
  Counting events in a \alert{sample space} (p. 78) or calculating \alert{combinations} (p. 184) \\*[15pt]
  The six possible results of rolling a single die are equally likely, $\mathrm P(x)=\frac{1}{6}$, if we assume the die is fair. Similarly, the probability of any of the 36 $(6 \times 6)$ possible results of rolling two dice are equally likely, $\mathrm P(x)=(\frac{1}{6})^2$. However, the probability of a particular total varies according to how many combinations lead to that total. Thus, for example, 7 can be rolled six different ways, so $\mathrm P(7)=\frac{6}{36}$, while 2 can only result one way, $\mathrm P(2)=\frac{1}{36}$.
  
}

\frame
{
  \frametitle{Sets, subsets, \& proper subsets}
  
  Definitions:\\*
  A \alert{set} is an unordered collection of elements.\\ e.g. \{red, white, blue\} (do not repeat elements)\\*[5pt]
  \alert{Subset}: Set $A$ is a subset of set $B$ if and only if all of the elements of $A$ are elements of $B$.\\
  Written: $A \subseteq B$\\[5pt]
  \alert{Proper subset}: $A \subseteq B$ and $A$ is not equal to $B$. Written: $A \subset B$\\[5pt]
  The \alert{empty set} is a subset of all sets. $\{\} \text{ or } \emptyset$
  
}


\frame
{
  \frametitle{GQ: Combinatorics problem}
  \framesubtitle{CCSS: F.IF.B.6 Calculate \& interpret the rate of change of a function}

  \begin{block}{Show the formula and then use your calculator function}
  \begin{enumerate}
      \item You have a \$1 bill, a \$5 bill, a \$10 bill, a \$20 bill, a quarter, a dime, a nickel, and a penny. How many different total amounts can you make by choosing six bills and coins?
  \end{enumerate}
  \end{block}
  What is the number of the set you are choosing from?\\%*[5pt]
  How many are you picking?\\%*[5pt]
  Does their order matter?
}

\begin{frame}{Do Now \#1: Phone preferences by gender}
    \framesubtitle{Given the frequency table, make a Venn diagram}
    \begin{tabular}{l|c|r|}
        & Android & iPhone\\ 
        \hline 
        Boys & 15 & 5 \\ 
        \hline 
        Girls & 5 & 15 \\
        \hline 
    \end{tabular}\\*[10pt]
    \centering
    $A=\{ \text{prefers Android}\}$ and $B=\{ \text{is a boy}\}$
    \begin{venndiagram2sets}[tikzoptions={scale=1.0}]
    \end{venndiagram2sets}
\end{frame}

\begin{frame}{Do Now \#2: Independence}
    \framesubtitle{Given the situation, make a Venn diagram, frequency table, and tree representing}
    $\mathrm{P}(A)=0.6$, $\mathrm{P}(B)=0.5$, $\mathrm{P}(A \cap B)=0.3$
    \centering
    \begin{venndiagram2sets}[tikzoptions={scale=1.0}]
    \end{venndiagram2sets}\\*[10pt]
    \begin{tabular}{l|c|r|}
        & $A$ & $A^\prime$\\ 
        \hline 
        $B$ &  \qquad \qquad &  \qquad \qquad \\ 
        \hline 
        $B^\prime$ &  &  \\
        \hline 
    \end{tabular}\\*[10pt]
\end{frame}

\begin{frame}{$\mathrm P(A \cup B) = \mathrm P(A) + \mathrm P(B) - \mathrm P(A \cap B)$}
    \framesubtitle{The addition rule}
    \begin{venndiagram2sets}[tikzoptions={scale=2}]
    \end{venndiagram2sets}
\end{frame}

\begin{frame}{Distributions}
    \framesubtitle{Tables and charts used to summarize a problem situation}
    A \alert{frequency distribution} displays the number of times each event in the sample space occurs, either in tabular or graphical form.\\*[10pt]
    A \alert{probability distribution} shows the same data, normalizing the totals to one.
\end{frame}


\begin{frame}{Technical writing}
    \framesubtitle{Write a short paper answering the query: \\* "How many subsets can be picked from a group of four students?"}
    \begin{enumerate}
        \item Logical, step-by-step explanation, using an example
        \item Precise terminology, succinct: combination, permutation, order (matters), event, sample space, set, subset, with /without replacement, factorial
        \item Notation: algebra symbols, tables, trees, grids
        \item Summary, big-picture, conceptual idea
        \item Audience: student peers
    \end{enumerate}
\end{frame}





\begin{frame}{Combinatorics formulas}
    \alert{Combinations}, when order doesn't matter
	$$_nC_r = \frac{n!}{(n-r)! r!} \qquad \text{''n pick r"}$$
    \alert{Permutations}, when order does matter
	$$_nP_r = \frac{n!}{(n-r)!} $$
\end{frame}

\begin{frame}{Definition of theoretical probability}
    The \alert{theoretical probability} of an event $A$ is $\displaystyle \mathrm P(A) = \frac{n(A)}{n(U)}$\\*[10pt]
    \quad where $n(A)$ is the number of ways an event can occur\\*[5pt]
    \quad and $n(U)$ is the total number of possible outcomes (p. 65)\\*[10pt]
    Theoretically, in $n$ trials, one would expect the event to occur $n \times \mathrm P(A)$ times\\*[10pt]
    Probabilities are between 0 and 1, inclusive. $0 \leq \mathrm P(X) \leq 1$
\end{frame}

\begin{frame}{Empirical (experimental) probability}
    The \alert{relative frequency} of an event can be used as an estimate of its probability. $$\displaystyle \mathrm P(A) = \frac{\text{number of occurrences of event } A}{\text{total number of trials}}$$
    The larger the number of trials the more reliable the estimate of probability.
\end{frame}

\begin{frame}{Independence and mutual exclusivity}
    Two events are \alert{independent} if the occurrence of one does not affect the probability of the other. $$\displaystyle \mathrm P(\text{both }A \text{ and }B \text{ occur}) = \mathrm P(A) \times \mathrm P(B)$$
    Two events are \alert{mutually exclusive} if they never occur together. 
    $$\displaystyle \mathrm P(\text{both }A \text{ and }B \text{ occur}) = 0 \qquad \text{and}$$
    $$\mathrm P(\text{either }A \text{ or }B \text{ occur}) = \mathrm P(A) + \mathrm P(B)$$
\end{frame}

\begin{frame}{Venn diagrams}
    \framesubtitle{For organizing compound events}
    When two events can occur, and perhaps both, or neither.
    \begin{venndiagram2sets}[tikzoptions={scale=1.5}]
    \end{venndiagram2sets}
\end{frame}

\begin{frame}{The union of sets: $A \cup B$}
    That $A$ happens, or $B$ happens, or both
    \begin{venndiagram2sets}[tikzoptions={scale=1.5}]
    \fillA
    \fillB
    \end{venndiagram2sets}
\end{frame}

\begin{frame}{The intersection of sets: $A \cap B$}
    That both $A$ and $B$ happen
    \begin{venndiagram2sets}[tikzoptions={scale=1.5}]
    \fillACapB
    \end{venndiagram2sets}
\end{frame}

\begin{frame}{The addition rule}
    \framesubtitle{That $A$ or $B$ or both occur}
    
    When two events can occur, and perhaps both
    
    \begin{venndiagram2sets}%[labelA={primes}, labelB={evens}, shade =lightgray]%
    %\fillA
    %\fillB
    %\fillACapB
    \end{venndiagram2sets}

    $$P(\text{either }A \text{ or }B \text{ occur}) = P(A) + P(B) - P(\text{both }A \text{ and }B \text{ occur})$$
\end{frame}

\begin{frame}{Vocabulary for probability \& statistics}
    event, experiment, random\\*[5pt]
    probability, P(A), values [0,1]\\*[5pt]
    theoretical, empirical, subjective\\*[5pt]
    sample space, U; frequency, trials\\*[5pt]
    n(U) = number of possibilities\\*[5pt]
    P(A) = n(A)/n(U); expected = n * P
\end{frame}

\frame
{
  \frametitle{Interpreting a displacement vs time graph}
  \framesubtitle{CCSS: F.IF.B.6 Calculate \& interpret the rate of change of a function}

  \begin{block}{Consider the function $f(x)=-x^2+2x+3$}
  \begin{enumerate}
      \item Factor $f$ and state its zeros.
      \item Restate $f$ in vertex form. Write down the vertex as an ordered pair.
      \item Over what intervals is the function increasing, decreasing, and neither?
      \item If $f(x)$ represents the height of a diver over the domain $0 \leq x \leq 3$, interpret $f(0)$, the vertex, and $f(3)$
      \item What does the "slope" of the curve represent?
  \end{enumerate}
  \end{block}
}


\end{document}
independent events: P(A&B)=P(A)*P(B) 
dependent events, mutually exclusive
P(A or B)= P(A) + P(B) - P(A&B)
P(A) = n(A)/n(U)