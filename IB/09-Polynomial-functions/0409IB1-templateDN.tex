\documentclass[12pt, oneside]{article}
\usepackage[letterpaper, margin=1in]{geometry}
\usepackage[english]{babel}
\usepackage[utf8]{inputenc}
\usepackage{amsmath}
\usepackage{amsfonts}
\usepackage{amssymb}
%\usepackage{tikz}
%\usepackage{tkz-fct}
%\usepackage{pgfplots}
%\pgfplotsset{width=10cm,compat=1.9}
%\usepackage{pgfplotstable}
%\usepackage{venndiagram}

\usepackage{fancyhdr}
\pagestyle{fancy}
\fancyhf{}
\rhead{\thepage \\Name: \hspace{1.5in}}
\lhead{BECA / Dr. Huson / 11.1 IB Math SL \\* 9 April 2018 \\*\textbf{DoNow practice for familiarity and speed
}}

\renewcommand{\headrulewidth}{0pt}

\title{11.1 DoNow review template}
\author{Chris Huson}
\date{April 2018}

\begin{document}
%\maketitle
\subsubsection*{\\ \textnormal{Work these problems rapidly in the space provided.}}

\begin{enumerate}
%\vspace{0.5 cm}
%\subsubsection*{Sequences and series}
\item In an arithmetic sequence, the first term is 5 and the third term is 17.
\begin{enumerate}
    \item Find the common difference.\\*[10pt]
    \item Write down an equation for $S_{10}$, the sum of the first 10 terms in the sequence, substituting values for $u_1$, $d$, and $n$. (you do not have to simplify the formula)\\*[10pt]
\end{enumerate}

%\subsubsection*{Geometric sequences}
\item Given that a geometric sequence begins with $u_1=9$ and has a common ratio of $r=\frac{2}{3}$.
\begin{enumerate}
    \item What is the third term of the sequence?\\*[20pt]
    \item Write down an equation for $u_{10}$, the 10th term in the sequence, substituting values for $u_1$, $r$, and $n$. (you do not have to simplify the formula)\\*[20pt]
    \item Does the sum of the infinite series have a finite value? Justify your answer in the simplest way possible. (you don't have to write any words, just a short algebraic expression)\\*[10pt]
\end{enumerate}

%\subsubsection*{Precision, rounding, 3 significant figures}
\item Round to three significant figures unless otherwise instructed
\begin{enumerate}
    \item 45.0951\\*[10pt]
    \item 0.031415926\\*[10pt]
    \item 25.36496481 \emph{to the nearest hundredth}\\*[10pt]
    \item $2.732 \times 10^{-3}$\\*[10pt]
\end{enumerate}

%\subsubsection*{Exponents}
\item Simplify the expression $\displaystyle \frac{x^{-1}}{x^4}$ to one with positive integer exponents and radicals.

\newpage
\subsubsection*{Probability}
\item Given $P(A)$ and B, find intersection, union, operations with complements
\item Calculations given independence and mutual exclusivity, test for both
\item Venn 2- and 3-circles, with various shadings, labels and inserted values
\item Given table of values, find probabilities and contingent probabilities


\end{enumerate}
\end{document}