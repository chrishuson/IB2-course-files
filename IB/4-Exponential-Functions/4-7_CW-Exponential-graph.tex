\documentclass[]{book}

\usepackage{import}
\usepackage{preamble}
\usepackage{tikz}

\begin{document}

\noindent BECA / Huson / 11.1 IB Math SL \hspace{2in} Name:\\*
29 November 2018\\[0.5cm]
Do Now: Exponential function graphing\\
\textit{Simplify, leaving no negative or fractional exponents.}\\


\begin{enumerate}

  \item $5x^{-2}y \times 3x^5 y^{2}$\\*[55pt]
  \item $\sqrt[3]{a^3 b^{-6}}$\\*[55pt]
  \item $log_7 49$\\*[55pt]
  \item $log_4 2 + log_4 8$\\*[55pt]


  \item Let $f(x) = \sqrt{2x} +6$ and $g(x)=2x^2$
  \begin{enumerate}
      \item Find $(f \circ g)(x)$\\*[65pt]
      \item Find $f^{-1}(x)$\\*[65pt]
  \end{enumerate}

\end{enumerate}
\newpage
\begin{enumerate}
\subsubsection{Classwork}

  \item The temperature of a hot iron as it cools is modeled by the function $T(x)=400e^{-0.05x}$ where $T(x)$ is the temperature in degrees Celsius and $x$ is the time in minutes.

  \begin{enumerate}
      \item Write down the initial temperature at time zero.\\*[25pt]
      \item Find the temperature after one hour.\\*[45pt]
      \item When will the temperature of the iron reach 100 degrees Celsius?\\*[55pt]
      \item On the graph below, sketch the temperature of the iron, labeling the points above A, B, and C.
  \end{enumerate}

  \begin{figure}[!htbp]
  \begin{center}
  \begin{tikzpicture}[xscale= 0.2, yscale=0.015]

  %grid
  %\draw [thick, color=black,, xstep=1.0cm,ystep=1.0cm] (-5.5,-1.5) grid (5.5,16.5);
  %\draw [thin, color=lightgray,, xstep=0.2cm,ystep=0.2cm] (-5.5,-1.5) grid (5.5,16.5);

  \foreach \x in {0,10,20,30,40,50,60}
  \draw[shift={(\x,0)},color=black] (0pt,-3pt) -- (0pt,3pt) node[below]  {$\x$};

  \foreach \y in {0,100,200,300,400,500}
  \draw[shift={(0,\y)},color=black] (2pt,0pt) -- (-2pt,0pt) node[left]  {$\y$};

  \draw [thick, ->] (0,0) -- (60,0) node [right] {$x$};
  \draw [thick, ->] (0,0) -- (0,500) node [right] {$T(x)$};

  %\draw [<-, ->] plot[domain= 0:60] (\x, e^\x);

  \end{tikzpicture}
  \end{center}
  \end{figure}

\end{enumerate}
\end{document}
