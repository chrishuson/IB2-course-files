\documentclass{beamer}
\usepackage{geometry}
\usepackage[english]{babel}
\usepackage[utf8]{inputenc}
\usepackage{amsmath}
\usepackage{amsfonts}
\usepackage{amssymb}
\usepackage{tikz}
\usepackage{graphicx}
\usepackage{venndiagram}

%\usepackage{pgfplots}
%\pgfplotsset{width=10cm,compat=1.9}
%\usepackage{pgfplotstable}

\setlength{\headheight}{26pt}%doesn't seem to fix warning

\usepackage{fancyhdr}
\pagestyle{fancy}
\fancyhf{}

%\rhead{\small{5 September 2018}}
\lhead{\small{BECA / Dr. Huson / IB Math - Exploration project paper}}

\renewcommand{\headrulewidth}{0pt}

\title{Mathematics Class Slides}
\subtitle{Bronx Early College Academy}
\author{Chris Huson}
\date{12 November 2019}

\begin{document}
\frame{\titlepage}
\section[Outline]{}
\frame{\tableofcontents}


\section{2.8 Revise Excel analysis summary page 25 Oct}
\frame
{
  \frametitle{GQ: How do we communicate statistical results?}
  \framesubtitle{CCSS: MP5 attend to precision \hfill \alert{2.8 Friday 25 Oct}}

  \begin{block}{Mini Exploration: What is the best route to school?}
    \begin{enumerate}
      \item Based on Excel model of commuter data (math.huson.com)
      \item Complete statistical calculations and written analysis
      \item Email the Excel file and a pdf version of spreadsheet \& paper\\
      (three attachments)
  \end{enumerate}
  \end{block}
  Exploration paper scoring criterion: Personal Engagement (Oct24) \\
  Homework: Complete your paper, Sunday 10:00PM deadline
}


\section{2.10 Revise Excel analysis summary page 29 Oct}
\frame
{
  \frametitle{GQ: How do we communicate statistical results?}
  \framesubtitle{CCSS: MP5 attend to precision \hfill \alert{2.10 Tuesday 29 Oct}}

  \begin{block}{Mini Exploration: What is the best route to school?}
    \begin{enumerate}
      \item Based on Excel model of commuter data (math.huson.com)
      \item Complete written analysis
      \item Email the Excel file and a pdf version of spreadsheet \& paper\\
      (three attachments)
  \end{enumerate}
  \end{block}
  Peer review of draft of subway commute analysis (Oct 28) \\
  Mind map / brainstorming an exploration topic p. 743 (Oct 30)\\
  Homework: Complete your paper, today 10:00PM deadline \\
  Homework: Exploration topic due Nov 4 \\
  Read example (subway platform delays) exploration paper
  }

\section{2.16 Cumulative distribution application, quiz 8 Nov}
\frame
{
  \frametitle{GQ: How do we display and interpret cumulative data?}
  \framesubtitle{CCSS: MP5 attend to precision \hfill \alert{Friday 8 Nov}}

  \begin{block}{2.16 Do Now Quiz: IB problems handout}
  \begin{enumerate}
      \item Write down your exploration topic 
      \item Summarizing frequency table data
      \item Interpreting box plots
  \end{enumerate}
  \end{block}
  Lesson: Comparing quantitative data in Excel, an exploration\\
  Make your own analysis of subway platform crowding versus delays. (use the raw data file on math.huson.com) \\ \smallskip
  Homework: Write up analysis. Email Excel, Word, \& pdf files. Due 10:00 Sunday
}

\section{2a.0 Exploration project paper schedule}
\frame
{
  \frametitle{GQ: How do we employ mathematics to explore a topic?}
  \framesubtitle{CCSS: MP5 attend to precision \hfill \alert{originally Thursday 31 Oct}}
  \begin{block}{Exploration: Schedule and deadlines}
    \begin{enumerate}
      \item Topic selection - Monday November 4th
      \item In class work sessions (you must work at home too)
      \begin{enumerate}
        \item Independent work on introduction, data, mathematics - Nov 11
        \item Complete design of methods, collect data - Nov 15
        \item Apply mathematics, write up methods \& results - Nov 19
        \item Finalize peer review paper, print - Nov 22
      \end{enumerate}
      \item Complete paper for peer review - Friday November 22nd
      \begin{enumerate}
        \item Peer review discussion \& partner feedback - Nov 26
        \item Formatting work \& independent polishing - Dec 3
      \end{enumerate}
      \item Complete paper for grade - Friday December 6th
      \item Final paper - Friday January 17th
    \end{enumerate}
    \end{block}
}

\section{2a.1 Exploration paper student work time (1) 12 Nov}
\frame
{
  \frametitle{GQ: How do we use mathematics to explore a topic?}
  \framesubtitle{CCSS: MP5 attend to precision \hfill \alert{2a.1 Tuesday 12 Nov}}

  \begin{block}{Work on exploration papers}
  \begin{enumerate}
      \item Inputs: what data will you use and how will you get it? 
      \item What mathematics will you apply (find the textbook chapter)
      \item Outputs: What results will you use to answer your aim?
      \item Start drafting and re-drafting your introduction (aim, rationale, personal engagement)
  \end{enumerate}
  \end{block}
  Scoring an exploration paper \\
  Homework: Develop exploration \\
  Homework (Nov13): Read and evaluate sample exploration paper according to criteria pp. 737-740
}

\section{2a.2 Exploration paper student work time (2) 15 Nov}
\frame
{
  \frametitle{GQ: How do we use mathematics to explore a topic?}
  \framesubtitle{CCSS: MP5 attend to precision \hfill \alert{2a.2 Friday 15 Nov}}

  \begin{block}{Work on exploration papers}
  \begin{enumerate}
      \item Inputs: what data will you use and how will you get it? 
      \item What mathematics will you apply (find the textbook chapter)
      \item Outputs: What results will you use to answer your aim?
      \item Start drafting and re-drafting your introduction (aim, rationale, personal engagement)
  \end{enumerate}
  \end{block}
  Homework: Develop exploration 
}

\section{2a.3 Exploration paper student work time (3) 19 Nov}
\frame
{
  \frametitle{GQ: How do we use mathematics to explore a topic?}
  \framesubtitle{CCSS: MP5 attend to precision \hfill \alert{2a.3 Tuesday 19 Nov}}

  \begin{block}{Work on exploration papers - quiet, independent work}
  \begin{enumerate}
      \item Organize your inputs or data. Do not worry about formatting it yet. 
      \item Apply mathematics, probably with technology. Use pencil \& paper for equations for now (reference the textbook)
      \item Study your initial results. Write down what you find! Brainstorm, outline, type up descriptions, findings, reflections. Tie back to your aim.
      \item Re-write your introduction (aim, rationale, personal engagement). Draft the conclusion (perhaps rough). 
  \end{enumerate}
  \end{block}
  Homework: Develop exploration 
}

\section{2a.4 Exploration paper student work time (4) 22 Nov}
\frame
{
  \frametitle{GQ: How do we use mathematics to explore a topic?}
  \framesubtitle{CCSS: MP5 attend to precision \hfill \alert{2a.4 Friday 22 Nov}}
  \begin{block}{Submit exploration papers for peer review - quiet, independent work}
  \begin{enumerate}
      \item Organize and print your inputs or data. Formatting is not critical, but label it clearly (by hand is fine). 
      \item Check mathematics. Include spreadsheets in submission to peer. Pencil \& paper for equations are fine, but organize and write clearly.
      \item Explain the results clearly. Complete descriptions, findings, reflections. Tie back to your aim.
      \item Lock down your introduction (aim, rationale, personal engagement)  conclusion (which must tie back to aim). 
  \end{enumerate}
  \end{block}
  Read peer paper, mark with comments, complete checklist (due Tuesday) 
}

\section{2a.5 Exploration project peer review feedback (5) 26 Nov}
\frame
{
  \frametitle{GQ: How do we give constructive feedback?}
  \framesubtitle{CCSS: MP5 attend to precision \hfill \alert{2a.5 Tuesday 26 Nov}}
  \begin{block}{Discuss and incorporate peer review - quiet partner work}
  \begin{enumerate}
      \item Organize and format. Consider digital versus handwritten alternatives. No cover page nor word count. Color pdf. MLA...MLA...MLA
      \item Check mathematics. Organize spreadsheets. Comply with IB standards (3 sig figs, no calculator notation, proper terminology).
      \item Explain the results clearly. Complete descriptions, findings, reflections. Tie back to your aim.
      \item Rewrite your introduction (aim, rationale, personal engagement) and  conclusion (which must tie back to aim). 
  \end{enumerate}
  \end{block}
  Weekend homework is to revise your project papers. Graded submission \alert{due Friday 6 December} 
}

\section{2a.5 Peer review submission criteria 25 Nov}
\frame
{
  \frametitle{GQ: How do we evaluate a paper without reading it?}
  \framesubtitle{CCSS: MP5 attend to precision \hfill \alert{2a.5 Tuesday 26 Nov}}

  \begin{block}{Scoring of project papers}
  \begin{enumerate}
      \item Submission of partial paper / introduction (55 points) 
      \item Includes data, values, or figures (10 points)
      \item Mathematics has been applied (10 points)
      \item Conclusion (10 points)
      \item Complete paper (5 points)
      \item pdf format attached to email (5 points)
      \item spreadsheet (data) included in email (5 points)
  \end{enumerate}
  \end{block}
  Standard late credit: 80\% (declining per day) 
}

\section{2a.6 Project paper submission format (6) 3 Dec}
\frame
{
  \frametitle{GQ: How do we submit an project?}
  \framesubtitle{CCSS: MP5 attend to precision \hfill \alert{2a.6 Tuesday 3 Dec}}
  \begin{block}{Do Now: ``Dry run'' practice of submission (20 minutes)}
    \begin{enumerate}
      \item Perform MLA quick fixes: (Edit...Select All) \\font Times New Roman 12 point, double space, left justified. 1st page header, page \# \& last name in top right margin.
      \item Make pdf: Print...Save PDF. Filename Lastname-IA-short-title
      \item Rename Excel spreadsheet file Lastname-IA-short-title.xlsx
      \item Email attachments to husonbeca@gmail.com; Subject line: ``IA Dry run - short-title''. Text: ``Please find attached a current draft of my exploration paper and supporting data.'' 
  \end{enumerate}
\end{block}

    Exploration project quiet, individual work\\[0.25cm]
    Graded submission \alert{due Friday 6 December 10:00pm} \\(standard policy 80\% late credit, 70\% ...)
}

\section{2a.7 Project paper deadline (7) 6 Dec}
\frame
{
  \frametitle{GQ: How do we meet the deadline for a project?}
  \framesubtitle{CCSS: MP5 attend to precision \hfill \alert{2a.7 Friday 6 Dec}}
  \begin{block}{Do Now: Review ``dry run'' submission results}
    \begin{enumerate}
      \item Follow MLA format.
      \item Make pdf: Print...Save PDF. Filename Lastname-IA-short-title
      \item Rename Excel spreadsheet file Lastname-IA-short-title.xlsx
      \item Email attachments to husonbeca@gmail.com; Subject line: ``IA Dry run - short-title''. Text: ``Please find attached a current draft of my exploration paper and supporting data.'' 
  \end{enumerate}
\end{block}

    Exploration project quiet, individual work\\[0.25cm]
    Graded submission \alert{due today 10:00pm} \\(standard policy 80\% late credit, 70\% ...)
}


\end{document}

Friday: Desmos graphing of soccer shot angle optimization project 

https://www.economist.com/graphic-detail/2019/10/25/how-being-second-choice-could-put-elizabeth-warren-on-top?cid1=cust/dailypicks1/n/bl/n/20191028n/owned/n/n/dailypicks1/n/n/NA/333375/n