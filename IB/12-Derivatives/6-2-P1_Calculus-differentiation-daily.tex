\documentclass[12pt, twoside]{article}
\usepackage[letterpaper, margin=1in, headsep=0.5in]{geometry}
\usepackage[english]{babel}
\usepackage[utf8]{inputenc}
\usepackage{amsmath}
\usepackage{amsfonts}
\usepackage{amssymb}
\usepackage{tikz}
%\usetikzlibrary{quotes, angles}

\usepackage{graphicx}
\usepackage{enumitem}
\usepackage{multicol}

\usepackage{fancyhdr}
\pagestyle{fancy}
\fancyhf{}
\renewcommand{\headrulewidth}{0pt} % disable the underline of the header

\fancyhead[RE]{\thepage}
\fancyhead[RO]{\thepage \\ Name: \hspace{3cm}}
\fancyhead[L]{BECA / Dr. Huson / 12.1 IB Math SL\\* 4 February 2019}

\begin{document}
\subsubsection*{Spiral Review: 6-2 P1 (No Calculator) Calculus Differentiation - A}
 \begin{enumerate}

  \item 11M.1.sl.TZ2.4 \hfill [6 marks]\\
  Let $h(x)= \frac{6x}{\cos x}$. Find $h'(0)$. \vspace{2cm}

  \item 17M.1.sl.TZ2.6 \hfill [6 marks]\\
  The values of the functions $f$ and $g$ and their derivatives for $x=1$ and $x=8$ are shown in the following table.
  \begin{center}
    \begin{tabular}{|c|c|c|c|c|}
      \hline
      $x$ & $f(x)$ & $f'(x)$ & $g(x)$ & $g'(x)$ \\
      \hline
      1 & 2 & 4 & 9 & $-3$ \\
      \hline
      8 & 4 & $-3$ & 2 & 5 \\
      \hline
    \end{tabular}
  \end{center}
  Let $h(x)=f(x)g(x)$
  \begin{enumerate}
    \item Find $h(1)$ \hfill [2]
    \item Find $h'(8)$ \hfill [3]
  \end{enumerate} \vspace{2cm}


  \item SPNone.1.sl.TZ0.7 \hfill [7 marks]\\
  Given that $f(x)= \frac{1}{x}$, answer the following.
  \begin{enumerate}
    \item Find the first four derivatives of $f(x)$. \hfill [4]
    \item Write an expression for $f^{(n)}(x)$ in terms of $x$ and $n$. \hfill [3]
  \end{enumerate}


  \end{enumerate}
  \newpage

\subsubsection*{Spiral Review: 6-2 P1 (No Calculator) Calculus Differentiation - B}
   \begin{enumerate}


   \item 11M.1.sl.TZ1.5 \hfill [7 marks]\\
   Let $g(x)= \frac{\ln x}{x^2}$, for $x>0$.
   \begin{enumerate}
     \item Use the quotient rule to show that $g'(x)= \frac{1-2\ln x}{x^3}$
     \hfill [4]
     \item The graph of $g$ has a maximum point at $A$. Find the $x$-coordinate of $A$. \hfill [3]
   \end{enumerate} \vspace{2cm}

   \item 09M.1.sl.TZ2.6 \hfill [5 marks]\\
   A function $f$ has its first derivative given by $f'(x)=(x-3)^3$.
   \begin{enumerate}
     \item Find the second derivative. \hfill [2]
     \item Find $f'(3)$ and $f''(3)$. \hfill [1]
     \item The point $P$ on the graph of $f$ has $x$-coordinate 3. Explain why $P$ is not a point of inflexion. \hfill [2]
   \end{enumerate} \vspace{2cm}

   \item 09M.1.sl.TZ2.8 \hfill [6 marks]\\
   Let $f(x)=e^{-3x}$ and $g(x)= \sin(x- \frac{\pi}{3})$.
   \begin{enumerate}
     \item Write down \hfill [2]
     \begin{enumerate}
       \item $f'(x)$;
       \item $g'(x)$.
     \end{enumerate}
     \item Let $h(x)=e^{-3x} \sin(x- \frac{\pi}{3})$. Find the exact value of $h'(\frac{\pi}{3})$. \hfill [4]
   \end{enumerate}

   \end{enumerate}
   \newpage

\subsubsection*{Spiral Review: 6-2 P1 (No Calculator) Calculus Differentiation - C}
  \begin{enumerate}

  \item 09N.1.sl.TZ0.5 \hfill [6 marks]\\
    Consider $f(x)= x^2+\frac{p}{x}$, $x \neq 0$, where $p$ is a constant.
    \begin{enumerate}
      \item Find $f'(x)$ \hfill [2]
      \item There is a minimum value of $f(x)$ when $x=-2$. Find the value of $p$. \hfill [4]
    \end{enumerate} \vspace{1cm}

  \item 14M.1.sl.TZ1.7 \hfill [7 marks]\\
    Let $f(x)= px^3 + px^2 + qx$.
    \begin{enumerate}
      \item Find $f'(x)$. \hfill [2]
      \item Given that $f'(x) \geq 0$, show that $p^2 \leq 3pq$. \hfill [5]
    \end{enumerate}  \vspace{1cm}

  \item 10M.1.sl.TZ1.8 \hfill [14 marks]\\
    Let $f(x)=\frac{1}{3}x^3-x^2-3x$. Part of the graph of $f$ is shown below.
      \begin{center}
        \begin{tikzpicture}[yscale=0.5]
          \draw [thick, ->] (-2,0) -- (5,0) node [below] {$x$};
          \draw [thick, ->] (0,-10) -- (0,3) node [left] {$y$};
          \draw[thick, domain=-2:5] plot[samples=100](\x, {0.333*(\x)^3 - (\x)^2 -3*(\x)});
          \node at (3,-9) {\textbullet};
          \node at (3,-9)[below] {$B$};
          \node at (-1,1.666) {\textbullet};
          \node at (-1,1.666)[above] {$A$};
        \end{tikzpicture}
      \end{center}
    There is a maximum point at $A$ and a minimum point at $B(3, -9)$.
    \begin{enumerate}
      \item Find the coordinates of $A$. \hfill [8]
      \item Write down the coordinates of \hfill [6]
      \begin{enumerate}
        \item the image of $B$ after reflection in the $y$-axis;
        \item the image of $B$ after translation by the vector $\left( \begin{array}{c} -2 \\ 5 \end{array}\right)$;
        \item the image of $B$ after reflection in the $x$-axis followed by a horizontal stretch with scale factor $\frac{1}{2}$.
      \end{enumerate}
    \end{enumerate}

   \end{enumerate}
\end{document}
