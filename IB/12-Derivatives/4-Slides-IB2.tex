\documentclass{beamer}
\usepackage{geometry}
\usepackage[english]{babel}
\usepackage[utf8]{inputenc}
\usepackage{amsmath}
\usepackage{amsfonts}
\usepackage{amssymb}
\usepackage{tikz}
\usepackage{graphicx}
\usepackage{venndiagram}

%\usepackage{pgfplots}
%\pgfplotsset{width=10cm,compat=1.9}
%\usepackage{pgfplotstable}

\setlength{\headheight}{26pt}%doesn't seem to fix warning

\usepackage{fancyhdr}
\pagestyle{fancy}
\fancyhf{}

\lhead{\small{BECA / Dr. Huson / Mathematics}}

\renewcommand{\headrulewidth}{0pt}

\title{Mathematics Class Slides}
\subtitle{Bronx Early College Academy}
\author{Chris Huson}
\date{5 November 2018}

\begin{document}

\frame{\titlepage}

%\section[Outline]{}
%\frame{\tableofcontents}

  \section{4.1 Drui - Graphing, Monday Nov 5}
  \frame
  {
    \frametitle{GQ: How does a function's graph relate to its derivatives?}
    \framesubtitle{CCSS: HSF.IF.B.4 Interpret key features of functions and their graphs \qquad \alert{4.1}}

    \begin{block}{Do Now: Differential calculus}
    \begin{enumerate}
        \item Take the 1st \& 2nd derivatives of $f(x)=x^3-6x^2+6x$.
        \item Sketch the function.\\*
        Challenge: Identify key features, graphically \& algebraically.
    \end{enumerate}
    \end{block}
    Lesson: Function graphs, extrema, the 1st \& 2nd derivative tests p. 233, 240\\%*[5pt]
    Task: 7Q p. 232 \#1-3; 7R p. 234 1, 2; 7S p. 236 1, 3 \\%*[5pt]
    Assessment: Handout graphing problem \#1 (\#2 challenge)
    \\%*[5pt]
    Homework: IB function / graphing problem set
  }

  \section{4.2 Drui - Graphing, Tuesday Nov 6}
  \frame
  {
    \frametitle{GQ: How does a function's graph relate to its derivatives?}
    \framesubtitle{CCSS: HSF.IF.B.4 Interpret key features of functions and their graphs \qquad \alert{4.2}}

    \begin{block}{Do Now: Given $f(x)=x \cos x, 0 \leq x \leq 2\pi$.}
    \begin{enumerate}
        \item Take the 1st \& 2nd derivatives of $f(x)$. \item Sketch the function. \item Over what intervals is the function increasing, decreasing?
    \end{enumerate}
    \end{block}
    Lesson: Function graphs, extrema, the 1st \& 2nd derivative tests p. 233, 240\\%*[5pt]
    Task: 7Q p. 232 \#1-3; 7R p. 234 1, 2; 7S p. 236 1, 3 \\%*[5pt]
    Assessment: Handout graphing problem \#1 (\#2 challenge)
    \\%*[5pt]
    Homework: Test corrections Paper 1
  }

  \section{4.3 Drui - Graphing, Wednesday Nov 7}
  \frame
  {
    \frametitle{GQ: How does a function's graph relate to its derivatives?}
    \framesubtitle{CCSS: HSF.IF.B.4 Interpret key features of functions and their graphs \qquad \alert{4.3}}

    \begin{block}{Do Now: Given $f\left(x\right) =-x^4 +2x^2 +x$. There are $x$-intercepts at $x=0$ and $x=p$. There is a maximum at A where $x=a$, and a point of inflection at B where $x=b$.}
    \begin{enumerate}
        \item Find the value of $p$.
        \item Write down the coordinates of A.
        \item Write down the rate of change of $f$ at A.
        \item Find the coordinates of B.
        \item Write down the rate of change of $f$ at B.
    \end{enumerate}
    \end{block}
    Lesson: The 1st \& 2nd derivative tests p. 233, 240\\%*[5pt]
    Task: 7Q p. 232 \#1-3; 7R p. 234 1, 2; 7S p. 236 1, 3 \\%*[5pt]
    Assessment: Calculator calculus functions in Do Now.
    \\%*[5pt]
    Homework: Handout IB function / graphing problem set
  }

  \section{4.4 Drui - Graphing, Thursday Nov 8}
  \frame
  {
    \frametitle{GQ: How does a function's graph relate to its derivatives?}
    \framesubtitle{CCSS: HSF.IF.B.4 Interpret key features of functions and their graphs \qquad \alert{4.4}}

    \begin{block}{Do Now: Find the 1st derivative of the function and solve for it's zeros as potential extrema (stationary points). Use the 1st derivative test to determine whether it is a max, min, or neither.}
    \begin{enumerate}
        \item $f(x)=x^3$.
        \item $\displaystyle f(x)=\frac{x^2-4}{x^2-1}$
    \end{enumerate}
    \end{block}
    Lesson: The 1st \& 2nd derivative tests p. 233, 240\\%*[5pt]
    Task: Homework review; 7Q p. 232 \#1-3; 7R p. 234 1, 2; 7S p. 236 1, 3 \\%*[5pt]
    Assessment: Use of the 1st \& 2nd derivative tests
    \\%*[5pt]
    Homework: Study for test \alert{tomorrow}
  }

  \section{4.5 Drui - Graphing, Friday Nov 9}
  \frame
  {
    \frametitle{GQ: How do we prepare for the IB final exams?}
    \framesubtitle{CCSS: HSF.IF.B.4 Interpret key features of functions and their graphs \qquad \alert{4.5}}

    \begin{block}{Do Now: 1st \& 2nd derivatives of a cubic function, sketch}
      \begin{enumerate}
      \item Given the function $f(x)=x^3-9x$
      \item Find $f'(x)$ and $f''(x)$.
      \item Sketch $f$ and its two derivatives on the same set of axes. Label the intersections and extrema.
      \end{enumerate}
   \end{block}
    Lesson: Last minute study practices (reflection) \\[5pt]
    Task: Homework review: Work homework problems on board\\%*[5pt]
    Assessment: Problem set and exam mark scheme\\%*[5pt]
    Homework: Prepare for final exams
  }

  \frame
  {
    \frametitle{Interpreting a displacement vs time graph}
    \framesubtitle{CCSS: F.IF.B.6 Calculate \& interpret the rate of change of a function}

    \begin{block}{Consider the function $f(x)=-x^2+2x+3$}
    \begin{enumerate}
        \item Factor $f$ and state its zeros.
        \item Restate $f$ in vertex form. Write down the vertex as an ordered pair.
        \item Over what intervals is the function increasing, decreasing, and neither?
        \item If $f(x)$ represents the height of a diver over the domain $0 \leq x \leq 3$, interpret $f(0)$, the vertex, and $f(3)$
        \item What does the ``slope" of the curve represent?
    \end{enumerate}
    \end{block}
  }

  \section{4.6 Drui - Graphing, Tuesday Nov 13}
  \frame
  {
    \frametitle{GQ: How do we prepare for the IB final exams?}
    \framesubtitle{CCSS: HSF.IF.B.4 Interpret key features of functions and their graphs \qquad \alert{4.6}}

    \begin{block}{Do Now: 1st \& 2nd derivatives of a cubic function, sketch}
      \begin{enumerate}
      \item Given the function $f(x)=x^3-9x$
      \item Find $f'(x)$ and $f''(x)$.
      \item Sketch $f$ and its two derivatives on the same set of axes. Label the intersections and extrema.
      \end{enumerate}
   \end{block}
    Lesson: Last minute study practices (reflection) \\[5pt]
    Task: Homework review: Work homework problems on board\\%*[5pt]
    Assessment: Problem set and exam mark scheme\\%*[5pt]
    Homework: Prepare for final exams
  }

\end{document}
