\documentclass[12pt, oneside]{article}
\usepackage[letterpaper, margin=1in, headsep=0.5in]{geometry}
\usepackage[english]{babel}
\usepackage[utf8]{inputenc}
\usepackage{amsmath}
\usepackage{amsfonts}
\usepackage{amssymb}
\usepackage{tikz}
%\usepackage{tkz-fct}
\usepackage{pgfplots}
\pgfplotsset{width=10cm,compat=1.9}
\usepgfplotslibrary{statistics}
\usepackage{pgfplotstable}
%\usepackage{venndiagram}

\usepackage{fancyhdr}
\pagestyle{fancy}
\fancyhf{}
\rhead{\thepage \\Name: \hspace{1.5in}.\\}
\lhead{BECA / Dr. Huson / 11.1 IB Math SL\\* 7 May 2018 \\*\textbf{Homework: Exponential functions, imaginary numbers, sequences, logs
}}

\renewcommand{\headrulewidth}{0pt}

\begin{document}
\begin{enumerate}

\item A bank account earns interest at a continuous interest rate of 5\% per year. The initial deposit is \$225.
\begin{enumerate}
    \item Express the balance in the account as a function in the form $P(t)=P_0 \cdot e^{rt}$\\[30pt]
    \item Convert the function to one without a coefficient in the exponent. \\[30pt]
    \item What is the interest rate expressed as a simple, annual rate?\\[30pt]
\end{enumerate}

\item Judith puts \$5000 into an investment account with interest compounded continuously. If the annual interest rate is 3.25\% what is the balance after 30 years?\\[60pt]

\item Lisa puts \$1000 into an investment account with interest compounded continuously. What is the approximate annual rate needed for the account to grow to \$1529.59 after 10 years?\\[60pt]


\item The function below models the average price of gas in a small town since January 1st.
\[G(t)=-0.0049t^4 + 0.0923t^3 - 0.56t^2 +1.166t+3.23 \text{, where } 0 \leq t \leq 10.\]
If $G(t)$ is the average price of gas in dollars and $t$ represents the number of months since January 1st, the absolute maximum $G(t)$ reaches over the given domain is about what value, to the nearest cent? (graph the function in your calculator and use the Max function)%Alg2 Regents Jan2018

\newpage


\item Write $\sqrt[3]x^8$ as a single term with a rational exponent.\\*[30pt]

\item Write $\sqrt{a^3} \div a^{\frac{1}{2}}$ as an expression with positive, integer exponents.\\*[30pt]

\item If $n=\sqrt{z^5}$ and $m=z^{\frac{7}{2}}$, where $a > 0$, express $\frac{n}{m}$ as a radical with positive, integer exponents. \\*[30pt]

\item What is the expression $5i^3(-2i+5)$ is equivalent to? Express your answer in the form $a+bi$, where $a, b \in \mathbb{R}$.\\*[50pt]  %Alg2 Regents Jun2017 multiple choice

\item Simplify the expression $(2x - i)^2$, where $i$ is the imaginary unit. Express your answer in the form $a+bi$, where $a, b \in \mathbb{R}$.\\[50pt] %Alg2 Regents Aug2017

\item Algebraically determine the values of $h$ and $k$ to correctly complete the identity stated below.
\[3x^3-7x^2+5x-7=(x-2)(3x^2+hx+3)+k\] %Alg2 Regents Jan 2017

\newpage
\item The expression $(x + a)(x + b)$ can not be written as
\begin{enumerate}
    \item $a(x + b)+ x(x + b)$
    \item $x^2 + (a + b)x + ab$ 
    \item  $x^2 + abx + ab$  
    \item $x(x + a)+ b(x + a)$
\end{enumerate}

\item In an arithmetic sequence, the first term is 3 and the second term is 7.
\begin{enumerate}
    \item Find the common difference.
        \begin{flushright}[2]\end{flushright}
    \item Find the tenth term.
        \begin{flushright}[2]\end{flushright}
    \item Find the sum of the first ten terms of the sequence. 
        \begin{flushright}[2]\end{flushright}
\end{enumerate}

\item Consider a geometric sequence where the first term is 768 and the second term is 576.\\
Find the least value of $n$ such that the $n$th term of the sequence is less than 7.
    \begin{flushright}[6]\end{flushright}

\item Let $x=\ln 7$ and $y= \ln 3$. Write the following expressions in terms of $x$ and $y$.
\begin{enumerate}
    \item $\ln \left( \frac{3}{7} \right)$.
        \begin{flushright}[2]\end{flushright}
    \item $\ln 63$.
        \begin{flushright}[4]\end{flushright}
\end{enumerate}


\item Let $f(x)= k \log_2 x$.
\begin{enumerate}
    \item Given that $f^{-1}(1) = 8$, find the value of $k$.
        \begin{flushright}[3]\end{flushright}
    \item Find $f^{-1}(\frac{2}{3})$
        \begin{flushright}[4]\end{flushright}
\end{enumerate}



\end{enumerate}
\end{document}