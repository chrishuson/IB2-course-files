\documentclass{beamer}
\usepackage{geometry}
\usepackage[english]{babel}
\usepackage[utf8]{inputenc}
\usepackage{amsmath}
\usepackage{amsfonts}
\usepackage{amssymb}
\usepackage{tikz}
\usepackage{graphicx}
\usepackage{venndiagram}

%\usepackage{pgfplots}
%\pgfplotsset{width=10cm,compat=1.9}
%\usepackage{pgfplotstable}

\setlength{\headheight}{26pt}%doesn't seem to fix warning

\usepackage{fancyhdr}
\pagestyle{fancy}
\fancyhf{}

%\rhead{\small{24 September 2018}}
\lhead{\small{BECA / Dr. Huson / Mathematics}}

%\vspace{1cm}

\renewcommand{\headrulewidth}{0pt}


\title{Mathematics Class Slides}
\subtitle{Bronx Early College Academy}
\author{Chris Huson}
\date{19 September - 5 October 2018}

\begin{document}

\frame{\titlepage}

\section[Outline]{}
\frame{\tableofcontents}

\section{2.1 Drui - Limits, Tuesday Sept 18}
  \frame
  {
    \frametitle{GQ: How do we work with infinite sequences?}
    \framesubtitle{CCSS: HSF.LB.B.5 Interpret the parameters in an exponential function \hspace{\stretch{1}} \alert{2.1}}

    \begin{block}{Do Now: Sequences and series. Use the formula sheet.}
      \begin{enumerate}
      \item Find the third term of an arithmetic sequence starting with 7 and having a constant difference of 2.
      \item Find the sum of the first 12 terms of the above sequence.
      \item Find the sum of the first 7 terms of a geometric sequence with $u_1=5$ and $r=1.1$.
      \item Find the sum of an infinite geometric series with $u_1=1$ and $r=\frac{1}{2}$.
      \end{enumerate}
   \end{block}
    Lesson: Limits and convergence pp. 194-200\\ \bigskip
    Homework: Exercises from 7A, 7B p. 197, 200
  }

  \section{2.2 Drui - Limits, Thursday Sept 20}
    \frame
    {
      \frametitle{GQ: How do we define the slope of a curve?}
      \framesubtitle{CCSS: HSF.LB.B.5 Interpret the parameters in an exponential function \hspace{\stretch{1}} \alert{2.2}}

      \begin{block}{Do Now: Point-slope formula of a line}
        \begin{enumerate}
        \item Find the equation of a linear function through $(1,2)$ with slope of 2.\\
        If you can't do the algebra, use graph paper.
        \item Find the equation of the line through the points $\displaystyle (\frac{1}{2}, \frac{1}{4})$ and $\displaystyle (\frac{3}{2}, \frac{9}{4})$.
        \item Sketch the two lines and the curve $y=x^2$.
        \end{enumerate}
     \end{block}
      Lesson: Tangent lines and the derivative pp. 200-205\\ \bigskip
      Homework: Exercises from 7C, 7D, 7E p. 201-3
    }

  \section{2.3 Drui - Tangent and normal lines, Monday Sept 24}
    \frame
    {
      \frametitle{GQ: How do we differentiate equations with multiple terms?}
      \framesubtitle{CCSS: HSF-IF.B.6 Interpret functions, and their rate of change \hspace{\stretch{1}} \alert{2.3}}

      \begin{block}{Do Now: Differentiating using the power rule}
        \begin{enumerate}
        \item Exercises 7F \#1-3 p. 205
        \item Convert expressions to terms having a power of $x$.
        \item Apply power rule.
        \end{enumerate}
     \end{block}
      Lesson: Tangent and normal lines and the derivative pp. 205-7\\ \bigskip
      Homework: Exercises 7G p. 207
    }

\section{2.4 Drui - Calculator practice, Tuesday Sept 25}
  \frame
  {
    \frametitle{GQ: How do we differentiate exponential functions?}
    \framesubtitle{CCSS: HSF-IF.B.6 Interpret functions, and their rate of change \hspace{\stretch{1}} \alert{2.4}}

    \begin{block}{Do Now: Differentiate polynomials to determine tangent and normal lines}
      \begin{enumerate}
      \item Differentiate the function $f(x)=x^3-3x^2+x-5$
      \item Find the slope of the tangent and normal to the function at $x=2$
      \item Find the equations of the tangent and normal lines to the function at $x=2$
      \item Sketch the two lines and the function curve.
      \end{enumerate}
   \end{block}
    Lesson: The derivatives of $e^x$ and $\ln x$ pp. 208-208\\ \bigskip
    Homework: Exercises from 7H p. 209
  }

\section{2.5 Drui - Product rule, Wednesday Sept 26}
  \frame
  {
    \frametitle{GQ: How do we differentiate the product of two functions?}
    \framesubtitle{CCSS: HSF-IF.B.6 Interpret functions, and their rate of change \hspace{\stretch{1}} \alert{2.5}}

    \begin{block}{Do Now: Differentiate each function}
      \begin{enumerate}
      \item $y= \ln x$
      \item Constant multiple rule: $y=5e^x$
      \item Derivative sum rule: $y=x^3+e^x$
      \item Find the equations of the tangent and normal lines to the function $f(x)=x+ \ln x$ at $x=2$
      \item Spicy: from first principles, differentiate $y=x^3$ using the definition of the derivative as a limit of the difference quotient.
      \end{enumerate}
   \end{block}
    Lesson: The product rule pp. 210-211\\ \bigskip
    Homework: Exercises from 7I \#2, 3, 4, 7, 9 p. 212
  }

\section{2.6 Drui - Quotient rule, Thursday Sept 27}
  \frame
  {
    \frametitle{GQ: How do we differentiate the quotient of two functions?}
    \framesubtitle{CCSS: HSF-IF.B.6 Interpret functions, and their rate of change \hspace{\stretch{1}} \alert{2.6}}

    \begin{block}{Do Now: Differentiate each function}
      \begin{enumerate}
      \item $y= -\frac{1}{4}x^2+6x-8+\frac{1}{x}$
      \item $y=5e^x(x+7)$
      \item Find the equations of the tangent and normal lines to the function $f(x)=x e^x$ at $x=1$. State the exact solution (not a decimal approximation).
      \item Spicy: from first principles, differentiate $y=x^3$ using the definition of the derivative as a limit of the difference quotient.
      \end{enumerate}
   \end{block}
    Lesson: The quotient rule pp. 210-212\\ \bigskip
    Homework: Exercises from 7I \#1, 3, 5, 6, 8, 10 p. 212
  }

\section{2.7 Drui - Quotient rule, Friday Sept 28}
  \frame
  {
    \frametitle{GQ: How do we differentiate functions?}
    \framesubtitle{CCSS: HSF-IF.B.6 Interpret functions, and their rate of change \hspace{\stretch{1}} \alert{2.7}}

    \begin{block}{Do Now: Differentiate each function. Use $\frac{\mathrm{d}y}{\mathrm{d}x}$ notation.}
      \begin{enumerate}
      \item $y= \frac{1}{3}x^3-9x$. Find the points on the function where the tangent lines are horizontal.
      \item $y=\sin x$. (Take out your formula sheet)
      \item Find $f(x)=(x^2-1) 2^x$ at $x=2$.
      \item Spicy: from first principles, differentiate $y=x^3$ using the definition of the derivative as a limit of the difference quotient.
      \end{enumerate}
   \end{block}
    Lesson: Differentiation practice Example 10, 11 pp. 213-214\\
    Higher order derivatives \\
    Example exam problems\\ \bigskip
    Homework: Exercises from 7J p. 214-5
  }

\section{2.8 Drui - Deltamath differentiation practice, Tuesday Oct 2}
  \frame
  {
    \frametitle{GQ: How do we differentiate functions?}
    \framesubtitle{CCSS: HSF-IF.B.6 Interpret functions, and their rate of change \hspace{\stretch{1}} \alert{2.8}}

    \begin{block}{Do Now: Given $y= x^4-2x^2$. \\Use $\frac{\mathrm{d}y}{\mathrm{d}x}$ notation for the Do Now.}
      \begin{enumerate}
      \item Find $\frac{\mathrm{d}y}{\mathrm{d}x}$ at $x=2$.
      \item Find the points on the function where the tangent lines are horizontal.
      \item Find $\frac{\mathrm{d}^2y}{\mathrm{d}x^2}$
      \item Spicy: from first principles, differentiate $y=x^3$ using the definition of the derivative as a limit of the difference quotient.
      \end{enumerate}
   \end{block}
    Lesson: Differentiation practice Example 13, 14 pp. 218-219\\
    Example exam problems\\ \bigskip
    Homework: Exercises from 7L odds p. 219
  }

\section{2.9 Drui - Pretest Review, Wednesday Oct 3}
  \frame
  {
    \frametitle{GQ: How do we differentiate functions?}
    \framesubtitle{CCSS: HSF-IF.B.6 Interpret functions, and their rate of change \hspace{\stretch{1}} \alert{2.9}}

    \begin{block}{Do Now: Review handout}
    \end{block}
    Lesson: Differentiation practice pp. 194-219\\
    Example exam problems\\ \bigskip
    Homework: Study for exam
  }

\section{2.10 Drui - Exam, Thursday Oct 4}
  \frame
  {
    \frametitle{GQ: How do we differentiate functions?}
    \framesubtitle{CCSS: HSF-IF.B.6 Interpret functions, and their rate of change \hspace{\stretch{1}} \alert{2.10}}

    \begin{block}{Exam}
    \end{block}
    Scope: Differentiation practice pp. 194-219\\
    Review problems\\ \bigskip
    Homework: Exercises from 7M odds p. 221
  }

\section{2.10x Drui - Trig review, Friday Oct 5}
  \frame
  {
    \frametitle{GQ: How do we se the law of cosines?}
    \framesubtitle{CCSS: HSF-IF.B.6 Interpret functions, and their rate of change \hspace{\stretch{1}} \alert{2.x}}

    \begin{block}{Exam}
    \end{block}
    Scope: Triangle problems, circle sectors\\ \bigskip
    Homework: Handout 2-10x-Trig-Functions.pdf
  }

\section{2b.1 Drui - Deltamath differentiation practice, Tuesday Oct 9}
  \frame
  {
    \frametitle{GQ: How do we differentiate functions?}
    \framesubtitle{CCSS: HSF-IF.B.6 Interpret functions, and their rate of change \hspace{\stretch{1}} \alert{2b.1}}

    Lesson: Differentiation practice \\ \bigskip
    Homework: Complete Deltamath problem set at home
  }


\section{2b.2 Drui - Chain rule, Wednesday Oct 10}
  \frame
  {
    \frametitle{GQ: How do we differentiate composite functions?}
    \framesubtitle{CCSS: HSF-IF.B.6 Interpret functions, and their rate of change \hspace{\stretch{1}} \alert{2b.2}}

    \begin{block}{Do Now: Given $y= x^4-2x^2$. \\Use $\frac{\mathrm{d}y}{\mathrm{d}x}$ notation for the Do Now.}
      \begin{enumerate}
      \item Find $\frac{\mathrm{d}y}{\mathrm{d}x}$ at $x=2$.
      \item Find the points on the function where the tangent lines are horizontal.
      \item Find $\frac{\mathrm{d}^2y}{\mathrm{d}x^2}$
      \item Spicy: from first principles, differentiate $y=x^3$ using the definition of the derivative as a limit of the difference quotient.
      \end{enumerate}
   \end{block}
    Lesson: Differentiation practice Example 12 pp. 215-216\\
    Example exam problems\\ \bigskip
    Homework: Exercises from 7K p. 217
  }

\section{2b.3 Drui - Chain rule, Thursday Oct 11}
  \frame
  {
    \frametitle{GQ: How do we differentiate composite functions?}
    \framesubtitle{CCSS: HSF-IF.B.6 Interpret functions, and their rate of change \hspace{\stretch{1}} \alert{2b.3}}

    \begin{block}{Do Now: Given $y= (2x-1)^2$. Use $\frac{\mathrm{d}y}{\mathrm{d}x}$ notation for the Do Now.}
      \begin{enumerate}
      \item Find $\frac{\mathrm{d}y}{\mathrm{d}x}$ at $x=1$.
      \item Find the equation of the tangent line to the function at $x=1$
      \item Find the points on the function where the tangent lines are horizontal.
      \item Find $\frac{\mathrm{d}^2y}{\mathrm{d}x^2}$
      \end{enumerate}
   \end{block}
    Lesson: Differentiation practice Example 14 pp. 218\\
    \bigskip
    Homework: Exercises from 7L p. 219; Unit circle worksheet
  }

\section{2b.4 Drui - Higher order derivatives, Friday Oct 12}
  \frame
  {
    \frametitle{GQ: How do we calculate higher order derivatives?}
    \framesubtitle{CCSS: HSF-IF.B.6 Interpret functions, and their rate of change \hspace{\stretch{1}} \alert{2b.4}}

    \begin{block}{Do Now: Use $\frac{\mathrm{d}y}{\mathrm{d}x}$ notation for the Do Now.}
      \begin{enumerate}
      \item Given $y=e^{x^2}$. Find $\frac{\mathrm{d}y}{\mathrm{d}x}$.
      \item Given $y=x^3-6x$. Find the points on the function where the tangent lines are horizontal.
      \item For the cubic function in \#2, find the value of $x$ such that $\frac{\mathrm{d}^2y}{\mathrm{d}x^2}=0$
      \item For the cubic function in \#2, find the zeros of  $\frac{\mathrm{d}^3y}{\mathrm{d}x^3}$
      \end{enumerate}
   \end{block}
    Lesson: Rates of change. Example 16 pp. 221\\
    \bigskip
    Homework: Exercises from 7M odds, 7N odds p. 221-3
  }

\section{2b.5 Drui - Kinematics, Monday Oct 15}
  \frame
  {
    \frametitle{GQ: How do we calculate displacement and instantaneous velocity?}
    \framesubtitle{CCSS: HSF-IF.B.6 Interpret functions, and their rate of change \hspace{\stretch{1}} \alert{2b.4}}

    \begin{block}{Do Now handout, mixed review.}
      Study these problems and practice these skills.
    \end{block} \bigskip
    Lesson: Motion in a line, velocity. Example 18 pp. 224\\
    \bigskip
    Homework: Exercises from 7N evens p. 223, 7O \#2 p. 225
  }

\section{2b.6 Drui - Deltamath differentiation practice, Tuesday Oct 16}
  \frame
  {
    \frametitle{GQ: How do we differentiate functions?}
    \framesubtitle{CCSS: HSF-IF.B.6 Interpret functions, and their rate of change \hspace{\stretch{1}} \alert{2b.6}}

    \begin{block}{Do Now quiz, mixed review.}
      Complete the problem set without a calculator, then begin Deltamath.
    \end{block} \bigskip

    Lesson: Differentiation practice \\ \bigskip
    Homework: Complete Deltamath problem set at home
  }

\section{2b.7 Drui - Kinematics, Wednesday Oct 17}
  \frame
  {
    \frametitle{GQ: How do we calculate displacement and instantaneous velocity?}
    \framesubtitle{CCSS: HSF-IF.B.6 Interpret functions, and their rate of change \hspace{\stretch{1}} \alert{2b.7}}

    \begin{block}{Do Now: 7O \#1, 3 p. 225}
      For these problems study the line graphs at the top of page 225.
    \end{block} \bigskip
    Lesson: Motion in a line, velocity. Example 19 pp. 226\\
    \bigskip
    Homework: Exercises 7P p. 229
  }

  % Ballistics problem 2K \#1 p. 55

\section{2b.8 Drui - Graphing, Thursday Oct 18}
  \frame
  {
    \frametitle{GQ: How does a function's graph relate to its derivatives?}
    \framesubtitle{CCSS: HSF.IF.B.4 Interpret key features of functions and their graphs \qquad \alert{2b.8}}

    \begin{block}{Do Now: Differential calculus}
    \begin{enumerate}
        \item Take the 1st \& 2nd derivatives of $f(x)=x^3-6x^2+6x$.
        \item Sketch the function.\\*
        Challenge: Identify key features, graphically \& algebraically.
    \end{enumerate}
    \end{block}
    Lesson: Function graphs, extrema, the 1st \& 2nd derivative tests p. 233, 240\\%*[5pt]
    Task: 7Q p. 232 \#1-3; 7R p. 234 1, 2; 7S p. 236 1, 3 \\%*[5pt]
    Assessment: Handout graphing problem \#1 (\#2 challenge)
    \\%*[5pt]
    Homework: IB function / graphing problem set
  }

  \section{2b.9 Drui - Graphing, Friday Oct 19}
  \frame
  {
    \frametitle{GQ: How does a function's graph relate to its derivatives?}
    \framesubtitle{CCSS: HSF.IF.B.4 Interpret key features of functions and their graphs \qquad \alert{2b.9}}

    \begin{block}{Do Now: Given $f(x)=x \cos x, 0 \leq x \leq 2\pi$.}
    \begin{enumerate}
        \item Take the 1st \& 2nd derivatives of $f(x)$. \item Sketch the function. \item Over what intervals is the function increasing, decreasing?
    \end{enumerate}
    \end{block}
    Lesson: Function graphs, extrema, the 1st \& 2nd derivative tests p. 233, 240\\%*[5pt]
    Task: 7Q p. 232 \#1-3; 7R p. 234 1, 2; 7S p. 236 1, 3 \\%*[5pt]
    Assessment: Handout graphing problem \#1 (\#2 challenge)
    \\%*[5pt]
    Homework: Test corrections Paper 1
  }

\end{document}
