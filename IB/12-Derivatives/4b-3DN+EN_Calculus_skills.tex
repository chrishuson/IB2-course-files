\documentclass[12pt, twoside]{article}
\usepackage[letterpaper, margin=1in, headsep=0.5in]{geometry}
\usepackage[english]{babel}
\usepackage[utf8]{inputenc}
\usepackage{amsmath}
\usepackage{amsfonts}
\usepackage{amssymb}
\usepackage{tikz}
%\usetikzlibrary{quotes, angles}
\usepackage{graphicx}
\usepackage{enumitem}
\usepackage{multicol}

\usepackage{fancyhdr}
\pagestyle{fancy}
\renewcommand{\headrulewidth}{0pt} % disable the underline of the header

\fancyhead[RO]{Name: \hspace{1.5in}}
\lhead{BECA / Dr. Huson / 12.1 IB Math SL\\* 7 December 2018}


\begin{document}

\subsubsection*{Do Now: Calculus skills review}

\begin{enumerate}

\subsubsection*{Differentiate each function.}

  \item $f(x)=2x^3-x^2+6$
  \item $g(x) = \frac{1}{x^4}$
  \item $f(x)=(x-2)(x+2)$
  \item $g(x)=\frac{2}{(3x)^2}$
  \item $h(x)=6e^x+\sqrt{x^3}$
  \item $f(x)=6\ln{x}$
  \item $g(x) = \displaystyle \frac{\sin x}{\pi}$
  \item $h(x)= \log_3{x}$

\subsubsection*{Use the product, quotient, or chain rule to differentiate each function.}
  \item $y=2xe^x$
  \item $f(x)= \displaystyle \frac{x^2-8}{x+1}$
  \item $g(x) =\displaystyle \frac{x}{x^2-x+6}$
  \item $y = (3x^3-x^2+4)^4$
  \item $y= \ln{2x^2-3x}$
  \item $y= \displaystyle \cos{\frac{x^3}{\pi}}$
  \item $y = \sqrt{\cos{3x}}$

\subsubsection*{Find the equation of the tangent or normal line to the function at the given point.}
  \item The tangent to $y=3x^2-e^x$ at $x=2$
  \item The normal line to $\displaystyle y=\ln(e^{x^2})$

\subsubsection*{Local extrema: find the value(s) of $x$ for which the function has a local minimum or maximum.}
  \item $g(x) = x^3-4x^2-6x+5$
  \item $h(x) = 2\ln x - x+4$


\subsubsection*{Rates of change and motion equations}

\item The path of a diver is modeled by the function $s(t)=-4.9t^2+4.9t+10$ where $s$ is the diver's height above the water in meters.
  \begin{enumerate}
      \item What is the initial height from which the diver begins her dive?
      \item What is the initial velocity of the diver?
      \item What is the maximum height above the water and at what point in time is that height reached?
      \item When does the diver enter the water?
      \item At what velocity does she enter the water?
  \end{enumerate}

\item The position of an object is given by the function $s(x) = 5\sin x +x$ over the interval $\{0\leq x  \leq 2\pi \}$
  \begin{enumerate}
      \item What is the object's initial velocity?
      \item At what value of $x$ is the object at its maximum distance from its starting point?
      \item What is its average velocity over the period from $x=0$ to when it achieves its maximum distance?
      \item Over what interval is the object moving in the negative direction?
  \end{enumerate}


\item A particle moves along a horizontal line with its displacement given by the function $s(t) = 20t - 100 \ln{t}$, for $t>1$.
  \begin{enumerate}
      \item Find the velocity of the particle.
      \item Over what period is the particle moving to the left?
      \item Show that the velocity of the particle is always increasing.
  \end{enumerate}

\end{enumerate}


\newpage
\setcounter{page}{1}
\subsubsection*{Exit Note: Calculus skills assessment}

\begin{enumerate}
  \item Given $f(x)= 3\sqrt{x}$. Find $f'(x)$. \vspace{4cm}
  \item Given $y= x^2\cos{x}$. Find $\displaystyle \frac{\text{d}x}{\text{d}y}$. \vspace{5cm}
  \item Find the equation of the tangent line to $y=4x^2$ at $x=1$. \vspace{5cm}
  \item Early finishers: Find the values of $x$ for which the function $f(x) = 8x^2-24x+7$ has a local minimum or maximum.


\end{enumerate}

\end{document}
