\documentclass{beamer}
\usepackage{geometry}
\usepackage[english]{babel}
\usepackage[utf8]{inputenc}
\usepackage{amsmath}
\usepackage{amsfonts}
\usepackage{amssymb}
\usepackage{tikz}
\usepackage{graphicx}
\usepackage{venndiagram}

%\usepackage{pgfplots}
%\pgfplotsset{width=10cm,compat=1.9}
%\usepackage{pgfplotstable}

\setlength{\headheight}{26pt}%doesn't seem to fix warning

\usepackage{fancyhdr}
\pagestyle{fancy}
\fancyhf{}

%\rhead{\small{5 September 2018}}
\lhead{\small{BECA / Dr. Huson / 11.1 IB Math Unit 1}}

%\vspace{1cm}

\renewcommand{\headrulewidth}{0pt}


\title{Mathematics Class Slides}
\subtitle{Bronx Early College Academy}
\author{Chris Huson}
\date{24 September - 5 October 2018}

\begin{document}

\frame{\titlepage}

\section[Outline]{}
\frame{\tableofcontents}

\section{2.1 Drui: Composite functions} %Monday Sept 24
  \frame
  {
    \frametitle{GQ: How do we combine functions?}
    \framesubtitle{CCSS: HSF.IF.C.7 Analyze functions  \hspace{\stretch{1}}  \alert{2.1}}

    \begin{block}{Do Now: Textbook exercises 1E \# 3-5 pp. 13-14}
    \begin{enumerate}
        \item Write practice problems on loose leaf paper
        \item In your notebook, write the Guiding Question and the date
        \item Take out homework, calculator
    \end{enumerate}
    \end{block}
    Lesson: Function composition, operations p 14-15 \\*[5pt]
    Homework: Problem set: Function operations
  }
%Prepare copies of formula sheets

\section{2.2 Drui: Laptop work with Desmos and Word. Tuesday Sept 25}
  \frame
  {
    \frametitle{How do we communicate mathematical results?}
    \framesubtitle{CCSS: MP.4 Model with mathematics \hspace{\stretch{1}} \alert{2.2}}

    \begin{block}{Technical skills needed to communicate mathematics}
    \begin{enumerate}
        \item Word processing: Microsoft Word and equation editor
        \item Computer calculators: Desmos; domain restriction, labeling
        \item Cloud storage: Dropbox
        \item Technical writing standards: MLA format (Purdue OWL)
        \item Writing style: declarative
        \item Assessment criteria: IB exploration criterion \emph{B: Mathematics Presentation}
    \end{enumerate}
    \end{block}
    Lesson: Shared folder structure, graph copy/paste, MLA template\\ \bigskip
    Homework: Pre-test
  }

\section{2.3 Drui: Inverse functions, Wednesday Sept 26}
  \frame
  {
    \frametitle{GQ: How do we find the inverse of functions?}
    \framesubtitle{CCSS: HSF.IF.C.7 Analyze functions  \hspace{\stretch{1}}  \alert{2.3}}

    \begin{block}{Do Now: Function composition, operations.}
    \begin{enumerate}
        \item Given $f(x)=x-5$ and $g(x)=x^2$. Find $f+g$, $f \circ g$, and $(g \circ f)(3)$
        \item If $r(x)=x-3$ and $s(x)=x^2$, find $(r \circ s)(x)$ and state its domain and range.
    \end{enumerate}
    \end{block}
    Lesson: Function inverses p 16-19; Exercise 1G p.18-19 \\*[5pt]
    Homework: Problem set: Function inverses
  }

\section{2.4 Drui: Inverse functions, Thursday Sept 27}
  \frame
  {
    \frametitle{GQ: How do we find the inverse of functions algebraically?}
    \framesubtitle{CCSS: HSF.IF.C.7 Analyze functions  \hspace{\stretch{1}}  \alert{2.4}}

    \begin{block}{Do Now: Function composition, inverses.}
    \begin{enumerate}
        \item Given $f(x)=2x-1$ and $g(x)=x^2+1$. Find $f+g$, $f \circ g$, and $(g \circ f)(-1)$.
        \item Graph the function $h=\{(-1,0),(1, 2),(3, 1), (4,5)\}$ and its inverse $h^{-1}$.
        \item Sketch $f(x)=e^x$ and its inverse $f^{-1}(x)=\ln x$. (use your calculator table function)
    \end{enumerate}
    \end{block}
    Review formula sheets\\
    Lesson: Function inverses p 19-20; Exercise 1H p.1-8 \\*[5pt]
    Homework: Problem set: Function inverses
  }

\section{2.5 Drui: Transformations, Monday Oct 1}
  \frame
  {
    \frametitle{GQ: How do we transform functions?}
    \framesubtitle{CCSS: HSF.IF.C.7 Analyze functions  \hspace{\stretch{1}}  \alert{2.5}}

    \begin{block}{Do Now: Function composition, inverses.}
    \begin{enumerate}
        \item Given $f(x)=x-2$ and $g(x)=x^2-1$. Find $f \circ g$, and $(g \circ f)(3)$.
        \item Find the inverse of the function $h(x)=5x+2$.
        \item Given a quadratic function with vertex $(3,2)$ and leading coefficient $a=1$. Write down the function in vertex form. Factor the function and state the zeros. Show that in standard form the function is $y=x^2-6x+11$.
    \end{enumerate}
    \end{block}
    Review formula sheets\\
    Lesson: Function inverses p 21-24; Exercise 1I p.1-7 \\*[5pt]
    Homework: Problem set: Function inverses
  }

\section{2.6 Drui: Laptop work with Desmos and Word. Tuesday Oct 2}
  \frame
  {
    \frametitle{How do we communicate mathematical results?}
    \framesubtitle{CCSS: MP.4 Model with mathematics \hspace{\stretch{1}} \alert{2.6}}

    \begin{block}{Technical skills needed to communicate mathematics}
    \begin{enumerate}
        \item Word processing: Microsoft Word and equation editor
        \item Computer calculators: Desmos; domain restriction, labeling
        \item Cloud storage: Dropbox
        \item Technical writing standards: MLA format (Purdue OWL)
        \item Writing style: declarative
        \item Assessment criteria: IB exploration criterion \emph{B: Mathematics Presentation}
    \end{enumerate}
    \end{block}
    Lesson: Shared folder structure, graph copy/paste, MLA template\\ \bigskip
    Homework: Function transformations practice
  }

\section{2.7 Drui: Review, Wednesday Oct 3}
  \frame
  {
    \frametitle{GQ: How do we transform functions?}
    \framesubtitle{CCSS: HSF.IF.C.7 Analyze functions  \hspace{\stretch{1}}  \alert{2.7}}

    \begin{block}{Do Now: Review handout, function composition, inverses.}
    \end{block}
    Scope p.1-29 \\*[5pt]
    Example exam problems\\ \bigskip
    Homework: Study for exam
    }

\section{2.8 Drui: Test, Thursday Oct 4}
  \frame
  {
    \frametitle{GQ: How do we transform functions?}
    \framesubtitle{CCSS: HSF.IF.C.7 Analyze functions  \hspace{\stretch{1}}  \alert{2.8}}

    \begin{block}{Exam: function composition, inverses.}
    \end{block}
    Scope p.1-29 \\ \bigskip
    Homework: Handout review problems
    }

\section{2.9 Drui: Laptop, Deltamath, Desmos /Word. Tuesday Oct 9}
  \frame
  {
    \frametitle{How do we communicate mathematical results?}
    \framesubtitle{CCSS: MP.4 Model with mathematics \hspace{\stretch{1}} \alert{2.9}}

    \begin{block}{Technical skills needed to communicate mathematics}
    \begin{enumerate}
        \item Word processing: Microsoft Word and equation editor
        \item Computer calculators: Desmos; domain restriction, labeling
        \item Cloud storage: Dropbox
        \item Technical writing standards: MLA format (Purdue OWL)
        \item Writing style: declarative
        \item Assessment criteria: IB exploration criterion \emph{B: Mathematics Presentation}
    \end{enumerate}
    \end{block}
    Lesson: Shared folder structure, graph copy/paste, MLA template\\ \bigskip
    Homework: Deltamath followup. Open textbook online
  }

\section{2.10 Drui: Asymptotes review, Wednesday Oct 10}
  \frame
  {
    \frametitle{GQ: How do we transform functions?}
    \framesubtitle{CCSS: HSF.IF.C.7 Analyze functions  \hspace{\stretch{1}}  \alert{2.10}}

    \begin{block}{Do Now: Asymptotic behavior}
    \begin{enumerate}
        \item Write down the equation of a horizonal line through $(3,2)$
        \item Write down the equation of the axis of symmetry of a parabola having the vertex $(h,k)$
        \item Write down the domain and range of the given parabolic function
        \item Sketch the function $\displaystyle f(x)=\frac{2x-7}{x+1}$, including the asymptotes.
        \item Explain the algebraic features of $f$ leading to the asymptotes
    \end{enumerate}
    \end{block}
    Lesson: Rational functions and graphical analysis \\*[5pt]
    Homework: Complete exercises 1I (?)
  }

\section{2.11 Drui: Transformations, Thursday Oct 11}
  \frame
  {
    \frametitle{GQ: How do we transform functions?}
    \framesubtitle{CCSS: HSF.IF.C.7 Analyze functions  \hspace{\stretch{1}}  \alert{2.11}}

    \begin{block}{Do Now: Translation warmup graphing}
    \begin{enumerate}
        \item 1.6 Investigation \#3 p. 21
        \item Spicy: Graph Exercises 1I \#1a, 1d on the same axes.
        \item Spicy: Answer \#2 and 3
    \end{enumerate}
    \end{block}
    %Review formula sheets\\
    Lesson: Function transformations p 21-24; Exercise 1I p.1-7 \\*[5pt]
    Homework: Complete review exercises p. 25-28
  }

\section{2.12 Drui: Function operations review, Monday Oct 15}
  \frame
  {
    \frametitle{GQ: How do we transform functions?}
    \framesubtitle{CCSS: HSF.IF.C.7 Analyze functions  \hspace{\stretch{1}}  \alert{2.12}}

    \begin{block}{Do Now: Exponent review}
    \begin{enumerate}
      \item $2^3 \cdot 2^2$
      \item $3^6 \div 3^2$
      \item $(5^3)^2$
      \item $\displaystyle \frac{x^2 \cdot x^4}{x^3}$
      \item $x^3 \cdot x^2$
      \item $(ab)^6 \div a^2 b$
      \item $(2m^3)^2$
      \item $\displaystyle x^{-\frac{1}{2}}$
    \end{enumerate}
    \end{block}
    Lesson: Function operations review p 25-28 \\
    Homework: Complete review exercises p. 25-28
  }

\section{2.13 Drui: Laptop, Deltamath, Desmos /Word. Tuesday Oct 16}
  \frame
  {
    \frametitle{How do we communicate mathematical results?}
    \framesubtitle{CCSS: MP.4 Model with mathematics \hspace{\stretch{1}} \alert{2.13}}

    \begin{block}{Technical skills needed to communicate mathematics}
    \begin{enumerate}
        \item Word processing: Microsoft Word and equation editor
        \item Computer calculators: Desmos; domain restriction, labeling
        \item Cloud storage: Dropbox
        \item Technical writing standards: MLA format (Purdue OWL)
        \item Writing style: declarative
        \item Assessment criteria: IB exploration criterion \emph{B: Mathematics Presentation}
    \end{enumerate}
    \end{block}
    Lesson: Shared folder structure, graph copy/paste, MLA template\\ \bigskip
    Homework: Deltamath followup. Open textbook online
  }

\end{document}
