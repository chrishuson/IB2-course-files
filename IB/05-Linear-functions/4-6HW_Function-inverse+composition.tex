\documentclass[12pt, twoside]{article}
\usepackage[letterpaper, margin=1in, headsep=0.5in]{geometry}
\usepackage[english]{babel}
\usepackage[utf8]{inputenc}
\usepackage{amsmath}
\usepackage{amsfonts}
\usepackage{amssymb}
\usepackage{tikz}

\usepackage{pgfplots}
\pgfplotsset{width=9cm,compat=1.9}

\usepackage{venndiagram}

\usepackage{graphicx}
\usepackage{enumitem}
\usepackage{multicol}

\usepackage{fancyhdr}
\pagestyle{fancy}
\fancyhf{}
\renewcommand{\headrulewidth}{0pt} % disable the underline of the header

\fancyhead[LE]{\thepage}
\fancyhead[RO]{\thepage \\ Name: \hspace{4cm} \,\\}
\fancyhead[LO]{BECA / Dr. Huson / IB Mathematics\\* Unit 4: Linear functions and regression\\* 9 January 2020}

\begin{document}
\begin{enumerate}
    \subsubsection*{4.6 Homework: Graphing linear equations}
      %\subsubsection*{Function substitution}
      \item Step by step: Given $f(x)=3x+2$. What is $f(2x-1)$?
        \begin{enumerate}
            \item Perform the substitution, putting $2x-1$ in parenthesis.
            \item Simplify, beginning each line with a leading equals sign if it is equal to the line above.
        \end{enumerate}
      \item Given $f(x)=x^2-1$. Simplify $f(2x-1)$
      \item Given $f(x)=x^3$. Simplify $f(x+1)$
      \item Given $f(x)=4-(2x^2+x)$. Simplify $f(\frac{1}{2}x-3)$
    
      \subsubsection*{Function composition}
      \item Step by step: Given $f(x)=x^2+2$ and $g(x)=x^2$. What is $(f \circ g)(x)$?
      \begin{enumerate}
          \item Rewrite $f \circ g$ and perform the inner substitution (i.e. for $g$): $f(g(x))=f(x^2)$
          \item Perform the substitution, putting $x^2$ in parenthesis (and using a leading equals sign).
          \item Simplify, beginning each line with a leading equals sign.
      \end{enumerate}
      In the following exercises, perform the composition $f \circ g$ and simplify.
      \item Given $f(x)=\frac{1}{2}x^2+1$ and $g(x)=2x$
      \item Given $f(x)=\sqrt{x-4}$ and $g(x)=x^2+4$
      \item Given $\displaystyle f(x)=\frac{1-x}{x^2}+1$ and $g(x)=2x+3$
    
      \subsubsection*{Function operations practice}
      \item Given $f(x)=\frac{1}{2}x^2-2$ and $g(x)=x+2$
      \begin{enumerate}
          \item Find $f + g$
          \item Find $f \times g$
          \item Find $f \div g$
      \end{enumerate}
    
    \newpage
    \subsubsection*{The inverse of a function}
    
      \item Given $f(x)=3x+2$. What is the inverse of the function $f^{-1} (x)$?
    
      \begin{enumerate}
          \item Rewrite the function reversing $x$ and $y$. (assume that $y$ and $f(x)$ are interchangeable)\\*[20pt]
          \item Solve for $x$. Finish by putting $y$ on the left side of the equality.\\*[60pt]
          \item State the answer as $f^{-1} (x)$ equals an expression.\\*[15pt]
      \end{enumerate}
    
      Derive the inverse of each function. Simplify the expression.
      \item   $f(x)=\frac{1}{2}x+2$
      \item   $f(x)=\frac{2}{3}x^2-3$
      \item   $f(x)=\sqrt{x-1}+\frac{1}{2}$
    
      \end{enumerate}
\end{document}