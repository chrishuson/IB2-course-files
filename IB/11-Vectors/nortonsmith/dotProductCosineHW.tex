\documentclass[12pt, twoside]{article}
\usepackage[letterpaper, margin=1in, headsep=0.5in]{geometry}
\usepackage[english]{babel}
\usepackage[utf8]{inputenc}
\usepackage{amsmath}
\usepackage{amsfonts}
\usepackage{amssymb}
\usepackage{tikz}
\usepackage{venndiagram}
\usetikzlibrary{angles, quotes}

\usepackage{graphicx}
\usepackage{enumitem}
\usepackage{multicol}

\usepackage{fancyhdr}
\pagestyle{fancy}
\fancyhf{}
\renewcommand{\headrulewidth}{0pt} % disable the underline of the header

\fancyhead[LE]{\thepage}
\fancyhead[RO]{\thepage \\ Name: \hspace{4cm} \,\\}
\fancyhead[LO]{BECA / Mr. Nortonsmith / IB Mathematics\\* Vector Homework\\* 14 January 2020}

\begin{document}
\begin{enumerate}[itemsep=1cm]
    
    \newpage
    \item Find the measure of the angle between $\vec{a}$ and $\vec{b}$ for each pair of vectors:
    
    
    \begin{enumerate}[itemsep=1cm]
    \item $\vec{a} = \begin{pmatrix} 1 \\ 0 \\ 3 \end{pmatrix}$, \hspace{.4cm}
    $\vec{b} = \begin{pmatrix} 0 \\ 5 \\ 0 \end{pmatrix}$
    
    \item $\vec{a} = \begin{pmatrix} 1 \\ 2 \end{pmatrix}$, \hspace{.4cm}
    $\vec{b} = \begin{pmatrix} 2 \\ 3 \end{pmatrix}$
    
    \item $\vec{a} = \begin{pmatrix} 1 \\ 1 \end{pmatrix}$, \hspace{.4cm}
    $\vec{b} = \begin{pmatrix} -1 \\ 3 \end{pmatrix}$
    
    \item $\vec{a} = \begin{pmatrix} 7 \\ 7 \end{pmatrix}$, \hspace{.4cm}
    $\vec{b} = \begin{pmatrix} -1 \\ 3 \end{pmatrix}$
    \end{enumerate}

    \vspace{1cm}

    Why should knowing the solution to part (c) above make part (d) easy?

    \vspace{2cm}

    \item Let $\vec{a} = \begin{pmatrix} -3 \\ 1 \\ 1 \end{pmatrix}$ and $\vec{b} = \begin{pmatrix} m \\ 1 \\ n \end{pmatrix}$
    
    For a given value of $n$, there is one value for $m$ that will make $\vec{a} \perp \vec{b}$. Write an equation that gives this value for $m$ in terms of $n$. (\emph{hint:} start with $\vec{a} \cdot \vec{b} = 0$)

    \newpage

    \item Consider the triangle with vertices at $(0, 0), (8, 4), (1, 3)$:
    
    \begin{center}
        \begin{tikzpicture}[scale=1.6]
    
            %grid
            \draw[->, thick](0, 0) -- (1, 3) node[midway, above left]{$\vec{a}$};
            \draw[->, thick](0, 0) -- (8, 4) node[midway, above left]{$\vec{b}$};
            \draw[->, thick](1, 3) -- (8, 4);
            
            \coordinate (A) at (1,3);
            \coordinate (B) at (0,0);
            \coordinate (C) at (8,4);
            \draw [-, thick] pic [draw=black, angle radius=9mm, "$\theta$"] {angle = C--B--A};
        \end{tikzpicture}
    \end{center}
    
    \begin{enumerate}
        \item Write $\vec{a}$ and $\vec{b}$ in column-vector notation:

        \vspace{2cm}

        \item Find the magnitude of $\vec{a}$:

        \vspace{2cm}

        \item Find the magnitude of $\vec{b}$:

        \vspace{2cm}
        
        \item Find the angle $\theta$ between $\vec{a}$ and $\vec{b}$:

        \vspace{2cm}

        \item Find the area of the triangle (\emph{hint:} look at IB formula sheet)
    \end{enumerate}

    \newpage

    \item If we know that all of the following statements are true for some vectors $\vec{a}$, $\vec{b}$, $\vec{c}$, and $\vec{d}$:
    \begin{enumerate}
        \item $\vec{a} \parallel \vec{b}$
        \item $\vec{b} \perp \vec{c}$
        \item $\vec{c} = k\vec{d}$ for some $k > 0$
    \end{enumerate}
    
    
    then what must the value of $\vec{a} \cdot \vec{d}$ be?

    \vspace{3cm}

    \item Of the vector concepts covered so far, what topic have you found the most confusing? What do you think would help you understand this concept (more examples, more intuitive explanations, more mathematical derivations, ...)?
    
\end{enumerate}
\end{document}