\documentclass{beamer}
\usepackage{geometry}
\usepackage[english]{babel}
\usepackage[utf8]{inputenc}
\usepackage{amsmath}
\usepackage{amsfonts}
\usepackage{amssymb}
\usepackage{tikz}
\usetikzlibrary{quotes, angles}
\usepackage{graphicx}

%\usepackage{pgfplots}
%\pgfplotsset{width=10cm,compat=1.9}
%\usepackage{pgfplotstable}

\usepackage{fancyhdr}
\pagestyle{fancy}
\setlength{\headheight}{12pt}%doesn't seem to fix warning
\fancyhf{}

%\rhead{\small{2 January 2020}}
\lhead{\small{BECA / Dr. Huson / Geometry Unit 7: Similarity}}

\renewcommand{\headrulewidth}{0pt}

\title{Mathematics Class Slides}
\subtitle{Bronx Early College Academy}
\author{Chris Huson}
\date{28 January 2020}

\begin{document}
\frame{\titlepage}
\section[Outline]{}
\frame{\tableofcontents}

\section{8.1 Circle and volume formulas, Tuesday 28 January}
\frame
{
  \frametitle{GQ: How do we calculate the area and circumference of a circle?}
  \framesubtitle{CCSS: HSG.GMD.A1 Circle formulas for circumference and area \hfill \alert{8.1 Tuesday 28 January}}

  \begin{block}{Do Now: Area and volume problems}
  \begin{itemize}
    \item Area of triangles and parallelograms
    \item Volume formula practice
    \item Circle area and circumference
    \item Circle vocabulary
  \end{itemize}
  \end{block}
  Lesson: Circle formulas \& terminology; \\
  Solids formula notation (start with a label variable, $A$, $V$, $C$, $P$)\\*[5pt]
  Homework: Review reference sheets; Deltamath
}

\section{8.2 Estimating, measuring, scale models, Wednesday 29 January}
\frame
{
  \frametitle{GQ: How do we estimate and work with appropriate precision?}
  \framesubtitle{CCSS: HSG.SRT.GMD.A3 Use volume formulas to solve problems \hfill \alert{8.2 Wednesday 29 January}}

  \begin{block}{Do Now: Area and volume problems}
  \begin{itemize}
    \item Circle area and circumference
    \item Volume formula practice
    \item Circle vocabulary
  \end{itemize}
  \end{block}
  Lesson: Scale drawings; Counting squares to estimate area, rounding\\
  Compound shapes\\*[5pt]
  Homework: Khan Academy volume review and introduction to density (watch video)
}

\section{8.3 Density, Thursday 30 January}
\frame
{
  \frametitle{GQ: How do we apply density ratios to calculate weight?}
  \framesubtitle{CCSS: HSG.MG.A2 Apply concepts of density to model \hfill \alert{8.3 Thursday 30 January}}

  \begin{block}{Do Now: Estimating and rounding problems}
  \begin{itemize}
    \item Scale drawing problems
    \item Area and volume formula practice
    \item Solving in terms of $\pi$ and rounding
    \item Compound shapes
  \end{itemize}
  \end{block}
  Lesson: Density ratios, unit changes, cost calculations\\*[5pt]
  Homework: Khan Academy
}

\section{8.4 Equation of a circle, Friday 31 January}
\frame
{
  \frametitle{GQ: How do we define a circle using analytic geometry?}
  \framesubtitle{CCSS: HSG.GPE.A1 Equation of a circle of given center and radius \hfill \alert{8.4 Friday 31 January}}

  \begin{block}{Do Now Quiz: Area and volume problems\\[0.25cm]
    \alert{Classwork counts double while Dr. Huson is out!}}
  \begin{itemize}
    \item Circle vocabulary
    \item Area and volume formula practice
    \item Solving in terms of $\pi$ and rounding
    \item Compound shapes
  \end{itemize}
  \end{block}
  Lesson: Equation of a circle $(x-a)^2+(y-b)^2=r^2$\\*[5pt]
  Homework: Deltamath due Sunday 10:00pm
}

\section{8.5 Cross sections in 3-dimensions, Monday 3 February}
\frame
{
  \frametitle{GQ: How do we imagine an object cut by a plane?}
  \framesubtitle{CCSS: HSG.GPE.A1 Equation of a circle of given center and radius \hfill \alert{8.5 Monday 3 February}}

  \begin{block}{Do Now Quiz: Area and volume problems\\[0.25cm]
    \alert{Classwork counts double while Dr. Huson is out!}}
  \begin{itemize}
    \item Circle vocabulary
    \item Area and volume formula practice
    \item Solving in terms of $\pi$ and rounding
    \item Compound shapes
  \end{itemize}
  \end{block}
  Lesson: Cross sections in 3-dimensions\\*[5pt]
  Homework: Deltamath due 10:00pm
}

\end{document}

