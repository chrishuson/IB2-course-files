\documentclass{beamer}
\usepackage{geometry}
\usepackage[english]{babel}
\usepackage[utf8]{inputenc}
\usepackage{amsmath}
\usepackage{amsfonts}
\usepackage{amssymb}
\usepackage{tikz}
\usetikzlibrary{quotes, angles}
\usepackage{graphicx}

%\usepackage{pgfplots}
%\pgfplotsset{width=10cm,compat=1.9}
%\usepackage{pgfplotstable}

\usepackage{fancyhdr}
\pagestyle{fancy}
\setlength{\headheight}{12pt}%doesn't seem to fix warning
\fancyhf{}

%\rhead{\small{24 February 2020}}
\lhead{\small{BECA / Dr. Huson / Unit 9: Congruence \& similarity transformations}}

\renewcommand{\headrulewidth}{0pt}

\title{Mathematics Class Slides}
\subtitle{Bronx Early College Academy}
\author{Chris Huson}
\date{24 February 2020}

\begin{document}
\frame{\titlepage}
\section[Outline]{}
\frame{\tableofcontents}

\section{9.1 Triangle congruence theorems, Monday 24 February} 
\frame
{
  \frametitle{GQ: How do we prove two triangles are congruent?}
  \framesubtitle{CCSS: HSG.CO.B6-8 Understand congruence in terms of rigid motions \hfill \alert{9.1 Monday 24 February}}

  \begin{block}{Do Now: Transformations}
  \begin{itemize}
    \item Rigid motions: translation, reflection, rotation
    \item Corresponding angles and lengths
    \item Symmetry in terms of transformations ``onto'' itself
    \item Using the properties of rigid motions in explanations
  \end{itemize}
  \end{block}
  Lesson: Side-side-side Triangle congruence postulate \\
  Corresponding parts of congruent triangles are congruent\\*[5pt]
  Homework: Transformations practice handout
}

\section{9.2 Exam review, Gradescope intro; Tuesday 25 February} 
\frame
{
  \frametitle{GQ: How do we learn from exam results using Gradescope?}
  \framesubtitle{CCSS: HSG.CO.B6-8 Understand congruence in terms of rigid motions \hfill \alert{9.2 Tuesday 25 February}}

  \begin{block}{Do Now: Algebra mastery practice on Deltamath}
    \begin{itemize}
      \item Circle equations (use Casio calculator)
      \item Linear equations of parallel \& perpendicular lines
    \end{itemize}
    \end{block}
    Lesson: Setting up and using Gradescope exam scoring system \\
    Test corrections\\*[5pt]
    Homework: Complete DoNow Deltamath problems (due 10PM); Test corrections due Friday 10PM
}

\section{9.3 Triangle congruence theorems, Wednesday 26 February} 
\frame
{
  \frametitle{GQ: How do we prove two triangles are congruent?}
  \framesubtitle{CCSS: HSG.CO.B6-8 Understand congruence in terms of rigid motions \hfill \alert{9.3 Wednesday 26 February}}

  \begin{block}{Do Now: Rigid motions, translation, reflection, rotation}
    \begin{itemize}
      \item Triangle congruence theorem applications
      \item Compositions of transformations
      \item Justifying congruence based on rigid motion
    \end{itemize}
    \end{block}
    Lesson: Side-side-angle ambiguous case \\
    Corresponding parts of congruent triangles are congruent\\*[5pt]
    Homework: Transformations practice handout
}

\section{9.4 Triangle congruence proofs, Thursday 27 February}
\frame
{
  \frametitle{GQ: How do we prove two triangles are congruent?}
  \framesubtitle{CCSS: HSG.CO.B6-8 Understand congruence in terms of rigid motions \hfill \alert{9.4 Thursday 27 February}}

  \begin{block}{Do Now: Transformations}
    \begin{itemize}
      \item Reflection over a line not an axis
      \item Rotation
      \item Symmetry in terms of transformations ``onto'' itself
      \item Using the properties of rigid motions in explanations
    \end{itemize}
    \end{block}
    Lesson: Proving triangles congruent, two column format \\*[5pt]
    Homework: Deltamath triangle congruence practice
}

\section{9.5 Proof - Proof intro on Deltamath (laptops), Friday 28 February}
\frame
{
  \frametitle{GQ: How do we prove two triangles are congruent?}
  \framesubtitle{CCSS: HSG.CO.B6-8 Understand congruence in terms of rigid motions \hfill \alert{9.5 Friday 28 February}}

  \begin{block}{Do Now: Transformations}
    \begin{itemize}
      \item Rigid motions: translation, reflection, rotation
      \item Corresponding angles and lengths
      \item Symmetry in terms of transformations ``onto'' itself
      \item Using the properties of rigid motions in explanations
    \end{itemize}
    \end{block}
    Extra credit: Practicing proof with Deltamath (laptops) \\
    10.2 Circle equations, Casio use\\*[5pt]
    Homework: Transformations practice handout
}

\section{9.6 Dilation and similarity review, Monday 2 March}
\frame
{
  \frametitle{GQ: How do we calculate dilation ratios?}
  \framesubtitle{CCSS: HSG.CO.B6-8 Understand congruence in terms of rigid motions \hfill \alert{9.6 Monday 2 March}}
  \begin{block}{Do Now: Similarity transformations}
    \begin{itemize}
      \item Dilation scale factor $k$
      \item Dilating segments and their properties
      \item Similarity ratio situations
    \end{itemize}
    \end{block}
    Lesson: Review congruence homework problem set \\*[5pt]
    Homework: Complete dilation and similarity problem set
}

\section{9.7 Dilation and similarity review, Tuesday 3 March}
\frame
{
  \frametitle{GQ: How do we calculate dilation ratios?}
  \framesubtitle{CCSS: HSG.CO.B6-8 Understand congruence in terms of rigid motions \hfill \alert{9.7 Tuesday 3 March}}
  \begin{block}{Do Now: Similarity transformations}
    \begin{itemize}
      \item Similarity ratio situations
      \item Intersecting chords, product format
      \item Sample triangle congruence proofs
    \end{itemize}
    \end{block}
    Lesson: HL congruence theorem; AA \& SAS similarity theorems \\
    Classwork: Deltamath comprehensive review \\*[5pt]
    Homework: Complete Deltamath (proof assignment is optional)
}

\section{9.8 Dilation and similarity review, Thursday 5 March}
\frame
{
  \frametitle{GQ: How do we calculate dilation ratios?}
  \framesubtitle{CCSS: HSG.CO.B6-8 Understand congruence in terms of rigid motions \hfill \alert{9.8 Thursday 5 March}}
  \begin{block}{Do Now: Similarity transformations}
    \begin{itemize}
      \item Similarity ratio situations
      \item Intersecting chords, product format
      \item Sample triangle congruence proofs
    \end{itemize}
    \end{block}
    Lesson: Transformation review \\*[5pt]
    Homework: Study for \alert{exam tomorrow}
}

\section{9.9 Exam: Transformations, Friday 6 March}
\frame
{
  \frametitle{GQ: How do we identify and perform transformations?}
  \framesubtitle{CCSS: HSG.CO.B6-8 Understand congruence in terms of rigid motions \hfill \alert{9.9 Friday 6 March}}
  \begin{block}{Exam: Congruence and similarity transformations}
    \begin{itemize}
      \item Rigid motions: translation, reflection, rotation
      \item Corresponding angles and lengths
      \item Using the properties of rigid motions in explanations
      \item Similarity ratio situations
      \item Intersecting chords, product format
      \item Sample triangle congruence proofs
    \end{itemize}
    \end{block}
    Homework: Afterschool session to complete trimester assignments
}
\end{document}

