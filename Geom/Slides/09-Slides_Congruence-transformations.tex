\documentclass{beamer}
\usepackage{geometry}
\usepackage[english]{babel}
\usepackage[utf8]{inputenc}
\usepackage{amsmath}
\usepackage{amsfonts}
\usepackage{amssymb}
\usepackage{tikz}
\usetikzlibrary{quotes, angles}
\usepackage{graphicx}

%\usepackage{pgfplots}
%\pgfplotsset{width=10cm,compat=1.9}
%\usepackage{pgfplotstable}

\usepackage{fancyhdr}
\pagestyle{fancy}
\setlength{\headheight}{12pt}%doesn't seem to fix warning
\fancyhf{}

%\rhead{\small{24 February 2020}}
\lhead{\small{BECA / Dr. Huson / Geometry Unit 9: Congruence transformations}}

\renewcommand{\headrulewidth}{0pt}

\title{Mathematics Class Slides}
\subtitle{Bronx Early College Academy}
\author{Chris Huson}
\date{24 February 2020}

\begin{document}
\frame{\titlepage}
\section[Outline]{}
\frame{\tableofcontents}

\section{9.1 Triangle congruence theorems, Monday 24 February} 
\frame
{
  \frametitle{GQ: How do we prove two triangles are congruent?}
  \framesubtitle{CCSS: HSG.CO.B6-8 Understand congruence in terms of rigid motions \hfill \alert{9.1 Monday 24 February}}

  \begin{block}{Do Now: Transformations}
  \begin{itemize}
    \item Rigid motions: translation, reflection, rotation
    \item Corresponding angles and lengths
    \item Symmetry in terms of transformations ``onto'' itself
    \item Using the properties of rigid motions in explanations
  \end{itemize}
  \end{block}
  Lesson: Side-side-side Triangle congruence postulate \\
  Corresponding parts of congruent triangles are congruent\\*[5pt]
  Homework: Transformations practice handout
}

\section{9.2 Volume exam review, Gradescope intro; Tuesday 25 February} 
\frame
{
  \frametitle{GQ: How do we learn from exam results using Gradescope?}
  \framesubtitle{CCSS: HSG.CO.B6-8 Understand congruence in terms of rigid motions \hfill \alert{9.2 Tuesday 25 February}}

  \begin{block}{Do Now: Transformations}
    \begin{itemize}
      \item Rigid motions: translation, reflection, rotation
      \item Corresponding angles and lengths
      \item Symmetry in terms of transformations ``onto'' itself
      \item Using the properties of rigid motions in explanations
    \end{itemize}
    \end{block}
    Lesson: Setting up and using Gradescope exam scoring system \\
    Test corrections\\*[5pt]
    Homework: Complete selected exam review problems on Deltamath
}

\section{9.3 Triangle congruence theorems, Wednesday 26 February} 
\frame
{
  \frametitle{GQ: How do we prove two triangles are congruent?}
  \framesubtitle{CCSS: HSG.CO.B6-8 Understand congruence in terms of rigid motions \hfill \alert{9.3 Wednesday 26 February}}

  \begin{block}{Do Now: Transformations}
    \begin{itemize}
      \item Rigid motions: translation, reflection, rotation
      \item Corresponding angles and lengths
      \item Symmetry in terms of transformations ``onto'' itself
      \item Using the properties of rigid motions in explanations
    \end{itemize}
    \end{block}
    Lesson: Side-side-side Triangle congruence postulate \\
    Corresponding parts of congruent triangles are congruent\\*[5pt]
    Homework: Transformations practice handout
}


\section{9.4 Triangle congruence theorems, Thursday 27 February}
\frame
{
  \frametitle{GQ: How do we prove two triangles are congruent?}
  \framesubtitle{CCSS: HSG.CO.B6-8 Understand congruence in terms of rigid motions \hfill \alert{9.4 Thursday 27 February}}

  \begin{block}{Do Now: Transformations}
    \begin{itemize}
      \item Rigid motions: translation, reflection, rotation
      \item Corresponding angles and lengths
      \item Symmetry in terms of transformations ``onto'' itself
      \item Using the properties of rigid motions in explanations
    \end{itemize}
    \end{block}
    Lesson: Side-side-side Triangle congruence postulate \\
    Corresponding parts of congruent triangles are congruent\\*[5pt]
    Homework: Transformations practice handout
}

\section{9.5 Triangle congruence theorems, Friday 28 February}
\frame
{
  \frametitle{GQ: How do we prove two triangles are congruent?}
  \framesubtitle{CCSS: HSG.CO.B6-8 Understand congruence in terms of rigid motions \hfill \alert{9.5 Friday 28 February}}

  \begin{block}{Do Now: Transformations}
    \begin{itemize}
      \item Rigid motions: translation, reflection, rotation
      \item Corresponding angles and lengths
      \item Symmetry in terms of transformations ``onto'' itself
      \item Using the properties of rigid motions in explanations
    \end{itemize}
    \end{block}
    Lesson: Side-side-side Triangle congruence postulate \\
    Corresponding parts of congruent triangles are congruent\\*[5pt]
    Homework: Transformations practice handout
}


\end{document}

