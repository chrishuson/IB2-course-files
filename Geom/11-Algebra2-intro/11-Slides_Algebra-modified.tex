\documentclass{beamer}
\usepackage{geometry}
\usepackage[english]{babel}
\usepackage[utf8]{inputenc}
\usepackage{amsmath}
\usepackage{amsfonts}
\usepackage{amssymb}
\usepackage{tikz}
\usetikzlibrary{quotes, angles}
\usepackage{graphicx}

\usepackage{multicol}
%\usepackage{pgfplots}
%\pgfplotsset{width=10cm,compat=1.9}
%\usepackage{pgfplotstable}

\usepackage{fancyhdr}
\pagestyle{fancy}
\setlength{\headheight}{12pt}%doesn't seem to fix warning
\fancyhf{}

%\rhead{\small{24 February 2020}}
\lhead{\small{BECA / Huson \& Kaplan / Unit 11: Algebra competencies}}

\renewcommand{\headrulewidth}{0pt}

\title{Mathematics Class Slides}
\subtitle{Bronx Early College Academy}
\author{Chris Huson}
\date{28 April 2020}

\begin{document}
\frame{\titlepage}
\section[Outline]{}
\frame{\tableofcontents}


\section{11.2 Literals, operations on radicals Tuesday 28 April} 
\frame
{
  \frametitle{GQ: How do we apply algebra to equations with literals?}
  \framesubtitle{HSA.CED.A.4 Rearrange formulas to highlight a quantity of interest \hfill \alert{11.2 Tuesday 28 April}}

  \begin{block}{Do Now: Deltamath (remember to submit ``Present'')}
    \begin{itemize}
      \item Convert standard linear equations to $y$-intercept form
    \end{itemize}

    \end{block}
    
    Lesson: Operations on radicals (square roots)\\
    Collecting like terms\\[0.25cm]
    Pear Deck practice problems \\[0.25cm]
    Exit note: stay after if you need help
}

\frame
{
  \frametitle{Take notes: Properties of square roots}

  \Large{
  Definition: $(\sqrt{a})^2=a$ \hfill note: $(-\sqrt{a})^2=a$ \\[1cm]
  example: \\ $x^2=25$ \\
  $x=\sqrt{25}=5$ \\[0.5cm] 
  check: $5^2=25$ \hfill but also $(-5)^2=25$\\[0.5cm]
  \hfill so, if $x^2=25$ \\ \hfill then $x=\pm 5$ \vspace{3cm}
  
}
}

\frame
{
  \frametitle{Addition of square roots: collect like terms}

  \Large{
  Addition \\
  $\sqrt{b}+\sqrt{b}=2\sqrt{b}$, \hfill but $\sqrt{a}+\sqrt{b}=\sqrt{a}+\sqrt{b}$ \\[1cm]
  examples\\
  $2\sqrt{3}+4\sqrt{3}=6\sqrt{3}$\\[1cm]
  $5{\color{red}\sqrt{7}}+2\sqrt{11}+3{\color{red}\sqrt{7}}=8{\color{red}\sqrt{7}}+2\sqrt{11}$\\[1cm]
  
}
}

\frame
{
  \frametitle{Multiplying and factoring square roots}
    \Large{
    Multiplication\\
    $\sqrt{c} \times \sqrt{d}=\sqrt{cd}$\\[0.5cm]
    examples\\
    $\sqrt{3} \times \sqrt{12}=\sqrt{3 \times 12}= \sqrt{36}$\\[2cm]
    $\sqrt{20}=\sqrt{4} \times \sqrt{5}=2\sqrt{5}$\\[1.5cm]
}
}

\frame
{
  \frametitle{Dividing square roots}
    \Large{
    Division, the multiplicative inverse (reciprocal)\\
    $\displaystyle \sqrt{\frac{j}{k}}= \frac{\sqrt{j}}{\sqrt{k}}$\\[0.5cm]
    example\\[0.5cm]
    $\displaystyle \sqrt{\frac{4}{9}}= \frac{\sqrt{4}}{\sqrt{9}}= \frac{2}{3}$\\[2.5cm]
}
}

\frame
{
  \frametitle{Practice with radicals and literals}
  \Large{
  Simplify the expression by ``collecting like terms''\\[0.5cm]
      $\sqrt{5}-x+6\sqrt{5}+2x$ \vspace{4cm}
  }
}

\frame
{
  \frametitle{Practice with radicals and literals}
  \Large{
  Solve for $x$. Start by ``collecting like terms''\\[0.5cm]
      $4x+6\sqrt{3}-2x-2\sqrt{3}=10\sqrt{3}$ \vspace{5cm}
  }
}

\frame
{
  \frametitle{Practice with radicals and literals}
  \Large{
  Simplify the expression by ``collecting like terms''\\[0.5cm]
      $5+2\sqrt{13}-3+3\sqrt{13}$ \vspace{4cm}
  }
}

\frame
{
  \frametitle{Practice with radicals and literals}
  \Large{
    Simplify each expression by factoring and then simplifying a perfect square\\[0.5cm]
      $\sqrt{18}$ \vspace{4cm}
  }
}

\frame
{
  \frametitle{Practice with radicals and literals}
  \Large{
    Simplify each expression by factoring and then simplifying a perfect square\\[0.5cm]
      $\sqrt{50}$ \vspace{4cm}
  }
}

\frame
{
  \frametitle{Practice with radicals and literals}

  Simplify each expression by factoring and then simplifying a perfect square
  
  \begin{enumerate}
    \begin{multicols}{2}
      \item $3x+2x$
      \begin{itemize}
        \item[$\square$] $5+x$
        \item[$\square$] $(x+x+x)+(x+x)$
        \item[$\square$] $5x$
        \item[$\square$] $(3+2)x$
      \end{itemize}
      \item $5\pi-2\pi+4\pi$
      \begin{itemize}
        \item[$\square$] $3\pi+4$
        \item[$\square$] $(5-2+4)\pi$
        \item[$\square$] $7+\pi$
        \item[$\square$] $7 \times \pi$
      \end{itemize}
    \end{multicols}
    \end{enumerate} \vspace{5cm}
}

\frame
{
  \frametitle{Practice with radicals and literals}

  Simplify each expression by ``collecting like terms''
\Large{  
  \begin{enumerate}%[itemsep=2cm]
    \begin{multicols}{2}
      \item $3x-2x+7y$ \vspace{2cm}
      \item $5z+5\pi-2\pi+z$
      \item $-k+7\sqrt{2}+2k+3\sqrt{2}$ \vspace{2cm}
      \item $5\pi x-2 \pi x +9y$
    \end{multicols}
    \end{enumerate} \vspace{2cm}
}}


\frame
{
  \frametitle{GQ: How do we apply algebra to equations with literals?}
  \framesubtitle{HSA.CED.A.4 Rearrange formulas to highlight a quantity of interest \hfill \alert{11.1 Wed. 22 April}}
  \Large{
  Solve each equation for the unknown
  
  \begin{enumerate}
    \begin{multicols}{2}
      \item $\displaystyle \frac{k}{\sqrt{3}}=11$
      \item $5z-2 \pi = 4\pi +z$
    \end{multicols}
    \end{enumerate} \vspace{7cm}
}}

\frame
{
  \frametitle{GQ: How do we apply algebra to equations with literals?}
  \framesubtitle{HSA.CED.A.4 Rearrange formulas to highlight a quantity of interest \hfill \alert{11.1 Wed. 22 April}}
  \Large{
  Solve each equation for the unknown
  
  \begin{enumerate}%[itemsep=2cm]
    \begin{multicols}{2}
      \item $4x-x\sqrt{3}=11$
      \item $5\pi x-2 \pi x= \pi x +14$
    \end{multicols}
    \end{enumerate} \vspace{7cm}
}}


\end{document}



